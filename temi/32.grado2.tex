\section{Modelli di secondo grado}


\subsection{Equazioni}
\begin{questions}


\question
\exonly{Risolvi le seguenti equazioni }

\begin{multicols}{2}
	\begin{parts}
		\setlength\itemsep{4mm}
		\part \exonly{$ 3x^2-5x=0 $	 }			
		\part \exonly{$ 3 x^2-5 =x^2+7$ }
		\part \exonly{$ 3(x^2-2)=5-x^2$ }
		\part \exonly{$ 2x(x-3)=(2-x)x $ }
		\part \exonly{$ 5x^2-3x-4=(x+2)(x-2) $ }
		\part \exonly{$ (x+3)(x+2)=2(x-3)(x-1)$ }
		\part \exonly{ $ 3x^2-5x=0 $ }
		\part \exonly{$ 2x^2+3x=0 $ }
		\part \exonly{ $ x^2-3x+2=-3(x-4) $ }
		\part \exonly{$ (2x+1)(x+2)=(1-x)(x-4) $ }
	\end{parts}
\end{multicols}


\question

\exonly{ Risolvi le seguenti equazioni: }
\begin{multicols}{2}
	\begin{parts}
		\setlength\itemsep{4mm}
		\part \exonly{ $ 2x^2-3x+5=0 $ }
		\part \exonly{$ 2x+1-3x^2=2x^2-5x+1 $ }
		\part \exonly{$ (x+3)(2x-5)=(3x+1)(2-x) $ }
		\part \exonly{$ \frac{2}{3}x^2+\frac{1}{4}x-\frac{1}{2}=0 $ }
		\part \exonly{ $ (\frac{1}{2}x+\frac{1}{3})(x-2)=\frac{2}{3}(x+\frac{1}{3})(x+\frac{2}{3}) $ }
		\part \exonly{$ \frac{2}{3}x^2-\frac{3}{2}x+\frac{1}{4}= \frac{1}{5}x^2-\frac{1}{3}x-\frac{2}{5} $ }
	\end{parts}
\end{multicols}

\end{questions}

\subsection{Funzioni e rappresentazione grafica}

\begin{questions}	

\question
	
\exonly{
	Data la parabola di equazione:
	
	$$y=2x^ 2 -4x-11$$
}
	
	\begin{parts}
		\part
			\exonly{ Calcolare le coordinate del vertice }
			\solonly{$V(1,-13)$ }
			
		\part
			\exonly{Calcolare i punti di intersezione della parabola con l'asse delle ascisse. }
			\solonly{ $I_x=\{(3.55;0),(-1.55;0)\}$}
		\part
			\exonly{Calcolare i punti di intersezione della parabola con l'asse delle ordinate. }	
			\solonly{$I_y(0,-11)$ }
	\end{parts}


	
	\question
\exonly{	
	Calcolare le coordinate $x_A$ e $y_B$ tali per cui i punti $A(x_A;-5)$ e $B(-8;y_B)$ appartengano alla parabola di equazione 

	
	$$y= -\dfrac{1}{2}x^2-6x-\dfrac{21}{2}$$ }
\solonly{ 	$x_A=\{-11,-1\}$
	
	$y_B=\frac{11}{2}$}
	

	
	\question
	
	
\exonly{
	Calcolare i punti di intersezione tra la parabola di equazione
	
	$$y= \dfrac{1}{4}x^2+4x+6$$
	
	e la retta di equazione
	
	$$y= 2x +3$$ }
	
	\solonly{$	S=\{(-2;-1),(-6;-9)\}$}
	
	
%	\question
%	
%\exonly{	Calculez les points d'intersection de la parabole 
%	
%	$$y= \dfrac{1}{2}x^2+7x-3$$
%	
%	avec la parabole
%	
%	$$y= -\dfrac{5}{2}x^2+4x+3$$ }
%	
%	\rsol{	$S=\{(1;\frac{9}{2}),(-2;-15)\}$}
%	
\question

	\exonly{
	Determinare il parametro $b$ tale per cui la parabola di equazione $y=x^2+b \cdot x +4$  passi per il punto $A(4;4)$. }
	
\solonly{$b=-4$ , $y=x^2-4x+4$}

\question
\exonly{Determinare l’equazione di una parabola passante per i seguenti punti: $(2; 3)$, $(-1; 6)$ e $(4; 21)$. }
\solonly{$x=2x^2-3x+1$ }

\question

\exonly{

La somma di due numeri è $36$. Determinare questi due numeri sapendo che il loro
prodotto è massimo. }

\solonly{$18$ e $18$ }

%\exnewpage
\question
\exonly{
Un’azienda fabbrica un certo oggetto la cui domanda $x$ è data dalla funzione
$x = 10200 - 300p$ dove $p$ rappresenta il possibile prezzo di vendita  dell'oggetto.

I costi fissi ammontano a \SI{14400}{\CHF} e i costi variabili a \SI{8}{\CHF}  per oggetto prodotto. 
}

\begin{parts}
	\part
	\exonly{Determinare la funzione dei costi $C(p)$ in funzione del prezzo $p$.  }
	\solonly{$C(p)=-\num{2400}p+\num{96000}$ }
	\part
	\exonly{Determinare la funzione dei ricavi $R(p)$ in funzione del prezzo $p$.  }
	\solonly{$R(p)=-300p^2+\num{10200}$ }
	\part
	\exonly{Determinare la funzione del guadagno (beneficio) $G(p)$ in funzione del prezzo $p$.  }
	\solonly{$G(p)=-300p^2+\num{12600}p -\num{96000}$ }
	\part
	\exonly{ Determinare il prezzo $p$ che massimizzi il guadagno. Di seguito determinare il guadagno massimo.}
	\solonly{Prezzo che massimizza il guadagno \SI{21}{\CHF}. Guadagno massimo \SI{36300}{\CHF} }

	
\end{parts}



\question
\exonly{
	Rappresenta graficamente le seguenti funzioni. Trova le intersezioni con gli assi e le coordinate del vertice }

	\begin{multicols}{2}
		\begin{parts}
			\setlength\itemsep{4mm}
			\part 
			\exonly{$\begin{aligned}[t]
			f:\mathbb{R} & \longrightarrow\mathbb{R}\\
			x & \longmapsto 3x^2-5			
			\end{aligned}$ }
			\part 
		\exonly{	$
			\begin{aligned}[t]
			f:\mathbb{R} & \longrightarrow\mathbb{R}\\
			x & \longmapsto -\frac{1}{4}x^2-1	
			\end{aligned}$ }
			\part 
			\exonly{$
			\begin{aligned}[t]
			f:\mathbb{R} & \longrightarrow\mathbb{R}\\
			x & \longmapsto 2x^2+3x	
			\end{aligned}$ }
			\part
			\exonly{ $
			\begin{aligned}[t]
			f:\mathbb{R} & \longrightarrow\mathbb{R}\\
			x & \longmapsto -\frac{1}{2}x^2+5x	
			\end{aligned}$ }
			\part 
			\exonly{$
			\begin{aligned}[t]
			f:w\mathbb{R} & \longrightarrow\mathbb{R}\\
			x & \longmapsto x^2-4x-32	
			\end{aligned}$ }
			\part 
			\exonly{$
			\begin{aligned}[t]
			f:\mathbb{R} & \longrightarrow\mathbb{R}\\
			x & \longmapsto 2x^2-5x-3	
			\end{aligned}$ }
		\end{parts}
	\end{multicols}




\question

\exonly{	Sono date le seguenti funzioni: 
	\begin{multicols}{2}
		\begin{itemize}
			\setlength\itemsep{4mm}
			\item $
			\begin{aligned}[t]
			f:\mathbb{R} & \longrightarrow\mathbb{R}\\
			x & \longmapsto 2x^2+3x-5			
			\end{aligned}$
			\item $
			\begin{aligned}[t]
			g:\mathbb{R} & \longrightarrow\mathbb{R}\\
			x & \longmapsto -\frac{1}{4}x^2-\frac{3}{5}x+\frac{2}{3}
			\end{aligned}$
		\end{itemize}
	\end{multicols}
}
	\begin{parts}
		\part \exonly{Trova le intersezioni di $f$ e di $g$ con gli assi.  }
		\part \exonly{Trova le intersezioni di $f$ con $g$. }
		\part \exonly{Trova le coordinate dei due vertici. }
	\end{parts}
	

 
\end{questions}	