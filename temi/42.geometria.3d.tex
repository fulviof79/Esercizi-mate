\section{Geometria dello spazio}
\subsection{Prismi e piramidi}
\begin{questions}

	\begin{qblock}
		\question
		\exonly{
			Consideriamo la  piramide di base $ABC$ e vertice $S$.
			I lati  $SA$, $SB$ e $SC$ sono perpendicolari tra loro e misurano rispettivamente \SI{1}{\metre}, \SI{2}{\metre}  e \SI{3}{\metre}.}


		\ifprintanswers   \else
		
		\begin{center}
				\begin{tikzpicture}
					\draw
					(0,0) coordinate [label=left:$A$] (A)
					(4,0) coordinate [label=right:$C$]  (C) --
					(2,-1) coordinate [label=below:$B$]  (B) -- (A) --
					(1,2) coordinate [label=above:$S$]  (S) edge (C)	 edge (B);
					\draw[dashed] (A) -- (C);
				\end{tikzpicture}
		\end{center}
		\fi


		\begin{parts}
			\part
			\exonly{
				Determinare l'area della base $ABC$.
				%	Déterminer l’aire de la base $ABC$ de cette pyramide.
			}
			\solonly{\SI{3.5}{\square\metre}}
			\part
			\exonly{
				Determinare il volume di questa piramide.
				%	Déterminer le volume de cette pyramide.
			}
			\solonly{\SI{1}{\cubic\metre}}
		\end{parts}
	\end{qblock}

	\begin{qblock}
		\question
		\exonly{
			I vertici di un cubo di lato \SI{10}{\centi\metre} vengono tagliati affinché le facce del cubo diventino degli ottagoni regolari.
		}

		\begin{parts}
			\part
			\exonly{
				Disegnare uno schizzo della situazione.
				%	Faire un croquis de la situation.
			}

			\ifprintanswers
				\begin{tikzpicture}[baseline={($(current bounding box.north)-(0,1.6ex)$)},scale=0.6]
					%\tikzstyle{isometric}=[x={(0.710cm,-0.410cm)},y={(0cm,0.820cm)},z={(-0.710cm,-0.410cm)}]
					\tikzset{every path/.style={isometric}}
					\tikzset{face/.style={fill=gray!20}}
					\pgfmathsetmacro{\cubex}{4}
					\pgfmathsetmacro{\cubey}{4}
					\pgfmathsetmacro{\cubez}{4}
					\pgfmathsetmacro{\ratio}{1/3}

					\coordinate (FTR) at (0,0,0);
					\coordinate (FTL) at (-\cubex,0,0);
					\coordinate (FBL) at (-\cubex,-\cubey,0);
					\coordinate (FBR) at (0,-\cubey,0);
					\coordinate (BTR) at (0,0,-\cubez);
					\coordinate (BBR) at (0,-\cubey,-\cubez);
					\coordinate (BTL) at (-\cubex,0,-\cubez);
					\coordinate (BBL) at (-\cubex,-\cubey,-\cubez);


					\draw[dotted] (FBL) -- (BBL);

					\shadedraw[face] ($(FTR)!\ratio!(FTL)$) -- ($(FTL)!\ratio!(FTR)$) -- ($(FTL)!\ratio!(FBL)$) -- ($(FBL)!\ratio!(FTL)$) --
					($(FBL)!\ratio!(FBR)$) -- ($(FBR)!\ratio!(FBL)$) --
					($(FBR)!\ratio!(FTR)$) -- ($(FTR)!\ratio!(FBR)$) --cycle;
					%top face
					\shadedraw[face] ($(FTR)!\ratio!(FTL)$) -- ($(FTR)!\ratio!(BTR)$) -- ($(BTR)!\ratio!(FTR)$) -- ($(BTR)!\ratio!(BTL)$) --
					($(BTR)!\ratio!(BTL)$) -- ($(BTL)!\ratio!(BTR)$) --
					($(BTL)!\ratio!(FTL)$) --  ($(FTL)!\ratio!(BTL)$)
					-- ($(FTL)!\ratio!(FTR)$) --cycle;
					%right face
					\shadedraw[face] ($(FTR)!\ratio!(BTR)$) -- ($(BTR)!\ratio!(FTR)$) -- ($(BTR)!\ratio!(BBR)$) -- ($(BBR)!\ratio!(BTR)$) --
					($(BBR)!\ratio!(FBR)$) -- ($(FBR)!\ratio!(BBR)$) --
					($(FBR)!\ratio!(FTR)$) --  ($(FTR)!\ratio!(FBR)$)
					-- ($(FTR)!\ratio!(FBR)$) -- cycle;

					%top right corner
					\shadedraw[face] ($(FTR)!\ratio!(BTR)$) -- ($(FTR)!\ratio!(FTL)$) -- ($(FTR)!\ratio!(FBR)$) --cycle;
					%top left corner
					\shadedraw[face] ($(FTL)!\ratio!(BTL)$) -- ($(FTL)!\ratio!(FTR)$) -- ($(FTL)!\ratio!(FBL)$) --cycle;

					%bottom right corner
					\shadedraw[face] ($(FBR)!\ratio!(FTR)$) -- ($(FBR)!\ratio!(FBL)$) -- ($(FBR)!\ratio!(BBR)$) --cycle;

					%top right corner
					\shadedraw[face] ($(BTR)!\ratio!(FTR)$) -- ($(BTR)!\ratio!(BTL)$) -- ($(BTR)!\ratio!(BBR)$) --cycle;
					%\fill[red] (0,0,0) circle (2pt);
					\draw[dotted] (FTR)  -- (FTL)  -- (FBL)  -- (FBR)  -- cycle;
					\draw[dotted] (FTR) -- (BTR)  -- (BTL)  -- (FTL) -- cycle;
					\draw[dotted] (FTR) -- (FBR) -- (BBR)  -- (BTR) -- cycle;
				\end{tikzpicture}
			\fi
			\part
			\exonly
			{
				Calcolare l'area esterna totale del solido ottenuto.
				%	Calculer l’aire extérieure totale du corps ainsi obtenu
			}
			\solonly{\SI{556.42}{\square\centi\metre}}
			\part
			\exonly{
				Calcolarne il volume.
				%	Calculer son volume.
			}
			\solonly{\SI{966}{\cubic\centi\metre}}
		\end{parts}
	\end{qblock}

	\begin{qblock}
		\question
		\exonly{
			Calcolare l'altezza della piramide retta a base rettangolare $ABCD$ il cui vertice é $S$.
			Le misure seguenti sono conosciute: $AB$=\SI{20}{\centi\metre}, $BC$=\SI{23}{\centi\metre} e $AS$=\SI{32}{\centi\metre}.
		}

		\ifprintanswers   \else
			
			\begin{center}
				\begin{tikzpicture}[scale=0.7]
					\coordinate[label=left:$A$] (A) at (0,0);
					\coordinate[label=right:$B$] (B) at (4,0);
					\coordinate[label=below:$C$] (C) at ($(B)+(35:3)$);
					\coordinate[label=below:$D$] (D) at ($(A)+(35:3)$);
					\draw (A)-- (B)--(C)--(D)-- cycle;
					\draw (3,5) coordinate [label=above:$S$]  (S)  edge (A) edge (D) edge (C)	 edge (B);
	
				\end{tikzpicture}
			\end{center}
		\fi
		\solonly{\SI{28.14}{\centi\metre}}
	\end{qblock}

	\begin{qblock}
		\question
		\exonly{
			Da uno degli spigoli di un cubo tagliamo un tetraedro i cui spigoli misurano  $AP=AQ=$\SI{3}{\centi\metre} e $AR=$\SI{4}{\centi\metre}.
		}

		
		\begin{center}
			\begin{tikzpicture}[baseline={($(current bounding box.north)-(0,1.6ex)$)}]
				%\tikzstyle{isometric}=[x={(0.710cm,-0.410cm)},y={(0cm,0.820cm)},z={(-0.710cm,-0.410cm)}]
				%\tikzset{every path/.style={isometric}}
				\tikzset{face/.style={fill=gray!20}}
				\pgfmathsetmacro{\cubex}{2}
				\pgfmathsetmacro{\cubey}{2}
				\pgfmathsetmacro{\cubez}{2}
				\pgfmathsetmacro{\ratio}{1/3}
	
				\coordinate (FTR) at (0,0,0);
				\coordinate[label=left:$A$] (FTL) at (-\cubex,0,0);
				\coordinate (FBL) at (-\cubex,-\cubey,0);
				\coordinate (FBR) at (0,-\cubey,0);
				\coordinate (BTR) at (0,0,-\cubez);
				\coordinate (BBR) at (0,-\cubey,-\cubez);
				\coordinate (BTL) at (-\cubex,0,-\cubez);
				\coordinate (BBL) at (-\cubex,-\cubey,-\cubez);
				\coordinate[label=left:$R$] (R) at ($(FTL)!0.6!(BTL)$);
				\coordinate[label=left:$Q$] (Q) at ($(FTL)!0.6!(FBL)$);
				\coordinate[label=right:$P$] (P) at ($(FTL)!0.6!(FTR)$);
	
				\draw[dotted] (FTR)  -- (FTL)  -- (FBL)  -- (FBR)  -- cycle;
				\draw[dotted] (FTR) -- (BTR)  -- (BTL)  -- (FTL) -- cycle;
				\draw[dotted] (FTR) -- (FBR) -- (BBR)  -- (BTR) -- cycle;
				\draw[very thick] (FTL) -- (R) -- (P) -- (Q) --cycle;
				\draw[very thick]  (FTL) -- (Q) -- (P);
				\draw[very thick, dashed]  (R) -- (Q);
			\end{tikzpicture}
		\end{center}

		\begin{parts}
			\part
			\exonly{
				Calcolare il volume del tetraedro.
				%Calculer le volume du tétraèdre.
			}
			\solonly{\SI{6}{\cubic\centi\metre}}

			\part
			\exonly{
				Calcolare l'altezza del tetraedro rispetto alla base $PQR$.
				%Calculer la hauteur du tétraèdre relative à la base $PQR$.
			}
			\solonly{\SI{1.9}{\centi\metre}}
		\end{parts}
	\end{qblock}

	\begin{qblock}
		\question
		\exonly{
			Un prisma retto a base quadrata di lato $b=\SI{2}{\centi\metre}$ é inscritto in una piramide regolare a base quadrata di lato $a=\SI{6}{\centi\metre}$ e altezza $h=\SI{10}{\centi\metre}$.

		}


		\begin{parts}
			\part
			\exonly{
				Eseguire uno schizzo della situazione.
				%Faire un croquis de la situation.
			}
			\ifprintanswers
				\begin{tikzpicture}[baseline={($(current bounding box.north)-(0,1.6ex)$)},scale=0.4]
					%\tikzstyle{isometric}=[x={(0.710cm,-0.410cm)},y={(0cm,0.820cm)},z={(-0.710cm,-0.410cm)}]
					%\tikzset{every path/.style={isometric}}
					\tikzset{face/.style={fill=gray!20}}
					\pgfmathsetmacro{\cubex}{4}
					\pgfmathsetmacro{\cubey}{4}
					\pgfmathsetmacro{\cubez}{4}
					\pgfmathsetmacro{\ratio}{1/2}
					\pgfmathsetmacro{\shift}{-3.5cm}

					\coordinate (S) at (0,7,0);
					\coordinate (FL) at (-\cubex,0,\cubez);
					\coordinate (FR) at (\cubex,0,\cubez);
					\coordinate (BR) at (\cubex,0,-\cubez);
					\coordinate (BL) at (-\cubex,0,-\cubez);
					\coordinate (PFTR) at ($(FR)!\ratio!(S)$);
					\coordinate (PFTL) at ($(FL)!\ratio!(S)$);
					\coordinate (PBTL) at ($(BL)!\ratio!(S)$);
					\coordinate (PBTR) at ($(BR)!\ratio!(S)$);
					\coordinate (PFBR) at ([yshift=\shift] PFTR);
					\coordinate (PFBL) at ([yshift=\shift] PFTL);
					\coordinate (PBBL) at ([yshift=\shift] PBTL);
					\coordinate (PBBR) at ([yshift=\shift] PBTR);


					\draw (FL) -- (FR) -- (BR);
					\draw[dotted] (BR) -- (BL) --(FL);
					\draw (S) edge (FR)  edge (FL) edge (BR);
					\draw[dotted] (S)  edge (BL) ;

					\draw[thick] (PFTL) -- (PFTR) -- (PBTR) --(PBTL) -- cycle;
					\draw[thick] (PFBR) -- (PFBL) -- (PBBL) -- (PBBR) -- cycle;
					\draw[thick] (PFTL) -- (PFBL) (PFTR) -- (PFBR) (PBTL) --(PBBL) (PBTR)--(PBBR);

				\end{tikzpicture}
			\fi
			\part
			\exonly{
				Calcolare il volume del prisma.
				%Calculer le volume du prisme.

			}
			\solonly{\SI{26.67}{\cubic\centi\metre}}
		\end{parts}
	\end{qblock}

	\begin{qblock}
		\question
		\exonly{
			I vertici di un tetraedro di lato \SI{12}{\centi\metre} vengono tagliati affinché le facce del tetraedro divengano degli esagoni regolari.

			Determinare il volume del solido ottenuto.
			%Déterminer le volume du corps ainsi obtenu.
		}
		\solonly{\SI{173.48}{\cubic\centi\metre}}

		
		\begin{center}
			\begin{tikzpicture}[scale=2]
				\tikzset{every path/.style={isometric}}
				\pgfmathsetmacro{\a}{2/ sqrt(3)}
				\pgfmathsetmacro{\b}{sqrt(3)/2}
				\pgfmathsetmacro{\h}{sqrt(23/4)}
				\pgfmathsetmacro{\ratio}{1/3}
				\pgfmathsetmacro{\Ratio}{2/3}
	
				\coordinate (A) at ( 0,0,\a);
				\coordinate (B) at ( 1,0,-\b);
				\coordinate (C) at ( -1,0,-\b);
				\coordinate (S) at ( 0,\h,0);
				\draw[dotted] (A)--(B);
	
				\draw[dotted] (C) edge (B) edge (A);
				\draw[dotted] (S) edge (A) edge (B);
				\draw [dotted]  (S) edge (C);
	
				\shadedraw[dashed] ($(A)!\Ratio!(S)$) -- ($(B)!\Ratio!(S)$) -- ($(C)!\Ratio!(S)$) -- cycle;
				\draw ($(A)!\Ratio!(S)$) -- ($(B)!\Ratio!(S)$) ;
	
				\shadedraw[dashed] ($(A)!\ratio!(S)$) -- ($(A)!\ratio!(B)$) -- ($(A)!\ratio!(C)$) --cycle;
				\draw ($(A)!\ratio!(S)$) -- ($(A)!\ratio!(B)$);
	
				\shadedraw[dashed] ($(B)!\ratio!(S)$) -- ($(B)!\ratio!(A)$) -- ($(B)!\ratio!(C)$) --cycle;
				\draw ($(B)!\ratio!(S)$) -- ($(B)!\ratio!(A)$);
	
				\draw[dashed] ($(C)!\ratio!(S)$) -- ($(C)!\ratio!(A)$) -- ($(C)!\ratio!(B)$) --cycle;
	
				\shade ($(A)!\Ratio!(S)$) -- ($(B)!\Ratio!(S)$) -- ($(B)!\ratio!(S)$) -- ($(B)!\ratio!(A)$)--($(B)!\Ratio!(A)$) -- ($(A)!\ratio!(S)$) --cycle;
	
			\end{tikzpicture}
		\end{center}
	\end{qblock}

\end{questions}

\subsection{Solidi di rotazione}
\begin{questions}

	\begin{qblock}
		\question
		\exonly{
			Il rapporto tra l'apotema e il diametro di un cono retto é di $\frac{5}{6}$ (vale a dire che $\frac{l}{d}=\frac{5}{6}$).\\
			Il volume del cono é di \SI{2.7}{\cubic\centi\metre}.\\
			Calcolare la superficie laterale del cono.\\
		}
		\solonly{\SI{8.13}{\square\centi\metre}}
	\end{qblock}


	\begin{qblock}
		\question
		\exonly{
			Pratichiamo un foro circolare di \SI{10}{\centi\metre}  di diametro in una placca metallica.

			Disponiamo nel foro una sfera di raggio $R>\SI{5}{\centi\metre} $ e constatiamo che la parte visibile della sfera ha un'altezza di \SI{13}{\centi\metre}.

		}

		\begin{parts}
			\part
			\exonly{
				Eseguire uno schizzo della situazione.

			}
			\ifprintanswers
				\begin{tikzpicture}[scale=0.5,baseline={($(current bounding box.north)-(0,1.6ex)$)}]
					\pgfmathsetmacro{\radius}{3}
					\pgfmathsetmacro{\planex}{5}
					\pgfmathsetmacro{\planez}{2.5}
					\pgfmathsetmacro{\ang}{30}
					\draw (-\planex,{-sin(\ang)*\radius},\planez) -- (-\planex,{-sin(\ang)*\radius},-\planez) -- (\planex,{-sin(\ang)*\radius},-\planez)--(\planex,{-sin(\ang)*\radius},\planez);
					\shadedraw[shading=ball,ball color=gray!10] (0,0) circle (\radius cm);
					\draw[very thick] (0,{-sin(\ang)*\radius}) ellipse ({cos(\ang)*\radius} and 0.5);
					\draw[very thick] (\planex,{-sin(\ang)*\radius},\planez) -- (-\planex,{-sin(\ang)*\radius},\planez) ;
					\draw [dashed] (-0.7*\planex,{-sin(\ang)*\radius}) |- (0,\radius);
					\draw[latex-latex] (-0.7*\planex,{-sin(\ang)*\radius}) -- node[left] {\SI{13}{\centi\metre}} ({$(-0.7*\planex,{-sin(\ang)*\radius})$} |- {$(0,\radius)$});

					\draw[dashed] ({cos(\ang)*\radius},{-sin(\ang)*\radius}) -- ({cos(\ang)*\radius},-4) -|  ({-cos(\ang)*\radius},{-sin(\ang)*\radius});
					\draw[latex-latex] ({cos(\ang)*\radius},-4) -- node[below] {\SI{10}{\centi\metre}}({-cos(\ang)*\radius},-4);
				\end{tikzpicture}

			\fi




			\part

			\exonly{
				Determinare il volume della sfera.

			}
			\sol{\SI{1.74}{\cubic\deci\metre}}
		\end{parts}
	\end{qblock}



	\begin{qblock}
		\question
		\exonly{
			Vogliamo costruire una boa avente la forma di un settore sferico in modo che la superficie della calotta sia identica alla superficie laterale del cono.

		}

		\begin{parts}
			\part

			\exonly{
				Calcolare il raggio $r$ della calotta e il volume $V$ del settore sferico se $R=\SI{70}{\centi\metre}$
			}
			\solonly{$r=\SI{56}{\centi\metre}$ , $h=\SI{28}{\centi\metre}$ e \\ $V=\SI{287351}{\cubic\centi\metre}$}
			\part
			\exonly{
				Calcolare il raggio $r$ della calotta e il volume $V$ del settore sferico in funzione del raggio $R$ della sfera da cui viene estratto.}
			\solonly{
				$r=\dfrac{4}{5}R$ e $V=\dfrac{4}{15}\pi R^3$
			}

		\end{parts}

		\ifprintanswers   \else
		\begin{center}
			\begin{tikzpicture}[baseline={($(current bounding box.north)-(0,1.6ex)$)},scale=0.6]
				\pgfmathsetmacro{\radius}{3}
				\pgfmathsetmacro{\planex}{5}
				\pgfmathsetmacro{\planez}{2.5}
				\pgfmathsetmacro{\ang}{30}
				\pgfmathsetmacro{\err}{0.1}
				\coordinate (O) at (0,0);
				\draw[] (O) circle (\radius cm);
				%manteau secteur spherique
				\shadedraw[gray!10, color=black] ({cos(\ang)*\radius-\err},{sin(\ang)*\radius-\err}) -- (O) node [below] {$O$} -- node[below] {$R$} ({-cos(\ang)*\radius+\err},{sin(\ang)*\radius-\err});
	
				\shadedraw[thick] ({cos(\ang)*\radius},{sin(\ang)*\radius}) arc (0:-180:{cos(\ang)*\radius} and 0.4) arc (180-\ang:\ang:\radius);
	
				\shadedraw[very thick] ({cos(\ang)*\radius},{sin(\ang)*\radius}) arc (\ang:180-\ang:\radius);
	
				\draw[dotted]  ({cos(\ang)*\radius},{sin(\ang)*\radius}) arc (0:180:{cos(\ang)*\radius} and 0.4);
	
				\draw[dashed] (0,{sin(\ang)*\radius} ) -- node [above] {$r$} ({cos(\ang)*\radius},{sin(\ang)*\radius});
				\draw [dashed] (0,\radius + 1) -- (0, -\radius -1);
			\end{tikzpicture}
		\end{center}
		\fi
	\end{qblock}


	\begin{qblock}
		\question
		\exonly{
			In una sfera di raggio $R=\SI{1}{\metre}$ é inscritto un cilindro la cui superficie laterale vale la metà della superficie della sfera.

			Calcolare il volume del cilindro.
		}

		\ifprintanswers   \else
		\begin{center}
			\begin{tikzpicture}[baseline={($(current bounding box.north)-(0,1.6ex)$)},scale=2]
				\pgfmathsetmacro{\Radius}{1}
				\pgfmathsetmacro{\radius}{1/sqrt(2)}
				\pgfmathsetmacro{\height}{sqrt(2) /2 }
	
				\coordinate (O) at (0,0);
				\shadedraw[shading=ball,ball color=gray!10] (O) circle (\Radius);
				\draw (0,\height  ) ellipse ({\radius}  and 0.1);
				\draw(0,-\height ) ellipse ({\radius}  and 0.1);
				\draw[dotted] (O) ellipse ({\Radius} and 0.1);
				\draw (-\radius, \height) -- (-\radius,- \height);
				\draw (\radius, \height) -- (\radius, -\height);
				\draw[dashed] (0,\height + 0.5) -- (0,-\height -0.5);
				\draw[thick] (O) -- node[above] {$R$} (\Radius,0);
				%manteau secteur spherique
			\end{tikzpicture}
		\end{center}

		\fi


		\solonly{$V=\SI{2.2}{\cubic\metre}$}
	\end{qblock}


	\begin{qblock}
		\question
		\exonly{
			Calcolare il volume di un cono retto di altezza $h=\SI{15}{\centi\metre}$ nel quale é inscritta una sfera di raggio $r=\SI{6}{\centi\metre}$.
		}
		\solonly{\SI{2827.4}{\cubic\centi\metre}}

		\ifprintanswers   \else
		\begin{center}
			\begin{tikzpicture}[scale=0.3]
				\pgfmathsetmacro\radius{9/sqrt(6)}
				\coordinate (a) at (0,\radius);
				\coordinate (b) at (0,18);
				\draw (\radius,0) -- (b) -- (-\radius,0);
				\draw (\radius,0) arc(0:-180:{\radius} and 1);
				\draw[dashed] (\radius,0) arc(0:180:{\radius} and 1);
				%\fill[black] (0,0) circle (2pt) node[below left] { $O_1$};
	
	
				\shadedraw[shading=ball,ball color=gray!10] (0,3) circle (3);
				\draw[-latex,thick] (0,3) -- ++(-10:3) node[midway,above] {$R$};
	
				\draw[dashed] (0,0) -- (0,18);
				\draw[dashed] (0,0) -- (6,0)  (6,18) --(0,18);
				\draw[latex-latex,thick] (6,18) -- (6,0) node[midway, right] {$h$};
			\end{tikzpicture}
		\end{center}
		\fi
	\end{qblock}


	\begin{qblock}
		\question
		\exonly{
			Una sfera di raggio $R$  é illuminata da una sorgente luminosa puntiforme situata ad una distanza $L$ dal centro $O$ della sfera.
		}

		\begin{parts}
			\part

			\exonly{
				Calcolare la porzione (in \%) di superficie non illuminata della sfera se
				$R=\SI{1}{\metre}$ e $L=\SI{2.5}{\metre}$}
			\solonly{$70\%$}

			\part
			\exonly{
				Calcolare la proporzione si superficie non illuminata della sfera in funzione di $R$ e $L$.}

			\solonly{$\dfrac{(L+R)}{2L}$}
			\part
			\exonly{
				Discutere i due casi estremi $L \gg R$ (che potrebbe modellizzare la situazione Terra-Sole) e $L\approx R$.}
			\solonly{$50\%$ e $0\%$}

		\end{parts}



		\ifprintanswers   \else
			
			\begin{center}
				\begin{tikzpicture}
					\node[circle, draw, very thick] (c) at (0, 0) [minimum size=3cm] {};
					\fill[red]   (4, 0) coordinate (a) circle (3pt);
					\draw (tangent cs:node=c, point={(a)}, solution=1) coordinate (t1)-- (a);
					\draw (tangent cs:node=c, point={(a)}, solution=2) coordinate (t2)-- (a);
					%\draw[dashed] (0,0) -- (t1);
					%\draw[dashed] (0,0) -- (t2);
					\fill[gray!10] (t1)  let \p1 = ($ (0,0) - (t1) $) in  arc ({atan(\y1/\x1)}:{360-atan(\y1/\x1)}:1.5);
	
	
					\path ($(t1)!0.5!(t2)$) coordinate (e1);
	
					\fill[gray!10] (e1)  let \p1 = ($ (t2) - (t1) $) in  ellipse (0.2 cm and {0.5*veclen(\x1,\y1)});
					\node[circle, draw, very thick] (c) at (0, 0) [minimum size=3cm] {};
					\draw (t1) let \p1 = ($ (t1) - (t2) $) in
					arc (90:-90: 0.2 cm and {0.5*veclen(\x1,\y1)});
					\draw (0,0) -- node[below left] {$R$} (150:1.5);
					\fill (0,0) circle (1pt) node [below] {$O$};
					\draw [dashed] (0,0) -- (0,-2) -| (a);
					\draw [latex-latex] (0,-2) -- node [above] {$L$}({$(0,-2)$} -| {$(a)$});
	
				\end{tikzpicture}
			\end{center}
		\fi

	\end{qblock}


\end{questions}


