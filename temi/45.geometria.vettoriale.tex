\section{Geometria vettoriale}

\subsection{Introduzione}
%%%%%%%%%%%%%%%%%%%%%%%%%%%%%%%%%%%%%%%%%%%%%%%%%%%%%%%%%%%%%%%%%%%%%%
%\subfile{../ch/BS-MPT-FP-rappel-fct-1-2}
\begin{questions}


\question
\exonly{Determinare dove costruire un ponte $HG$ perpendicolare al fiume così da minimizzare la distanza tra $A$ e $B$.
}
%\begin{tikzpicture}[baseline={($(current bounding box.north)-(0,1.6ex)$)}]
%
%\draw[-] (-4,0)-- (3.5,0);
%\draw[-,name path=river] (-4,1)-- (3.5,1);
%\fill (-3,2) coordinate[label=above:$A$] (A)  circle (1pt);
%\fill (-2,1) coordinate[label=above:$H$] (H) circle (1pt);
%\fill (-2,0) coordinate[label=above:$G$] (G) circle (1pt);
%\fill (2,-2) coordinate[label=above:$B$] (B)  circle (1pt);
%\draw[dotted] (A) -- (H) --(G) --(B);
%\ifprintanswers 
%\coordinate (B1) at ($(B)+(0,1)$);
%\path[name path=shortpath] (B1) -- (A);
%\path [name intersections={of=shortpath and river,by=G1}];
% \draw[red] (A) -- (G1) -- ++(0,-1) -- (B) ; 
% \draw[red,dashed] (B) -- (B1) -- (A);
% 
%  \fi
%\end{tikzpicture}

\solonly{Discussione in classe }


\question
\exonly{Consideriamo i vettori rappresentati: }
%\solonly{\hfill}
\begin{parts}
\part
\exonly{
Quali vettori sono uguali? }

\solonly{
$\vec{a}$ e $\vec{d}$
}

\part
\exonly{
Quali vettori hanno la stessa intensità?
}

\solonly{
$\vec{a}$ , $\vec{d}$ e $\vec{e}$

$\vec{b}$ e $\vec{c}$	
}

\part
\exonly{
Quali vettori sono opposti?
}
\solonly{
$\vec{b}$ e $\vec{c}$
}

\begin{tikzpicture}[baseline={($(current bounding box.north)-(0,1.6ex)$)}]
\ifprintanswers   \else  
\begin{axis}[
AxisDefaults,
TinyAxisLabels,
xmin=-6,
xmax=6,
ymin=-6,
ymax=6,
yticklabels={},
xticklabels={},
]
\addplot[black] {0};
\draw[->] (axis cs:-2,3)-- node[midway,above]{$\vec{a}$} +(axis direction cs:2,1);
\draw[->] (axis cs:-5,-4)--node[midway,above]{$\vec{b}$}  +(axis direction cs:1,4);
\draw[->] (axis cs:3,-1)--node[midway,above]{$\vec{c}$} +(axis direction cs:-1,-4);
\draw[->] (axis cs:3,-5)-- node[midway,above]{$\vec{d}$}  +(axis direction cs:2,1);
\draw[->] (axis cs:1,1)-- node[midway,above]{$\vec{e}$} +(axis direction cs:2,-1);
\end{axis}
\fi
\end{tikzpicture}

\end{parts}

\question
\exonly{
Consideriamo le forze $\vec{F_1}$ e $\vec{F_2}$. Determinare graficamente e algebricamente il vettore della forza $\vec{F_3}$ tale per cui $\vec{F_1}+\vec{F_2}+\vec{F_3}=\vec{0}$.}



\begin{tikzpicture}[baseline={($(current bounding box.north)-(0,1.6ex)$)}]

\draw[->] (0,0)-- node[midway,above]{$\vec{F_1}$} +(3,2);
\draw[->] (0,0)--node[midway,above]{$\vec{F_2}$}  +(-2,1);
\ifprintanswers  
\draw[->,red] (0,0)--node[midway,above]{$\vec{F_3}$}  +(-1,-3);
\fi
\end{tikzpicture}

\solonly{$\vec{F_3}=-\left( \vec{F_1}+\vec{F_2} \right) $}


\question
\exonly{
Consideriamo le forze $\vec{F_1}$ e $\vec{F_2}$. Determinare graficamente e algebricamente il vettore della forza $\vec{F_3}$ tale per cui $\vec{F_1}+\vec{F_2}+\vec{F_3}=\vec{R}$.}


\begin{tikzpicture}[baseline={($(current bounding box.north)-(0,1.6ex)$)}]

\draw[->] (0,0)-- node[midway,above]{$\vec{F_1}$} +(3,2);
\draw[->] (0,0)--node[midway,above]{$\vec{F_2}$}  +(-2,1);
\draw[->] (0,0)--node[midway,above]{$\vec{R}$}  +(-2,4);
\ifprintanswers  
\draw[->,blue] (3,2)--node[midway,above]{$\vec{F_2}$}  +(-2,1);
\draw[->,red] (1,3)--node[midway,above]{$\vec{F_3}$}  +(-3,1);
\fi
\end{tikzpicture}

\solonly{$\vec{F_3}=\vec{R}-\left( \vec{F_1}+\vec{F_2} \right) $}



\end{questions}

\solonly{\newpage}
\subsection{Componenti, operazioni di base}

\begin{questions}

\question
\exonly{Determinare le componenti dei vettori $\vec{a}$, $\vec{b}$ e $\vec{c}$ 

\begin{tikzpicture}[baseline={($(current bounding box.north)-(0,1.6ex)$)}]
 
\begin{axis}[
AxisDefaults,
TinyAxisLabels,
width=\linewidth,
xmin=-8,
xmax=8,
ymin=-8,
ymax=8,
yticklabels={},
xticklabels={},
]
\addplot[black] {0};
\draw[->] (axis cs:-6,1)-- node[midway,above]{$\vec{a}$} +(axis direction cs:8,5);
\draw[->] (axis cs:7,-7)--node[midway,left]{$\vec{b}$}  +(axis direction cs:0,9);
\draw[->] (axis cs:3,1)--node[midway,left]{$\vec{c}$} +(axis direction cs:-2,-8);
\draw[->,thick] (axis cs:3,3)-- node[midway,above]{$\vec{u}$}  +(axis direction cs:2,-1);
\draw[->,thick] (axis cs:3,3)-- node[midway,left]{$\vec{v}$} +(axis direction cs:1,4);
\end{axis}
\end{tikzpicture}
}
%\solonly{\hfill}

\begin{parts}
\part
\exonly{rispetto alla base $\left(\vec{u},\vec{v} \right)$}
\solonly{
$\vec{a}=\vecII{3}{2}$, $\vec{a}=\vecII{-1}{2}$, $\vec{c}=\vecII{0}{-2}$
}
\part
\exonly{rispetto alla standard}
\solonly{
$\vec{a}=\vecII{8}{5}$, $\vec{a}=\vecII{0}{9}$, $\vec{c}=\vecII{-2}{-8}$
}
\end{parts}





\question
\exonly{Dati i vettori $\vec{a}=\vecII{3}{5}$, $\vec{b}=\vecII{-4}{2}$ e $\vec{c}=\vecII{-3}{-4}$ determinare le componenti di
	
	 $\vec{d}=\vec{a}+\vec{b}+\vec{c}$, $\vec{e}=\vec{a}-\vec{b}-\vec{c}$, $\vec{f}=\frac{1}{2}\vec{a}$ e $\vec{g}=-\dfrac{3}{2}\vec{c}$

}

\solonly{$\vec{d}=\vecII{-4}{3}$, $\vec{e}=\vecII{10}{7}$, $\vec{f}=\vecII{\sfrac{3}{2}}{\sfrac{5}{2}}$ e $\vec{g}=\vecII{\sfrac{9}{2}}{6}$}

\question
\exonly{Dati i vettori $\vec{a}=\vecII{2}{3}$, $\vec{b}=\vecII{-4}{2}$.

 Calcolare $\norm{\vec{a}}$, $\norm{\vec{b}}$, $\norm{\vec{a}+\vec{b}}$ e $\norm{\vec{a}}+\norm{\vec{b}}$}
\solonly{$\norm{\vec{a}}=\sqrt{13}$, $\norm{\vec{b}}=\sqrt{20}$, $\norm{\vec{a}+\vec{b}}=\sqrt{29}\approx \num{5.39}$ e $\norm{\vec{a}}+\norm{\vec{b}}\approx \num{8.07}$}

\exnewpage
\question 
\exonly{
Dati i vettori $\vec{v}=\vecII{3}{-5}$e $\vec{w}=\vecII{-2}{3}$ calcolare:
}
%\solonly{\hfill}
\begin{multicols}{2}
\begin{parts}
\part
\exonly{$\vec{v}+\vec{w}$}
\solonly{$\vecII{1}{-2}$}

\part
\exonly{$\vec{w}-\vec{v}$}
\solonly{$\vecII{-5}{8}$}

\part
\exonly{$-5\vec{v}$}
\solonly{$\vecII{-15}{25}$}

\part
\exonly{$\norm{\vec{v}}$}
\solonly{$\sqrt{34}$}

\part
\exonly{$2\vec{v}+3\vec{w}$}
\solonly{$\vecII{0}{-1}$}

\part
\exonly{$3\vec{v}-2\vec{w}$}
\solonly{$\vecII{13}{-21}$}

\part
\exonly{$\norm{\vec{v}-\vec{w}}$}
\solonly{$\sqrt{89}$}

\part
\exonly{$\norm{\vec{v}}-\norm{\vec{w}}$}
\solonly{$\sqrt{34}-\sqrt{13}$}


\part
\exonly{$\norm{5\vec{v}}$}
\solonly{$5\sqrt{34}$}

\part
\exonly{$\norm{-5\vec{v}}$}
\solonly{$5\sqrt{34}$}
\end{parts}
\end{multicols}


\question
\exonly{Sono dati due vettori  $\vec{a}$ e $\vec{b}$ (vedi schizzo sotto) con $\norm{\vec{a}}=2$ e $\norm{\vec{b}}=3$.

\begin{tikzpicture}[baseline={($(current bounding box.north)-(0,1.6ex)$)}]
\draw[-latex] (-3,0) -- (3,0) coordinate (X);
\draw[-latex] (0,-3) -- (0,3);
\draw[->] (0,0) coordinate (O)-- node[midway,above]{$\vec{a}$} +(1,2) coordinate (A);
\draw[->] (O)--node[midway,above]{$\vec{b}$}  +(-3,-2) coordinate (B);
\draw pic["$70\degree$",draw, angle eccentricity=1.5, blue] {angle=X--O--A};
\draw pic["$150\degree$",draw, angle radius=0.7cm,angle eccentricity=1.5, blue] {angle=B--O--X};
\end{tikzpicture}
}

\begin{parts}
\part

\exonly{Calcolare le componenti in base standard dei vettori $\vec{a}$ e $\vec{b}$. Approssimare il risultati a due decimali. }	
\solonly{$\vec{a} \approx \vecII{\num{0.68}}{\num{1.88}}$, $\vec{b} \approx \vecII{\num{-2.6}}{\num{-1.5}}$}

\part
\exonly{ Determinare il vettore $\vec{c}=\vec{a}+\vec{b}$.  }

\solonly{ $\vec{c}=\vecII{\num{-1.92}}{\num{0.38}}$  }

\part
\exonly{Esprimere $\vec{c}$ in forma polare. }
\solonly{$\vec{c}=1.96 \angle \ang{168.8}$}

\end{parts}


\question
\exonly{
Ad un certo instante un aeroplano vola ad una velocità di
\SI{700}{\kilo\meter/\hour} e il pilota punta in direzione nord. Un vento di \SI{120}{\kilo\meter/\hour}
 proveniente da est ne perturba la traiettoria. 
Calcolare intensità e direzione della velocità reale  dell'aeroplano.
}
\solonly{Velocità al suolo di \SI{710.21}{\kilo\meter/\hour} in direzione $\num{9.7}\degree  NO$}

\question
\exonly{Dato i punti $A(2;3)$, $B(-1;2)$ e $C(5;-4)$ determinare le componenti dei vettori $\overrightarrow{AB}$,$\overrightarrow{BC}$ e $\overrightarrow{AC}$}
\solonly{$\overrightarrow{AB}=\vecII{-3}{-1}$,$\overrightarrow{BC}=\vecII{6}{-6}$ e $\overrightarrow{AC}=\vecII{3}{-7}$}

\solcbreak
\question
\exonly{Date le coordinate $A(1;-1)$, $B(2;2)$, $C(-3;2)$ , $D(-4;-1)$ e $E(0,2)$. }
%\solonly{\hfill}

\begin{parts}
\part
\exonly{Determinare la natura del quadrilatero $ABCD$ (quadrilatero qualunque, trapezio, parallelogrammo, rettangolo, quadrato)}
\solonly{

\begin{tikzpicture}[baseline={($(current bounding box.north)-(0,1.6ex)$)},scale=0.5]
\draw[] (-3,2) coordinate[label=left:$C$] (C) -- (2,2) coordinate[label=right:$B$] (B) -- (1,-1) coordinate[label=right:$A$] (A) -- (-4,-1) coordinate[label=left:$D$] (D) -- cycle;
\end{tikzpicture}

Parallelogrammo: $ \overrightarrow{AB} = \overrightarrow{DC}$}
\part
\exonly{Determinare la natura del quadrilatero $ABED$ (quadrilatero qualunque, trapezio, parallelogrammo, rettangolo, quadrato)}
\solonly{

\begin{tikzpicture}[baseline={($(current bounding box.north)-(0,1.6ex)$)},scale=0.5]
\draw[] (0,2) coordinate[label=left:$E$] (E) -- (2,2) coordinate[label=right:$B$] (B) -- (1,-1) coordinate[label=right:$A$] (A) -- (-4,-1) coordinate[label=left:$D$] (D) -- cycle;
\end{tikzpicture}

Trapezio: $ \overrightarrow{AD} =\frac{5}{2} \overrightarrow{BE}$

}
\end{parts}




\end{questions}
%\exonly{\newpage}
\newpage
\subsection{Prodotto scalare e angoli}
\begin{questions}

\question
\exonly{Calcolare l'angolo tra i vettori $\vec{a}$ e $\vec{b}$}
%\solonly{\hfill}
\begin{parts}
\part
\exonly{$\vec{a}=\vecII{-2}{5}$ , $\vec{b}=\vecII{3}{6}$}
\solonly{$\num{48.4}\degree$}

\part
\exonly{$\vec{a}=\vecII{4}{7}$ , $\vec{b}=\vecII{-2}{3}$}
\solonly{$\num{63.4}\degree$}
\end{parts}

\question
\exonly{Determinare se i vettori dati sono ortogonali, paralleli o nessuna delle due cose.}
%\solonly{\hfill}
%\begin{multicols}{2}
\begin{parts}
\part
\exonly{$\vecII{4}{-1}$, $\vecII{2}{8}$}
\solonly{$\bot$}
\part
\exonly{$\vecII{3}{6}$, $\vecII{4}{-2}$}
\solonly{$\bot$}

\part
\exonly{$\vecII{3}{5}$, $\vecII{7}{1}$}
\solonly{Né $\bot$ né $\parallel$}

\part
\exonly{$\vecII{6}{-18}$, $\vecII{-4}{12}$}
\solonly{$\parallel$}

\end{parts}
%\end{multicols}

%\solcbreak
\question
\exonly{Dati i punti $A(-2;3)$, $B(1;-1)$ e $C(3;\sfrac{1}{2})$. Determinare le coordinate di $D$ tali per cui il quadrilatero $ABCD$ sia un parallelogrammo. Può essere un rettangolo?}
\solonly{$D(0,\sfrac{9}{2})$

\begin{tikzpicture}[baseline={($(current bounding box.north)-(0,1.6ex)$)},scale=0.5]
\draw[] (-2,3) coordinate[label=left:$A$] (E) -- (1,-1) coordinate[label=right:$B$] (B) -- (3,0.5) coordinate[label=right:$C$] (A) -- (0,4.5) coordinate[label=left:$D$] (D) -- cycle;
\end{tikzpicture}
}

\question
\exonly{Determinare la lunghezza della proiezione del vettore $\vec{b}=\vvec{1,6}$ su $\vec{a}=\vvec{3,3}$.}
\solonly{\num{4.95} }


\question
\exonly{Dati punti $A(2;3)$, $B(8;2)$ e $C(k;8)$ determinare $k$ tale per cui $\angle CBA = \SI{90}{\degree}$ }
\solonly{$9$ }

\end{questions}
\exnewpage

\solnewpage
\subsection{Vettori dello spazio}
\begin{questions}


\question
\exonly{Dati $A(-1;8;2)$, $B(4;5;-1)$ e $C(2;7;1)$
Determinare le coordinate del punto $D$ tali per cui $ABCD$ sia un parallelogrammo. }
\solonly{$D(-3;10;4)$}

\question
\exonly{Dati $A(2;3;-1)$, $B(-1;5;2)$ e $C(-3;4;1)$}
%\solonly{\hfill}
\begin{parts}
\part
\exonly{Determinare le componenti di $\overrightarrow{AB}$, $\overrightarrow{AC}$ e $\overrightarrow{BC}$}
\solonly{$\overrightarrow{AB}=\vecIII{-3}{2}{3}$, $\overrightarrow{AC}=\vecIII{-5}{1}{2}$ e $\overrightarrow{BC}=\vecIII{-2}{-1}{-1}$}

\part
\exonly{Calcolare la norma di $\overrightarrow{AB}$, $\overrightarrow{AC}$ e $\overrightarrow{BC}$}
\solonly{$\norm{\overrightarrow{AB}}=\sqrt{22}$, $\norm{\overrightarrow{AC}}=\sqrt{30}$ e $\norm{\overrightarrow{BC}}=\sqrt{6}$}

\part
\exonly{Calcolare l'angolo tra $\overrightarrow{AB}$ e $\overrightarrow{AC}$}
\solonly{$\num{26.46}\degree$}

\part
\exonly{Determinare le coordinate del punto $D$ tale per cui $ABCD$ sia un parallelogrammo.}
\solonly{$D(0;2;-2)$}

\part
\exonly{Determinare l'area del parallelogrammo $ABCD$.  }
\solonly{$  \sqrt{131}\approx 11.45$ }
\end{parts}

\question
\exonly{Determinare le coordinate mancanti tali per cui i punti $A(2;-1;10)$, $B(8;5;1)$ e $C(x;3;z)$ siano allineati.}
\solonly{$x=6$ e $z=4$}



\question
\exonly{Dati i punti $A(-1;-1;-1)$, $B(-4;3;3)$, $C(-5;6;0)$ , $D(-2;2;-4)$ e $E(5;2;6)$

\begin{tikzpicture}[baseline={($(current bounding box.north)-(0,1.6ex)$)}]
\coordinate (BC) at (1,0.8);
\coordinate (AE) at (1,4);
\draw[] (0,0) coordinate[label=left:$A$] (A) -- ++(2,-1) coordinate[label=below:$B$] (B) -- ++(BC) coordinate[label=right:$C$] (C);
\draw (A) -- ++ (BC) coordinate[label=above left:$D$] (D) -- (C);
\draw (A) -- ++(AE) coordinate[label=left:$E$] (E);
\draw (B) -- ++(AE) coordinate[label=below right:$H$] (H);
\draw (C) -- ++(AE) coordinate[label=right:$G$] (G);
\draw (D) -- ++(AE) coordinate[label=above:$F$] (F);
\draw (E) -- (H) -- (G) -- (F) -- cycle;
\end{tikzpicture}
}
%\solonly{\hfill}

\begin{parts}
\part
\exonly{Mostrare che il quadrilatero $ABCD$ é un parallelogrammo}
\solonly{$\overrightarrow{AB}=\vecIII{-3}{4}{4}=\overrightarrow{DC}=\vecIII{-3}{4}{4}$}

\part
\exonly{Determinare le coordinate di $H$, $G$ e $F$ tali per cui il solido risultante (vedi schizzo) sia un prisma (costruito per estrusione della base $ABCD$)}
\solonly{$F(4;5;3)$, $G(1;9;7)$ , $H(2;6;10)$}

\part

\exonly{Calcolare gli angoli $\angle EAB$ e $\angle DAB$}
\solonly{
$\angle EAB= \num{69.24}\degree$
$\angle DAB= \num{83.83}\degree$}

\part
\exonly{Ammettendo che l'unità sia di \SI{1}{\centi\metre} calcolare l'area di $ABCD$}
\solonly{$\num{27.75} \si{\square\centi\metre}$}

\part
\exonly{Determinare le coordinate del punto $I$ sul segmento [AE] tale per cui a distanza tra $A$ e $I$ sia di \SI{2}{\centi\metre}}
\solonly{$I(\num{0.24},-\num{0.38},\num{0.44})$}
\end{parts}
\end{questions}



\exnewpage
\solnewpage
\subsection{Problemi misti}

\begin{questions}
\question
\exonly{Una forza $\vec{F}=\vecII{5}{3}$ \si{\newton} sposta un corpo di $\vec{s}=\vecII{12}{21} \si{\metre}$. }
%\solonly{\hfill}
\begin{parts}
\part
\exonly{Determinare il lavoro compiuto dalla forza $\vec{F}$}
\sol{$123 \si{\joule}$}

\part


\exonly{Determinare l'angolo tra la forza $\vec{F}$ et la direzione dello spostamento.}
\solonly{$\num{29.29}\degree$}
\end{parts}


\question
\exonly{Di una piramide a base rettangolare nello spazio si conoscono:\\
$A(5;-3;9)$, $B(-3;3;4)$, $C(0;7;z)$ e il vertice $V(6;10;12)$

\begin{tikzpicture}[baseline={($(current bounding box.north)-(0,1.6ex)$)}]
\coordinate (BC) at (1,0.8);
\coordinate (AE) at (2,3);
\draw[] (0,0) coordinate[label=left:$A$] (A) -- ++(2,-1) coordinate[label=below:$B$] (B) -- ++(BC) coordinate[label=right:$C$] (C);
\draw (A) -- ++ (BC) coordinate[label=above right:$D$] (D) -- (C);
\draw (A) -- ++(AE) coordinate[label=left:$V$] (E);\draw[dotted] (D)--(E);
\draw (B) -- (E) (C)--(E);
\draw[dotted] (D)--(E);

\end{tikzpicture}

Determinare: }
\begin{parts}
	 \part
	\exonly{ L'ampiezza dell'angolo $\angle AVB$ }
	\solonly{$\ang{48.29}$ }
	
	\part
	\exonly{L'area del triangolo $AVB$ }
	\solonly{ $\num{69.55}$}
	
	\part
	\exonly{La coordinata mancante di $C$ }
	\solonly{ $z=4$}
	
	\part
	\exonly{Le coordinate di $D$ }
	\solonly{$D(8;1;9)$ }
	
	\part
	\exonly{L'altezza della piramide }
	\solonly{$h=\num{5.81}$ }
\end{parts}








\question
\exonly{Determinare il valore del parametro $n$ sapendo che $\vec{a}=\vecIII{1}{n}{0}$, $\vec{b}=\vecIII{0}{n}{1}$ et che l'angolo tra i due vettori é di $60\degree$}
\solonly{$n\approx \pm \num{1}$}


\end{questions}


\exnewpage
\solnewpage
\subsection{Equazione vettoriale della retta}

\begin{questions}
	
	\question
	\exonly{Determinare l'equazione vettoriale della retta passante per i punti $A(-1;5)$ e $B(4;6)$}
	\solonly{$\vecII{x}{y}=\vecII{-1}{5}+\lambda \vecII{5}{1}$}
	
	\question
	\exonly{Determinare l'equazione vettoriale della retta passante per $A(8;-1)$ e $B(2;7)$ e la sua distanza dal punto $C(7;17)$}
	
	\solonly{$\vvec{x,y}=\vvec{8,-1}+ \lambda \vvec{-6,8}$ \\ Distanza:$10$}
	
	\question
	\exonly{Determinare la distanza della retta $4x+3y+9=0$ dal punto $C(3;-2)$}
	
	\solonly{$3$}
	
	\question
	\exonly{Dato il quadrilatero $ABCD$ con $A(-2; -1)$, $B(6; 1)$, $C(5; 5)$ e $D(-1; 3)$}
	%\solonly{\hfill}
	
	\begin{parts}
		\part
		\exonly{Determinare il punto d’intersezione delle due diagonali}
		\solonly{$(\num{1.75};\num{2.21})$}
		
		\part
		\exonly{Determinare l’angolo acuto che formano le due diagonali tra di loro}
		\solonly{$\num{56.54}\degree$}
		
		\part
		\exonly{Determinare la distanza tra il punto d’intersezione e la retta passante per $A$ e $D$}
		\solonly{$\approx \num{2.9}$}
		
		\part
		
		\exonly{Determinare l’area del quadrilatero}
		\solonly{$\num{28}$}
	\end{parts}

	\question
\exonly{Determinare l'equazione vettoriale della retta passante per i punti $A(-1;5;4)$ e $B(4;6;-8)$}
\solonly{$\vvec{x,y,z}=\vvec{-1,5,4}+\lambda \vvec{5,1,-12}$}


\question
\exonly{Dato il quadrilatero $ABCD$ con $A(-2; -1;4)$, $B(6; 1;-2)$, $C(5; 5;1)$ e $D(\frac{3}{2};4;4)$}
%\solonly{\hfill}

\begin{parts}
	\part
	\exonly{Determinare il punto d’intersezione delle due diagonali}
	\solonly{$(\frac{8}{3};3;2)$}
	
	\part
	\exonly{Determinare l’angolo acuto che formano le due diagonali tra di loro}
	\solonly{$\num{64.44}\degree$}
	
	\part
	\exonly{Determinare la distanza tra il punto d’intersezione e la retta passante per $A$ e $D$}
	\solonly{$\approx \num{2.79}$}
	
	\part
	
	\exonly{Determinare l’area del quadrilatero}
	\solonly{$\num{36.59}$}
\end{parts}
	
	\question
	\exonly{Determinare la distanza della retta  $\vvec{x,y,z}=\vvec{1,2,5}+ \lambda \vvec{2,1,1}$ dal punto $P(-5;3;-1)$}
	\solonly{$\num{4.98}$}
	
	\question
		\exonly{Determinare un'equazione vettoriale della retta passante per $A(8;-1;3)$ e $B(2;7;5)$ e la sua distanza dal punto $C(7;17;10)$}
	
	\solonly{$\vvec{x,y,z}=\vvec{8,-1,3}+ \lambda \vvec{-6,8,2}$ \\ Distanza: $10.74$}
	
\end{questions}
