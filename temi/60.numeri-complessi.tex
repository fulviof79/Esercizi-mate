\section{Numeri complessi}
\subsection{Forma algebrica e forma polare}
%%%%%%%%%%%%%%%%%%%%%%%%%%%%%%%%%%%%%%%%%%%%%%%%%%%%%%%%%%%%%%%%%%%%%%

\begin{questions}

\question
\exonly{
Convertire i seguenti numeri complessi dalla forma algebrica alla forma polare e viceversa}

\begin{parts}
\part 
\exonly{$5-5 \sqrt{3}\mathrm{j}$ }
\solonly{$10 \angle \frac{\pi}{3}$ }

\part
\exonly{ $3+4 j$ }
\solonly{$5\angle \varphi \operatorname{con} \varphi=\tan ^{-1} 4 / 3 \approx 0,92 \mathrm{rad}$ }

\part
 \exonly{$-6-6 j$ }
\solonly{$6 \sqrt{2}\angle  \frac{3\pi}{4}$ }

\part 
\exonly{$-5-12 j$ }
\solonly{$12 \angle \varphi$ con $\varphi=-\pi+\tan ^{-1} 12 / 5 \approx-1,96 \mathrm{rad}$ }

\part  \exonly{$-4 \sqrt{3}+4 j$ }
\solonly{$8 \frac{5\pi}{6}$ }

\part 
\exonly{$6 \angle \frac{\pi}{6}$ }
\solonly{$3 \sqrt{3}+3 j$ }


\part 

%\exonly{ $7 \sqrt{2} \mathrm{e}^{-\mathrm{j} \pi / 4}$ }
\exonly{ $7 \sqrt{2} \angle  \frac{\pi}{4}$ }
\solonly{$7+7 j$ }
\part 
%\exonly{ $10 \mathrm{e}^{\mathrm{j} 2 \pi / 3}$ }

\exonly{ $10 \angle \frac{2\pi}{3} $ }
\solonly{$-5+5 \sqrt{3} j$ }

\part \exonly{$8 \angle \frac{3\pi}{2}$ }

\solonly{$-8\mathrm{j}$ }

\part \exonly{$4 \angle \frac{\pi}{12}$ }
\solonly{$\sqrt{2}[(\sqrt{3}+1)+j(\sqrt{3}-1)]\approx \num{3.86}+j\num{1.04}$ }

\end{parts}


\end{questions}

\subsection{Operazioni di base con numeri complessi}

\begin{questions}
	
	\question
	\exonly{Sono dati i numeri complessi: 
		\begin{multicols}{2}
			\begin{itemize}
				\setlength\itemsep{1mm}
				\item $a=3+\ii 5$
				\item $b=-2+\ii 3$
				\item $c=4-\ii 7$
				\item $d=-6-\ii 2$
			\end{itemize}
			
		\end{multicols}
	}
	\exonly{Calcola: }	
	\begin{multicols}{2}	
		\begin{parts}
			\setlength\itemsep{2mm}
			\part 
			\exonly{$a+b=$ }
			\solonly{$1+8\ii$ }
			
			\part 
			\exonly{$b-c=$ }
			\solonly{$-6+10\ii$ }
			
			\part 
			\exonly{$a+b-c+d=$ }
			\solonly{ $ -9+13\ii$ }
			
			\part 
			\exonly{$(a-b)\cdot(c-d)=$ }
			\solonly{$60-5\ii$ }
			
			\part 
			\exonly{$\frac{b}{c}=$ }
			\solonly{$\frac{29}{65}-\frac{2}{65}\ii $ }
			
			\part 
			\exonly{$\frac{a-c}{b+d}=$ }
			\solonly{$\frac{4}{13}-\frac{19}{13}\ii $ }
		\end{parts}	
	\end{multicols}	


	\question
	\exonly{Calcolare le seguenti somme e differenze di numeri complessi: }
	\begin{multicols}{2}
	\begin{parts}
		\part 
	\exonly{ $(3-3 j)+(2-2 j)$ }
	\solonly{$5-5 j$ }
		\part 
		\exonly{$(4+5 j)+(7-4 j)$ }
		\solonly{$11+\mathrm{j}$ }
		\part 
		\exonly{$(-8+5 j)+(2-j)$ }
		\solonly{$-6+4 j$ }
	\part 
	\exonly{$(7-7 j)+(3+6 j)$ }
	\solonly{$10-\mathrm{j}$ }
	\part \exonly{$(5+j)+(-2-j)$ }
	\solonly{$3$ }
	\part 
	\exonly{$(5-2 j)+(2+3 j)$ }
	\solonly{$7+\mathrm{j}$ }
	\part \exonly{ $(2+5 j)+(1-9 j)$ }
	\solonly{$3 - 4 j$ }
	\part  \exonly{$(4+2 j)+(4-2 j)$ }
	\solonly{$8$ }
	\part \exonly{$(-2-\mathrm{j})+(4-\mathrm{j})$ }
	\solonly{$2-2 j$ }
	\part \exonly{$(5+4 j)+(-3+2 j)$ }
	\solonly{$2+6 j$}

\part \exonly{		(13+4 j)-(-2+8 j) }
\solonly{$15-4 j $ }
\part \exonly{(-1+9 j)-(-7+9 j) }	
\solonly{$6$ }	
\part \exonly{		(2+3 j)-(-11-2 j)  }
\solonly{$13+5 j$ }
\part \exonly{		(1-6 j)-(-8+j) }
\solonly{$9-7 j $ }
\part  \exonly{		(1+4 j)-(4-7 j)  }
\solonly{$-3+11 j $ }
 \part \exonly{		(2+5 j)-(1-4 j) }
 \solonly{$1+9 j $ }
\part \exonly{		(1-2 j)-(1+2 j)  }
\solonly{$-4 j $ }
\part \exonly{		(3-2 j)-(2-j)  }
\solonly{$1-j $ }
\part \exonly{		(3+7 j)-(1+3 j) }
\solonly{$2+4 j $ }
\part	\exonly{	(2+4 j)-(2-9 j) }
\solonly{$13 j$ }

	\end{parts}
\end{multicols}


\exnewpage
\question
\exonly{Calcolare le seguenti moltiplicazioni e divisioni di numeri complessi: }
\begin{multicols}{2}
\begin{parts}

\part\exonly{$		(7-3 j)(4+6 j) $}
\solonly{$46+30 j$ }
\part \exonly{	$	(4+5 j)(3+2 j) $}
\solonly{$2+23 j$ }
\part \exonly{$(-8-2 j)(4+3 j) $ }
\solonly{$-26-32 j$ }
\part \exonly{	$(5+j)(2-2 j)$ }
\solonly{ $12-8 j$ }
\part \exonly{	$	(3+6 j)(-6-6 j) $}
\solonly{$18-54 j$ }
	\part 	\exonly{$(4+2 j)(1+j) $}
	\solonly{$	2+6 j$ }

\part \exonly{ $\dfrac{(2+4 j)}{(1-3 j)} $ }
\solonly{$-1+j$ }
\part \exonly{$	\dfrac{(21+20 j)}{(5+2 j)} $  }
\solonly{$5+2 j $ }
\part \exonly{$	\dfrac{(-23-j) }{(3+j)} $ }
\solonly{$-7+2 j $ }
\part \exonly{$	\dfrac{(-36+2 j)}{(3+4 j)}$ }
\solonly{$-4+6 j$ }
\part \exonly{$	\dfrac{(-7-24 j)}{(3-4 j)} $ }
\solonly{$3-4 j$ }
	\part \exonly{$\dfrac{(16+11 j)}{(2+5 j)}$ }
	\solonly{$3-2 j$ }


\end{parts}
\end{multicols}




\end{questions}

\subsection{Esercizi di calcolo}

\begin{questions}
	\question
	\exonly{Calcolare: }
	\begin{multicols}{2}
	\begin{parts}
		\setlength\itemsep{2mm}
		\part 
		\exonly{$ 3\angle \ang{0}+2\angle \ang{180}= $ }
		\solonly{$1$ }
		
		\part 
		\exonly{$ 2\angle \frac{\pi}{2} + 2\angle \frac{3\pi}{2}= $ }
		\solonly{$0 $ }
		
		\part 
		\exonly{$ 5\angle \ang{20} - 3\angle \ang{20}= $ }
		\solonly{$\num{1.88}+\num{0.68}j $ }
		
		\part 
		\exonly{$ 6\angle \frac{\pi}{3} + 2\angle \frac{-2\pi}{3}= $ }
		\solonly{$2 + 2\sqrt{3}j$ }
		
		\part 
		\exonly{$ 3\angle \ang{30}+2\angle \ang{45}= $ }
		\solonly{$\num{4.01}+\num{2.91}j$ }
		
		
		\part 
		\exonly{$ 7\angle \frac{\pi}{2} - 5\angle \frac{2\pi}{3}=  $ }
		\solonly{$\num{-1.32}-\num{4.76}j$ }
		
		\uplevel{Calcolatrice}
		\part 
		\exonly{$ \dfrac{5\angle \ang{23} +5 \angle  \ang{32}}{3\angle\ang{-35}-3\angle \ang{40}}= $}
		\solonly{ $\num{-1.16}+\num{2.47}j$  }
		
		\part 
		\exonly{ $ \dfrac{3\angle \frac{2\pi}{5} + 5 \angle \frac{3\pi}{4}}{3\angle \frac{3\pi}{2} - 3\angle \frac{-2\pi}{3}}= $ }
		\solonly{ $\num{0.24}-\num{2.16}j$  }
		
	\end{parts}	
\end{multicols}

\question
\exonly{Calcolare }
\begin{multicols}{2}
\begin{parts}
	\part
	 \exonly{$(\sqrt{2}-\sqrt{2} \mathrm{j})^{2} $ }
	\solonly{$-4j$ }
	
	\part
	 \exonly{$(-\sqrt{2}+\sqrt{2} \mathrm{j})^{3}$ }
	\solonly{$4 \sqrt{2}+4 \sqrt{2} j$ }
	
	\part
	 \exonly{$(1+\sqrt{3} \mathrm{j})^{4}$ }
	\solonly{$-8-8 \sqrt{3} j$ }
	
	\part
	 \exonly{$	\sqrt{-2 j} $ }
	\solonly{$-1+j$ e $1-j$ }
	
	\part 
	\exonly{$\sqrt[3]{-8} $ }
	\solonly{$1+\sqrt{3}j$ , $-2$ e $1-\sqrt{3}j$}
	
	\part 
	\exonly{$\sqrt[4]{-\frac{1}{2}+\frac{\sqrt{3}}{2}j} $}
	\solonly{$\dfrac{\sqrt{3}}{2}+\dfrac{1}{2}j$,$ -\dfrac{1}{2}+\dfrac{\sqrt{3}}{2}j$,$-\dfrac{\sqrt{3}}{2}+\dfrac{1}{2}j$, $ \dfrac{1}{2}-\dfrac{\sqrt{3}}{2}j$}
	

\end{parts}
\end{multicols}


\end{questions}

\exnewpage
\subsection{Equazioni}
\begin{questions}

\question
\exonly{ Risolvi le seguenti equazioni indicando i risultati sotto forma di numeri complessi e poi verifica la correttezza del risultato. }
\begin{multicols}{2}
	\begin{parts}
		\setlength\itemsep{2mm}
		\part \exonly{$2x+3=0$ }
		\solonly{$\es{-\dfrac{3}{2}}$ }
		\part \exonly{$5x+4=2x+2$  }
		\solonly{$\es{-\dfrac{2}{3}}$ }
		\part \exonly{$3x\cdot(2x+1)=3\cdot(x-5)$ }
		\solonly{$\es{\sqrt{\dfrac{5}{2}}\ii , -\sqrt{\dfrac{5}{2}}\ii}$ }
		\part \exonly{$2x^2-4x+12=0$ }
		\solonly{$\es{1+\sqrt{5}\ii,1-\sqrt{5}\ii}$ }
		\part \exonly{$3x^2+2x+5=0$ }
		\solonly{$\es{-\dfrac{1}{3}-\dfrac{\sqrt{14}}{3}\ii,-\dfrac{1}{3}+\dfrac{\sqrt{14}}{3}\ii}$ }
		\part \exonly{ $x^2-3x+4=0$ }
		\solonly{$\es{\dfrac{3}{2}-\dfrac{\sqrt{7}}{2}\ii,\dfrac{3}{2}+\dfrac{\sqrt{7}}{2}\ii}$ }
		\part \exonly{$2x(x-1)=3(x-5)$ }
		\solonly{$\es{\dfrac{5}{4}-\dfrac{\sqrt{95}}{4}\ii,\dfrac{5}{4}+\dfrac{\sqrt{95}}{4}\ii}$ }
		\part \exonly{$(2x+3)(3x+5)=(x-1)(x+2)$ }
		\solonly{$\es{-\dfrac{9}{5}-\dfrac{2}{5}\ii,-\dfrac{9}{5}+\dfrac{2}{5}\ii}$ }
	\end{parts}	
\end{multicols}	

\question

\exonly{Risolvi le seguenti equazioni indicando i risultati sotto forma di numeri complessi e poi verifica la correttezza del risultato. }
\begin{multicols}{2}
	\begin{parts}
		\setlength\itemsep{2mm}
		\part 
		\exonly{$2x+3-\ii 5=5x-4+\ii 2$ }
		\solonly{$\es{\dfrac{7}{3}-\dfrac{7}{3}\ii}$ }
		
		\part 
		\exonly{$(2+\ii 3)x+5- \ii 2= -(1+\ii 2)x - (4+\ii 3)$ }
		\solonly{$\es{-\dfrac{16}{17}+\dfrac{21}{17}\ii}$ }
		
		\part 
		\exonly{$3x^2 +\ii 2 -5=0 $	 }
		\solonly{$\es{\sqrt{\dfrac{5}{3}-\dfrac{2}{3}\ii}}\approx\newline \left\lbrace   \num{-1.316}+\num{0.253}\ii, \num{1.316}-\num{0.253}\ii\right\rbrace $ }
		
		\part 
		\exonly{$2x -\ii 4 +3=0 $		 }
		\solonly{ $\es{-\frac{3}{2}+2\ii}$ }
		
		\part 
		\exonly{$3x^2-\ii 2 x + 3 = 5x-4 $ }
		\solonly{$\es{\num{1.04}+\num{1.67}\ii , \num{0.63}-\num{1.01}\ii}$ }
		
	%	\part \exonly{$3x-\ii 4=(2x+3-\ii 2)\cdot (3-\ii 2)\cdot x$ }
		%\part \exonly{$3x^2-(2+\ii 3)\cdot x +4-\ii 3=0$ }
		
		
		%\part \exonly{$(2x+3-\ii 4)^2=5$  }
	\end{parts}		
\end{multicols}

\end{questions}

\subsection{Applicazioni}
\begin{questions}
	
	
	\question
	\exonly{Immaginiamo di avere un quadrilatero i cui vertici, nel piano complesso, sono dati da $z_1$, $z_2$, $z_3$ e $z_4$.
	
Se moltiplicassimo ognuno dei vertici per il numero complesso $w=2+\sqrt{5}\ii$ quali operazioni verrebbero effettuata sul poligono? Notare tutte le risposte corrette. }
\solonly{ Si otterranno contemporaneamente}
	
	\begin{checkboxes}
		\choice Rotazione di $\num{41.81}\degree$
		\CorrectChoice Omotetia (ingrandimento) di fattore $3$
		\choice Rotazione di $-\num{48.19}\degree$
		\CorrectChoice Rotazione di $\num{48.19}\degree$
		\choice Omotetia (rimpicciolimento) di fattore $\frac{1}{3}$
		\choice Rotazione di $-\num{41.81}\degree$ 
	\end{checkboxes}
	
	\question
	\exonly{	Sono date le intensità di corrente:
	
	\begin{align*}
	I_1(t) & =\SI{1.5}{A} \cdot \sin{(100\pi \si{s^{-1}} \cdot t + \frac{\pi}{3})} \\
	I_2(t) & =\SI{0.8}{A}\cdot \sin{(100\pi \si{s^{-1}} \cdot t + \frac{\pi}{6})} \\
	I_3(t) & =\SI{0.5}{A}\cdot \sin{(100\pi \si{s^{-1}} \cdot t + \frac{\pi}{4})}
	\end{align*} }
	
	
	\exonly{Determina  $I(t)=I_1(t) + I_2(t) - I_3(t)$
	
Verifica la correttezza del risultato con la calcolatrice, Desmos o GeoGebra }
	\solonly{$I(t)=1.731\sin\left(100\pi t+0.8902\right)$ }
	
%	\begin{itemize}
%		\item Aiutandoti con la calcolatrice traccia un grafico accurato di \begin{equation*}
%		I(t)=I_1(t) + I_2(t) - I_3(t)
%		\end{equation*}
%		\item In base al grafico ottenuto determina la funzione $I(t)$
%		\item Verifica la correttezza del risultato utilizzando i numeri complessi
%	\end{itemize}	

\end{questions}
