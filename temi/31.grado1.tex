\section{Modelli di primo  grado}

\subsection{Invertire formule} 
\begin{questions}
	
	\question Legge oraria del moto rettilineo uniforme MRU 
	
	\begin{equation*}
	x=v \cdot t + x_0
	\end{equation*}
	
	\begin{center}
		\renewcommand\arraystretch{1.2}	
		\begin{tabular}{lll}
			\hline
			variabile & grandezza & unità di misura \\
			\hline		
			$x$ & posizione & $\si{m}$ \\
			$v$ & velocità & $\si{\frac{m}{s}}$ \\
			$t$ & tempo trascorso & $\si{s}$ \\
			$x_o$ & posizione iniziale & $\si{m}$\\
			\hline
		\end{tabular}
	\end{center}
	
	
	
	\begin{parts}	
		\item Trova la formula per calcolare ogni variabile in funzione delle altre
		
		
		
		$
		\begin{aligned}
		v & =\\
		&\\
		t & =\\
		&\\
		x_0 & =
		\end{aligned}
		$
		
		
		
		\item Calcola i valori mancanti nella seguente tabella
		
		\bigskip
		
		\renewcommand\arraystretch{2.2}
		\begin{tabular}{|C{3cm}|C{3cm}|C{3cm}|C{3cm}|}
			\hline
			$x$ & $v$ & $t$ & $x_o$ \\
			\hline
			& $\SI{3}{\frac{m}{s}}$ & $\SI{13}{s}$ & $\SI{2.5}{m}$ \\
			%		\midrule
			$\SI{23}{km}$ &  & $\SI{3.2}{h}$ & $\SI{4}{km}$ \\
			%		\midrule
			$\SI{173}{m}$ & $\SI{2.7}{\frac{m}{s}}$ &  & $\SI{42}{m}$ \\
			%		\midrule
			$\SI{236}{km}$ & $\SI{123}{\frac{km}{h}}$ & $\SI{5}{h}$ &  \\
			%		\midrule
			$\SI{1323}{m}$ &  & $\SI{47}{min}$ & $\SI{300}{m}$ \\
			\hline
		\end{tabular}	
		
	\end{parts}
	
	
	
	\exnewpage
	\question Resistenza di un conduttore 
	
	\begin{equation*}
	R=\dfrac{\rho \cdot l}{A}
	\end{equation*}
	
	\begin{center}
		\renewcommand\arraystretch{1.2}	
		\begin{tabular}{lll}
			\hline
			variabile & grandezza & unità di misura \\
			\hline
			$R$ & resistenza del conduttore & $\si{\ohm}$ \\
			$\rho$ & resistività del materiale & $\si{\frac{\ohm \cdot mm^2}{m}}$\\
			$l$ & lunghezza del conduttore & $\si{m}$ \\
			$A$ & sezione del conduttore & $\si{mm^2}$ \\
			\hline
		\end{tabular}
	\end{center}
	
	
	\begin{parts}	
		\item Trova la formula per calcolare ogni variabile in funzione delle altre
		
		
		
		$
		\begin{aligned}
		\rho & =\\
		&\\
		l & =\\
		&\\
		A & =
		\end{aligned}
		$
		
		
		
		\item Calcola i valori mancanti nella seguente tabella
		
		\bigskip
		
		
		\renewcommand\arraystretch{2.2}
		\begin{tabular}{|C{3cm}|C{3cm}|C{3cm}|C{3cm}|}
			\hline
			$R$ & $\rho$ & $l$ & $A$ \\
			\hline
			& $\SI{0.0175}{\frac{\ohm \cdot mm^2}{m}}$ & $\SI{250}{m}$ & $\SI{0.5}{mm^2}$ \\
			%		\midrule
			$\SI{10}{\ohm}$ &  & $\SI{1.5}{km}$ & $\SI{2.5}{mm^2}$ \\
			%		\midrule
			$\SI{152}{m\ohm}$ &  $\SI{0.0287}{\frac{\ohm \cdot mm^2}{m}}$ &  & $\SI{0.4}{mm^2}$ \\
			%		\midrule
			$\SI{0.5}{\ohm}$ & $\SI{0.0175}{\frac{\ohm \cdot mm^2}{m}}$ & $\SI{3.5}{km}$ \\
			%		\midrule
			$\SI{2}{\ohm}$ &  & $\SI{105}{m}$ & $\SI{1.5}{mm^2}$ \\
			\hline
		\end{tabular}
		
		
		
	\end{parts}
	
\end{questions}

%
%
\exnewpage
\subsection{Equazioni di primo grado} 
\begin{questions}
	\question
	
	\exonly{ Risolvi le seguenti equazioni: }
	\begin{parts}
		\setlength\itemsep{3mm}
		\item \exonly{$x-3=2$ }
		\item \exonly{$3x=9$ }
		\item \exonly{$2x+3=7$ }
		\item \exonly{$3x-5=7-x$ }
		\item \exonly{$3x+2=5$ }
		\item \exonly{$5x+3=2x-7$ }
		\item \exonly{$2x+3-7x=3x-5+2x-1$ }
	\end{parts}	
	
	
	
	\question 
	\exonly{Risolvi le seguenti equazioni: }
	
	\begin{parts}
		\setlength\itemsep{3mm}
		\item \exonly{$2(3x-2)+3(x-1)=4(x-1)-(3-x)$ }
		\item \exonly{$\frac{2}{3}(5x+3)-2(\frac{1}{3}x-\frac{3}{4})=\frac{1}{2}(\frac{3}{5}-\frac{3}{7}x)$ }
		\item \exonly{$3(x-5)+2(3-2x)=x+3-2(x-4)$ }
		\item \exonly{$(x+3)(x-2)=x(x+3)-3(x-4)$ }
		\item \exonly{$2(3x-4)=4(x+1)+2(x-6)$ }
	\end{parts}
	
	
	
	\question
	\exonly{ Risolvi le seguenti equazioni nell'incognita $x$ e discuti le soluzioni rispetto al parametro $a$: }
	
	\begin{parts}
		\setlength\itemsep{3mm}
		\item \exonly{$2x+3-2a=3x-5+2a-1$ }
		\item \exonly{$2(3x-2a)+3(x-a)=4(x-a)-(3a-x)$ }
		\item \exonly{$3(x-5a)+2(3-2x)=x+3-2(x-4a)$ }
		\item \exonly{$(x+3)(x-a)=x(x+2a)-3(x-4)$ }
	\end{parts}
	
	
	
	
	
	
	\exnewpage
	\question
	\exonly{ Risolvi i seguenti sistemi nelle incognite $x$ e $y$: }
	
	\begin{minipage}{\textwidth}
	\begin{multicols}{2}
	\begin{parts}
		\setlength\itemsep{3mm}
		
		\item 
		\exonly{$
		\left\{
		\begin{aligned}
		x+3 & = 3x-1 \\
		x-y & = 5 
		\end{aligned}
		\right. $ }
	\solonly{ }
	
		\item
		\exonly{ $
		\left\{
		\begin{aligned}
		2x-3 & = 4-5x \\
		3x-2y & = 2 
		\end{aligned}
		\right. $ }	\\
	
		\item 
		\exonly{$
		\left\{
		\begin{aligned}
		2x-4 & = 3 \\
		-x+2y & = 5 
		\end{aligned}
		\right. $ }	
			
			
		\item 
		 \exonly{$
		\left\{
		\begin{aligned}
		5x-5 & = 1 \\
		-6x+9y & = -3 
		\end{aligned}
		\right. $ }		
	\end{parts}
		\end{multicols}
	\end{minipage}
	
	
	\question
\exonly{	Risolvi i  seguenti sistemi nelle incognite $x$ e $y$ e $z$.  }


\begin{minipage}{\textwidth}
	\begin{multicols}{2}		
		\begin{parts}
			\setlength\itemsep{3mm}
			\item 		
				\exonly{$\left\{
			\begin{aligned}
			3x-y & = 2 \\
			2x-5y & = 4 
			\end{aligned}
			\right. $ }
			\item 
			\exonly{$
			\left\{
			\begin{aligned}
			x+3y & = 3 \\
			2x+4y & = -8 
			\end{aligned}
			\right. $ }
			\item \exonly{$
			\left\{
			\begin{aligned}
			2x-3y & = 1 \\
			2x-5y & = 4 
			\end{aligned}
			\right. $ }
			\item \exonly{$
			\left\{
			\begin{aligned}
			3x+3y & = 3 \\
			2x-3y & = -2 
			\end{aligned}
			\right. $ }
			\item \exonly{$
			\left\{
			\begin{aligned}
			4x-3 & = 2 \\
			2x-6y & = 4 
			\end{aligned}
			\right. $ }
			\item \exonly{$
			\left\{
			\begin{aligned}
			2x+3y & = 3 \\
			6x+4y & = -3 
			\end{aligned}
			\right. $ }
			\item \exonly{$
			\left\{
			\begin{aligned}
			3x-4y & = 6 \\
			2x-3y & = -8 
			\end{aligned}
			\right. $ }
			\item \exonly{$
			\left\{
			\begin{aligned}
			4x+5y & = 3 \\
			2x+7y & = -8 
			\end{aligned}
			\right. $ }
			\item \exonly{$
			\left\{
			\begin{aligned}
			3x+5y & = -2a \\
			x+4y & = 4 
			\end{aligned}
			\right. $	 }		
			\item\exonly{ $
			\left\{
			\begin{aligned}
			2x+a\cdot y & = -6 \\
			2x-7y & = -a 
			\end{aligned}
			\right. $ }
		\end{parts}
	\end{multicols}
\end{minipage}

\end{questions}
\exnewpage


\subsection{Rappresentazione grafica }
\begin{questions}
%
\question
\exonly{Determinare la legge di assegnazione (l'equazione) delle rette) $a$, $b$, $c$, $d$ et $e$. }

%\includegraphics[scale=1]{ex-eq-droites-rev.png}
\exonly{
 \begin{tikzpicture}
\begin{axis}[
AxisDefaults,
SmallAxisLabels,
width=0.75\linewidth,
domain=-6:10,
restrict y to domain=-5:8,
xmin=-6,xmax=10, ymin=-4, ymax=7,
]
\addplot[]{5*x/2+3/2} node[above left, pos=0.8] {$a$} ; 
\addplot[]{-5*x/6+7/3}node[above, pos=0.15] {$b$} ; 
\addplot[]{2}node[above, pos=0.2] {$c$} ; 
\addplot[]{7*x/3-37/3}node[ right, pos=0.8] {$d$} ;
\addplot[]{-9*x/5+66/5}node[above right, pos=0.2] {$e$} ;
\end{axis}
\end{tikzpicture} }

\solonly{
	\begin{minipage}{\textwidth}
	
	
	\begin{multicols}{2}
	\begin{enumerate}[a)]
		\item $y=\frac{5}{2}x+\frac{3}{2}$
		\item $y=-\frac{5}{6}x+\frac{7}{3}$
		\item $y=2$
		\item $y=\frac{7}{3}x-\frac{37}{3}$
		\item $y=-\frac{9}{5}x+\frac{66}{5}$
		
	\end{enumerate}
	\end{multicols}
\end{minipage}

}
%
\question
\exonly{Determinare la legge di assegnazione (equazioni) delle rette qui sotto: }


\exonly{\begin{tikzpicture}
\begin{axis}[
AxisDefaults,
SmallAxisLabels,
width=0.75\linewidth,
domain=-8:10,
restrict y to domain=-8:8,
ymin=-6, ymax=7, %set the min and max y tick
xmin=-8, xmax=10,  %set the min and max x tick
]
\addplot[black]{-5*\x/6+1} node[above,pos=0.1] {$a$};
\addplot[black]{\x/5-4} node[above,pos=0.8] {$b$};
\addplot[black]{\x+4} node[right,pos=0.9] {$c$};
\addplot[black]{2*\x/5-3} node[above,pos=0.9] {$d$};
\addplot[black]{-6*\x +6} node[right,pos=0.8] {$e$};
\end{axis}
\end{tikzpicture} }


\solonly{
	\begin{minipage}{\textwidth}
	\begin{multicols}{2}
		
		
		\begin{enumerate}[a)]
			\item   $y= -\frac{5}{6}x+1$
			
			\item  $y=\frac{1}{5}x-4$ 
			
			\item  $y= x+4$
			\item  $y=\frac{2}{5}x-3$ 
			\item  $y=-6x+6 $ 
		\end{enumerate}
	\end{multicols}
\end{minipage}
}
\end{questions}

\exnewpage

\subsection{Determinare equazione dai dati }
\begin{questions}


\question
\exonly{Determinare l'equazione della retta passante per  A(-3;5) e B(-2;1). }

\solonly{$y=-4x-7$}
 
\question
\exonly{Determinare l'equazione della retta passante per A(-3;-8) e B(3;1). }

\solonly{$y=\dfrac{3}{2}x-\dfrac{7}{2}$}

\question
\exonly{Determinare l'equazione della retta passante per A($\frac{3}{4}$;1) e B(1;3). }

\solonly{$y=8x-5$}

\question
\exonly{Determinare l'equazione della retta passante per  A(4;1)  avente pendenza $m= \dfrac{1}{3}$. }

\solonly{$y=\dfrac{1}{3}x-\dfrac{1}{3}$}

%\question
%Dessiner la droite passant par le point A(1;2) et ayant pente $m= -\dfrac{2}{3}$. Quelle est son équation?
%
%\rsol{$y=-\dfrac{2}{3}x+\dfrac{8}{3}$}


\question
\exonly{Qual'é la pendenza della retta di equazione  $12x = 4y -1$? Qual'é la sua ordinata all'origine? }

\solonly{
	Pendenza: $m=3$\\	
	Ordinata all'origine: $b=\dfrac{1}{4}$
}

\question
\exonly{Determinare l'equazione della retta passante per  A(-2;1) che sia parallela alla retta di equazione  $21y+42x=7$. }

\solonly{$y= -2x-3$}

%\question
%Déterminer l'équation de la droite passant par A(-5;5) et perpendiculaire à la droite d'équation $3y-7x+4=0$.
%
%\rsol{$y= -\dfrac{3}{7}x+\dfrac{20}{7}$}
\end{questions}

\subsection{Intersezioni}
\begin{questions}
%%%%%%%%%%%%%%%%%%%%%%%%%%%%%%
\question

\exonly{Determinare i punti d'intersezione della retta di equazione $5x + 6y = 8$ con : }

\begin{parts}
	\part 
		\exonly{l'asse delle ascisse $x$ }
		\solonly{Asse  x: $y=0 \Rightarrow (\dfrac{8}{5};0)$ }
	\part  
	\exonly{l'asse delle ordinate $y$ }
	\solonly{Asse y: $x=0 \Rightarrow (0;\dfrac{4}{3})$ }
\end{parts}



%%%%%%%%%%%%%%%%%%%%%%%%%%%%%%%%%%%%%%%%%%%%%%%
\question
\exonly{Determinare i punti d'intersezione della retta di equazione $6x - 3y = 16$ con: }

\begin{parts}
	\part \exonly{l'asse delle ascisse $x$ } \solonly{$\left( \dfrac{8}{3};0\right)$ }
	\part  \exonly{l'asse delle ordinate $y$ } \solonly{$\left( 0;-\dfrac{16}{3}\right) $ }
\part \exonly{la retta di equazione  $y=4x-3$ } \solonly{$\left( -\dfrac{7}{6};-\dfrac{23}{3}\right) $ }
\part \exonly{la retta di equazione $5x + 6y = 8$ } \solonly{$\left( \dfrac{40}{17};-\dfrac{32}{51}\right) $ }
\end{parts}


\end{questions}

\subsection{Problemi}
\begin{questions}
	

\question
\exonly{Gérard si é appena trasferito  dal canton Friburgo e vuole provare ad importare la famosa "double crème de la Gruyère" in Ticino. Ha calcolato che i suo costi fissi quotidiani sono di \SI{64 }{\CHF} e che ogni confezione di "double crème" gli costa \SI{12}{\CHF}. Gérard decide di vendere ad un prezzo di \SI{20}{\CHF} la confezione da mezzo chilo. }

\begin{parts}
\part
\exonly{ Quante confezioni dovrebbe vedere in media al giorno per coprire i  costi. }

\solonly{ Dovrebbe vendere almeno 8 confezioni.}

\exonly{\uplevel{
Un rappresentante di propone a Gérard un nuova macchina per gestire l'imballaggio. Se decidesse di investire in questa innovazione i suoi costi fissi salirebbero a \SI{144}{\CHF} ma in compenso il costo per confezione scenderebbe a \SI{8}{\CHF}. Gérard vuole sempre vendere a \SI{20}{\CHF} ogni confezione. 
} }
\part
\exonly{Quali scenari si presentano a Gérard e quale decisione dovrebbe prendere in funzione del numero di confezione vendute, in media, quotidianamente. }

\solonly{$<8:$ chiudere \\ $8\leq x \leq 20$: mantenere sistema attuale \\ $x>20$ innovare il sistema di imballaggio }

\end{parts}

\end{questions}