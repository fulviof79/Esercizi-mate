\section{Modelli di primo  grado}

\subsection{Invertire formule} 
\begin{questions}
	
	\question 
	\exonly{Legge oraria del moto rettilineo uniforme MRU 
	
	\begin{equation*}
	x=v \cdot t + x_0
	\end{equation*}
	
	\begin{center}
		\renewcommand\arraystretch{1.2}	
		\begin{tabular}{lll}
			\hline
			variabile & grandezza & unità di misura \\
			\hline		
			$x$ & posizione & $\si{m}$ \\
			$v$ & velocità & $\si{\frac{m}{s}}$ \\
			$t$ & tempo trascorso & $\si{s}$ \\
			$x_o$ & posizione iniziale & $\si{m}$\\
			\hline
		\end{tabular}
	\end{center}
	 }
	
	
	\begin{parts}	
		\part \exonly{Trova la formula per calcolare ogni variabile in funzione delle altre }
		
		$v  =  \solonly{\dfrac{x-x_0}{t} }$ \\
	$	t  = \solonly{\dfrac{x-x_0}{v} }$ \\
		$	x_0  = \solonly{x-vt} 
			$ 

		
		
		
		\part \exonly{Calcola i valori mancanti nella seguente tabella }
		
		\bigskip
		
		\renewcommand\arraystretch{2.2}
		\begin{tabular}{|C{3cm}|C{3cm}|C{3cm}|C{3cm}|}
			\hline
			             $x$              &                   $v$                   &      $t$       &     $x_o$     \\
			\hline
			\solonly{\SI{41.5}{\metre}  } &          $\SI{3}{\frac{m}{s}}$          &  $\SI{13}{s}$  & $\SI{2.5}{m}$ \\
			%		\midrule
			        $\SI{23}{km}$         & \solonly{$\SI{5.94}{\frac{km}{h}}  $  } & $\SI{3.2}{h}$  & $\SI{4}{km}$  \\
			%		\midrule
			        $\SI{173}{m}$         &         $\SI{2.7}{\frac{m}{s}}$         &    \solonly{$\SI{48.5}{s}$ }            & $\SI{42}{m}$  \\
			%		\midrule
			       $\SI{236}{km}$         &        $\SI{123}{\frac{km}{h}}$         &  $\SI{5}{h}$   &  \solonly{$\SI{-379}{km}$ }             \\
			%		\midrule
			       $\SI{1323}{m}$         &         \solonly{$\SI{21.77}{\frac{m}{min}}$ }                                & $\SI{47}{min}$ & $\SI{300}{m}$ \\
			\hline
		\end{tabular}	
		
	\end{parts}
	
	
	
	\exnewpage
	\question \exonly{Resistenza di un conduttore  
	
	\begin{equation*}
	R=\dfrac{\rho \cdot l}{A}
	\end{equation*}
	
	
	\begin{center}
		\renewcommand\arraystretch{1.2}	
		\begin{tabular}{lll}
			\hline
			variabile & grandezza & unità di misura \\
			\hline
			$R$ & resistenza del conduttore & $\si{\ohm}$ \\
			$\rho$ & resistività del materiale & $\si{\frac{\ohm \cdot mm^2}{m}}$\\
			$l$ & lunghezza del conduttore & $\si{m}$ \\
			$A$ & sezione del conduttore & $\si{mm^2}$ \\
			\hline
		\end{tabular}
	\end{center}
}
	
	
	\begin{parts}	
		\part  \exonly{Trova la formula per calcolare ogni variabile in funzione delle altre }
		
		$\rho= \solonly{\dfrac{AR}{l} }$\\
		$l=\solonly{\dfrac{AR}{\rho} }$\\		
		$A=\solonly{\dfrac{\rho l }R }$

		
		
		
		\part \exonly{Calcola i valori mancanti nella seguente tabella }
		
		\bigskip
		
		
		\renewcommand\arraystretch{2.2}
		\begin{tabular}{|C{3cm}|C{3cm}|C{3cm}|C{3cm}|}
			\hline
			            $R$              &                       $\rho$                        &              $l$               &       $A$        \\
			\hline
			\solonly{$\SI{8.75}{\ohm} $} &      $\SI{0.0175}{\frac{\ohm \cdot mm^2}{m}}$       &         $\SI{250}{m}$          & $\SI{0.5}{mm^2}$ \\
			%		\midrule
			      $\SI{10}{\ohm}$        & \solonly{$\SI{0.017}{\frac{\ohm \cdot mm^2}{m}}$  } &         $\SI{1.5}{km}$         & $\SI{2.5}{mm^2}$ \\
			%		\midrule
			     $\SI{152}{m\ohm}$       &      $\SI{0.0287}{\frac{\ohm \cdot mm^2}{m}}$       & \solonly{$\SI{2.12}{\metre}$ } & $\SI{0.4}{mm^2}$ \\
			%		\midrule
			      $\SI{0.5}{\ohm}$       &      $\SI{0.0175}{\frac{\ohm \cdot mm^2}{m}}$       &         $\SI{3.5}{km}$         &   \solonly{ $\SI{122.5}{mm^2}$ }               \\
			%		\midrule
			       $\SI{2}{\ohm}$        &     \solonly{$\SI{0.0286}{\frac{\ohm \cdot mm^2}{m}}$  }                                                &         $\SI{105}{m}$          & $\SI{1.5}{mm^2}$ \\
			\hline
		\end{tabular}
		
		
		
	\end{parts}
	
\end{questions}

%
%
\exnewpage
\solnewpage
\subsection{Equazioni e sistemi di primo grado} 
\begin{questions}
	\question
	
	\exonly{ Risolvi le seguenti equazioni: }
	\begin{parts}
		\setlength\itemsep{3mm}
		\item 
		\exonly{$x-3=2$ }
		\solonly{$\es{5}$}
		
		\item \exonly{$3x=9$ } \solonly{$\es{3 }$} 
		\item \exonly{$2x+3=7$ } \solonly{$\es{2 }$ }
		\item \exonly{$3x-5=7-x$ } \solonly{$\es{3 }$ }
		\item \exonly{$3x+2=5$ } \solonly{$\es{1 }$ }
		\item \exonly{$5x+3=2x-7$ } \solonly{$\es{-\frac{10}{3} }$ }
		\item \exonly{$2x+3-7x=3x-5+2x-1$ } \solonly{$\es{\frac{9}{10} }$ }
	\end{parts}	
	
	
	
	\question 
	\exonly{Risolvi le seguenti equazioni: }
	
	\begin{parts}
		\setlength\itemsep{3mm}
		\item \exonly{$2(3x-2)+3(x-1)=4(x-1)-(3-x)$ }
		\solonly{$\es{0 }$ }
		
		\item \exonly{$\frac{2}{3}(5x+3)-2(\frac{1}{3}x-\frac{3}{4})=\frac{1}{2}(\frac{3}{5}-\frac{3}{7}x)$ }
		\solonly{$\es{-\frac{672}{605} }$ }
		
		\item \exonly{$3(x-5)+2(3-2x)=x+3-2(x-4)$ }
		\solonly{$S=\emptyset$ }
		
		\item \exonly{$(x+3)(x-2)=x(x+3)-3(x-4)$ }
		\solonly{$\es{18 }$ }
		\item \exonly{$2(3x-4)=4(x+1)+2(x-6)$ }
			\solonly{$S=\R$ }
	\end{parts}
	
	
	
	\question
	\exonly{ Risolvi le seguenti equazioni nell'incognita $x$ e discuti le soluzioni rispetto al parametro $a$: }
	
	\begin{parts}
		\setlength\itemsep{3mm}
		\item \exonly{$2x+3-2a=3x-5+2a-1$ } \solonly{$\es{9-4a \mid a \in \R }$ }
		\item \exonly{$2(3x-2a)+3(x-a)=4(x-a)-(3a-x)$ } \solonly{$\es{0 \mid a \in \R }$ }
		\item \exonly{$3(x-5a)+2(3-2x)=x+3-2(x-4a)$ } \solonly{Se $a=\frac{3}{23}$: $S=\R$, se $a\neq\frac{3}{23}$: $S=\emptyset$ }
		\item \exonly{$(x+3)(x-a)=x(x+2a)-3(x-4)$ } \solonly{$\es{\frac{a+4}{2-a} \mid a \neq 2 }$ , se $a=2$: $S=\emptyset$ }
	\end{parts}
	
	
	
	
	
	
	\exnewpage
	\question
	\exonly{Risolvere i seguenti sistemi. }
	
	\begin{minipage}{\textwidth}
	\begin{multicols}{2}
	\begin{parts}
		\setlength\itemsep{3mm}
		
		\item 
		\exonly{$
		\left\{
		\begin{aligned}
		x+3 & = 3x-1 \\
		x-y & = 5 
		\end{aligned}
		\right. $ }
		\solonly{$\es{\left(2;-2\right)}$ }
	
		\item
		\exonly{ $
		\left\{
		\begin{aligned}
		2x-3 & = 4-5x \\
		3x-2y & = 2 
		\end{aligned}
		\right. $ }
		\solonly{$\es{\left( 1;\frac{1}{2}\right)  }$ }
	
		\item 
		\exonly{$
		\left\{
		\begin{aligned}
		2x-4 & = 3 \\
		-x+2y & = 5 
		\end{aligned}
		\right. $ }	
			\solonly{$\es{\left( \frac{7}{2};\frac{17}{4}\right) }$ }


			
			
		\item 
		 \exonly{$
		\left\{
		\begin{aligned}
		5x-5 & = 1 \\
		-6x+9y & = -3 
		\end{aligned}
		\right. $ }	
	
			\solonly{$\es{\left( \frac{6}{5};\frac{7}{15}\right)  }$ }	
	\end{parts}
		\end{multicols}
	\end{minipage}
	
	
	\question
\exonly{Risolvere i  seguenti sistemi e indicare l'insieme delle soluzioni. }


\begin{minipage}{\textwidth}
\begin{multicols}{2}		
\begin{parts}
\setlength\itemsep{3mm}
\item 		
\exonly{
	$\left\{
	\begin{aligned}
	3x-y & = 2 \\
	2x-5y & = 4 
	\end{aligned}
	\right. $ 
}

\solonly{
	$\es{\left( \frac{6}{13};-\frac{8}{13}\right)  }$ 
}

\item 
\exonly{$
\left\{
\begin{aligned}
x+3y & = 3 \\
2x+4y & = -8 
\end{aligned}
\right. $ }

\solonly{
	$\es{\left( -18;7\right)  }$ 
}
\item \exonly{$
\left\{
\begin{aligned}
2x-3y & = 1 \\
2x-5y & = 4 
\end{aligned}
\right. $ }

\solonly{
	$\es{\left( -\frac{7}{4};-\frac{3}{2}\right)  }$ 
}
\item 
\exonly{$
\left\{
\begin{aligned}
3x+3y & = 3 \\
2x-3y & = -2 
\end{aligned}
\right. $ }

\solonly{
	$\es{\left( \frac{1}{5};\frac{4}{5}\right)  }$ 
}		
		
\item \exonly{$
\left\{
\begin{aligned}
4x-3 & = 2 \\
2x-6y & = 4 
\end{aligned}
\right. $ }

\solonly{
	$\es{\left( \frac{5}{4};-\frac{1}{4}\right)  }$ 
}		

\item \exonly{$
\left\{
\begin{aligned}
2x+3y & = 3 \\
6x+4y & = -3 
\end{aligned}
\right. $ }

\solonly{
	$\es{\left(- \frac{21}{10};\frac{12}{5}\right)  }$ 
}		



\item \exonly{$
\left\{
\begin{aligned}
3x-4y & = 6 \\
2x-3y & = -8 
\end{aligned}
\right. $ }

\solonly{
	$\es{\left(-14;-12\right)  }$ 
}		



\item \exonly{$
\left\{
\begin{aligned}
4x+5y & = 3 \\
2x+7y & = -8 
\end{aligned}
\right. $ }

\solonly{
	$\es{\left(\frac{61}{18};-\frac{19}{9}\right)  }$ 
}		

		
		
\end{parts}
\end{multicols}
\end{minipage}

\question 
\exonly{Risolvere i seguenti sistemi e indicare l'insieme delle soluzioni discutendo il parametro $a$.}

\begin{parts}
	\item \exonly{$
\left\{
\begin{aligned}
3x+5y & = -2a \\
x+4y & = 4 
\end{aligned}
\right. $	 }	

\solonly{
	$\es{\left(- \frac{8a+20}{7};\frac{2a+12}{7}\right) \mid a \in \R }$ 
}		
	
\item\exonly{ $
\left\{
\begin{aligned}
2x+a\cdot y & = -6 \\
2x-7y & = -a 
\end{aligned}
\right. $ }

\solonly{
Se $a \neq -7$: 	$\es{\left(\frac{-a^2 - 42}{2a+14}; \frac{a-6}{a+7}\right) }$  \\
Se $a = -7$: 	$S=\emptyset $  \\
}	



\end{parts}


\question
\exonly{Risolvere i seguenti sistemi}

\begin{minipage}{\textwidth}
	\begin{multicols}{2}		
		\begin{parts}
		\part 
		\exonly{$
		\left\{
		\begin{aligned}
		x+y+z & = 3 \\
		x-y-z & = 4 \\
		-x+y-z &= -2 
		\end{aligned}
		\right. $ }
	
	\solonly{$\es{\left(\frac{7}{2}; \frac{1}{2};-1\right) }$   }
	
	
		\part
		\exonly{ $
		\left\{
		\begin{aligned}
		3x-4y+z & = 2 \\
		2x+2y-3z & = 1 \\
		3x-3y+2z &= -2 
		\end{aligned}
		\right. $ }
	
	\solonly{$\es{\left(-1; -\frac{9}{5};-\frac{11}{5}\right) }$   }
	
\part
\exonly{ $
\left\{
\begin{aligned}
2x - 3y + z & =  x -  y - z \\
x + 2y - z & =  2 - 3x \\
2x + 4y +3z & = -x +z + 1
\end{aligned}
\right. $ }

\solonly{$\es{\left(\frac{11}{25}; \frac{1}{50};-\frac{1}{5}\right) }$   }

\end{parts}
\end{multicols}
\end{minipage}


\end{questions}
\exnewpage
\solnewpage

\subsection{Funzioni e rappresentazione grafica }
\begin{questions}
%
\question
\exonly{Determinare la legge di assegnazione (l'equazione) delle rette) $a$, $b$, $c$, $d$ et $e$. }

%\includegraphics[scale=1]{ex-eq-droites-rev.png}
\exonly{
 \begin{tikzpicture}
\begin{axis}[
AxisDefaults,
SmallAxisLabels,
width=0.75\linewidth,
domain=-6:10,
restrict y to domain=-5:8,
xmin=-6,xmax=10, ymin=-4, ymax=7,
]
\addplot[]{5*x/2+3/2} node[above left, pos=0.8] {$a$} ; 
\addplot[]{-5*x/6+7/3}node[above, pos=0.15] {$b$} ; 
\addplot[]{2}node[above, pos=0.2] {$c$} ; 
\addplot[]{7*x/3-37/3}node[ right, pos=0.8] {$d$} ;
\addplot[]{-9*x/5+66/5}node[above right, pos=0.2] {$e$} ;
\end{axis}
\end{tikzpicture} }

\solonly{
	\begin{minipage}{\textwidth}
	
	
	\begin{multicols}{2}
	\begin{enumerate}[a)]
		\item $y=\frac{5}{2}x+\frac{3}{2}$
		\item $y=-\frac{5}{6}x+\frac{7}{3}$
		\item $y=2$
		\item $y=\frac{7}{3}x-\frac{37}{3}$
		\item $y=-\frac{9}{5}x+\frac{66}{5}$
		
	\end{enumerate}
	\end{multicols}
\end{minipage}

}
%
\question
\exonly{Determinare la legge di assegnazione (equazioni) delle rette qui sotto: }


\exonly{\begin{tikzpicture}
\begin{axis}[
AxisDefaults,
SmallAxisLabels,
width=0.75\linewidth,
domain=-8:10,
restrict y to domain=-8:8,
ymin=-6, ymax=7, %set the min and max y tick
xmin=-8, xmax=10,  %set the min and max x tick
]
\addplot[black]{-5*\x/6+1} node[above,pos=0.1] {$a$};
\addplot[black]{\x/5-4} node[above,pos=0.8] {$b$};
\addplot[black]{\x+4} node[right,pos=0.9] {$c$};
\addplot[black]{2*\x/5-3} node[above,pos=0.9] {$d$};
\addplot[black]{-6*\x +6} node[right,pos=0.8] {$e$};
\end{axis}
\end{tikzpicture} }


\solonly{
	\begin{minipage}{\textwidth}
	\begin{multicols}{2}
		
		
		\begin{enumerate}[a)]
			\item   $y= -\frac{5}{6}x+1$
			
			\item  $y=\frac{1}{5}x-4$ 
			
			\item  $y= x+4$
			\item  $y=\frac{2}{5}x-3$ 
			\item  $y=-6x+6 $ 
		\end{enumerate}
	\end{multicols}
\end{minipage}
}
%\end{questions}

\exnewpage

%\subsection{Determinare equazione dai dati }
%\begin{questions}


\question
\exonly{Determinare l'equazione della retta passante per  A(-3;5) e B(-2;1). }

\solonly{$y=-4x-7$}
 
\question
\exonly{Determinare l'equazione della retta passante per A(-3;-8) e B(3;1). }

\solonly{$y=\dfrac{3}{2}x-\dfrac{7}{2}$}

\question
\exonly{Determinare l'equazione della retta passante per A($\frac{3}{4}$;1) e B(1;3). }

\solonly{$y=8x-5$}

\question
\exonly{Determinare l'equazione della retta passante per  A(4;1)  avente pendenza $m= \dfrac{1}{3}$. }

\solonly{$y=\dfrac{1}{3}x-\dfrac{1}{3}$}

%\question
%Dessiner la droite passant par le point A(1;2) et ayant pente $m= -\dfrac{2}{3}$. Quelle est son équation?
%
%\rsol{$y=-\dfrac{2}{3}x+\dfrac{8}{3}$}


\question
\exonly{Qual'é la pendenza della retta di equazione  $12x = 4y -1$? Qual'é la sua ordinata all'origine? }

\solonly{
	Pendenza: $m=3$\\	
	Ordinata all'origine: $b=\frac{1}{4}$
}

\question
\exonly{Determinare l'equazione della retta passante per  A(-2;1) che sia parallela alla retta di equazione  $21y+42x=7$. }

\solonly{$y= -2x-3$}

\question
\exonly{Determinare l'equazione della retta passante per  A(-5;5) che sia perpendicolare alla retta di equazione  $3y-7x+4=0$. }

\solonly{$y= -\frac{3}{7}x+\frac{20}{7}$}

%
\question

\exonly{Determinare i punti d'intersezione della retta di equazione $5x + 6y = 8$ con : }

\begin{parts}
	\part 
		\exonly{l'asse delle ascisse $x$ }
		\solonly{Asse  x: $y=0 \Rightarrow \left( \frac{8}{5};0\right) $ }
	\part  
	\exonly{l'asse delle ordinate $y$ }
	\solonly{Asse y: $x=0 \Rightarrow \left( 0;\frac{4}{3}\right) $ }
\end{parts}



%%%%%%%%%%%%%%%%%%%%%%%%%%%%%%%%%%%%%%%%%%%%%%%
\question
\exonly{Determinare i punti d'intersezione della retta di equazione $6x - 3y = 16$ con: }

\begin{parts}
	\part \exonly{l'asse delle ascisse $x$ } \solonly{$\left( \dfrac{8}{3};0\right)$ }
	\part  \exonly{l'asse delle ordinate $y$ } \solonly{$\left( 0;-\dfrac{16}{3}\right) $ }
\part \exonly{la retta di equazione  $y=4x-3$ } \solonly{$\left( -\dfrac{7}{6};-\dfrac{23}{3}\right) $ }
\part \exonly{la retta di equazione $5x + 6y = 8$ } \solonly{$\left( \dfrac{40}{17};-\dfrac{32}{51}\right) $ }
\end{parts}


\end{questions}

\newpage
\subsection{Problemi}
\begin{questions}
	

\question
\exonly{Gérard si é appena trasferito  dal canton Friburgo e vuole provare ad importare la famosa "double crème de la Gruyère" in Ticino. Ha calcolato che i suo costi fissi quotidiani sono di \SI{64 }{\CHF} e che ogni confezione di "double crème" gli costa \SI{12}{\CHF}. Gérard decide di vendere ad un prezzo di \SI{20}{\CHF} la confezione da mezzo chilo. }

\begin{parts}
\part
\exonly{ Quante confezioni dovrebbe vedere in media al giorno per coprire i  costi. }

\solonly{ Dovrebbe vendere almeno 8 confezioni.}

\exonly{\uplevel{
Un rappresentante di propone a Gérard un nuova macchina per gestire l'imballaggio. Se decidesse di investire in questa innovazione i suoi costi fissi salirebbero a \SI{144}{\CHF} ma in compenso il costo per confezione scenderebbe a \SI{8}{\CHF}. Gérard vuole sempre vendere a \SI{20}{\CHF} ogni confezione. 
} }
\part
\exonly{Quali scenari si presentano a Gérard e quale decisione dovrebbe prendere in funzione del numero di confezione vendute, in media, quotidianamente. }

\solonly{$<8:$ chiudere \\ $8\leq x \leq 20$: mantenere sistema attuale \\ $x>20$: rinnovare il sistema di imballaggio }

\end{parts}

%%%%%%%%%%%%%%%%%%%%%%%%%%%%%%%%%%%%%%%%%%%%%%%%%%%%%%%%%%%%%%%%%%
\question
\exonly{
Seicento persone hanno assistito ad uno spettacolo teatrale. I prezzi dei biglietti erano di \SI{5}{\CHF} per gli adulti e di \SI{2}{\CHF} per i bambini.
A fine rappresentazione in cassa ci  sono \SI{2400}{\CHF}. 
Quanti adulti e quanti bambini hanno assistito alla rappresentazione?
%Six cents personnes assistent à la première d'un film. Les billets pour adultes coûtent \SI{5}{\CHF}, et les enfants sont admis pour \SI{2}{\CHF}. Si la caisse contient \SI{2400}{\CHF}, combien d'enfants assistaient à la première? 

}
\solonly{
$200$ bambini e $400$ adulti.
}







%%%%%%%%%%%%%%%%%%%%%%%%%%%%%%%%%%%%%%%%%%%%%%%%%%%%%%%%%%%%%%%%%%
\question
\exonly{
Nell'ambito di un test medico che mira a misurare la tolleranza ai carboidrati si somministrano ad un adulto \SI{7}{\centi\litre}  di una soluzione con una concentrazione del 30\% di glucosio.

Una tale concentrazione non é applicabile nel caso debba essere somministrata ad un bambino. Per un bambino bisognerebbe somministrare \SI{7}{\centi\litre} ad una concentrazione del 20\%.

Quanta soluzione per adulti bisognerà diluire per ottenere tale scopo?



}
\solonly{
$\dfrac{14}{3}\approx \SI{4.67}{\centi\litre}$ de soluzione e $\dfrac{7}{3} \approx \SI{2.33}{\centi\litre}$ d'acqua.
}






%%%%%%%%%%%%%%%%%%%%%%%%%%%%%%%%%%%%%%%%%%%%%%%%%%%%%%%%%%%%%%%%%%
\question
\exonly{

Un farmacista deve preparare \SI{15}{\milli\litre} di gocce speciali per gli occhi per un paziente affetto da glaucoma.
La soluzione deve contenere il 2\% di un particolare principio attivo ma il farmacista ne ha attualmente in stock solo in soluzione. Una soluzione al 10\% e una all'1\%.

Quali quantità delle due soluzioni deve utilizzare per soddisfare la prescrizione del medico?

%Un pharmacien doit préparer \SI{15}{\milli\litre} de gouttes spéciales pour les yeux pour un patient atteint d'un glaucome. 
%La solution doit contenir 2\% d'un élément actif, mais le pharmacien n'a en stock qu'une solution à 10\% et une solution à 1\%. 

%Combien de solution de chaque sorte doit être utilisée pour satisfaire la prescription ?

}
\solonly{
$\dfrac{5}{3}\approx \SI{1.67}{\milli\litre}$ della soluzione al 10\% e $\dfrac{40}{3}\approx \SI{13.33}{\milli\litre}$  all'1\%
}






%%%%%%%%%%%%%%%%%%%%%%%%%%%%%%%%%%%%%%%%%%%%%%%%%%%%%%%%%%%%%%%%%%%
\question \exonly{

La théophylline, médicament contre l'asthme, est préparée à partir d'un élixir contenant une concentration de \SI{5}{\milli\gram/\milli\litre} d'un produit et d'un sirop parfumé à la cerise pour masquer le goût du produit. 

Combien de chaque ingrédient doit-on utiliser pour préparer \SI{100}{\milli\litre} de solution avec une concentration de \SI{2}{\milli\gram/\milli\litre} ?

}
\solonly{
\SI{40}{\milli\litre} d'élixir et \SI{60}{\milli\litre} de sirop
}






%%%%%%%%%%%%%%%%%%%%%%%%%%%%%%%%%%%%%%%%%%%%%%%%%%%%%%%%%%%%%%%%%%
\question \exonly{
Abbiamo a disposizione una grande quantità di una lega rame--argento al \num{7.5}\% di rame e del rame puro.
%Le métal de la livre anglaise est un alliage cuivre-argent à \num{7.5}\% de cuivre. 
Quanti grammi di questa lega e quanti grammi di rame deve mischiare per preparare \SI{200}{\gram} di una lega rame--argento al 10\% di rame?
%Combien de grammes de cuivre pur et combien de grammes d'alliage de la %livre devrait-on utiliser pour préparer \SI{200}{\gram}  d'alliage cuivre--argent à 10\% de cuivre ?

}
\solonly{
\SI{194.6}{\gram} di lega e \SI{5.4}{\gram} di rame puro.
}








%%%%%%%%%%%%%%%%%%%%%%%%%%%%%%%%%%%%%%%%%%%%%%%%%%%%%%%%%%%%%%%%%%
\question \exonly{
Un corridore comincia  un percorso di allenamento e corre alla velocità costante di \SI{9.7}{\kilo\meter/\hour}.
Cinque minuti più tardi un secondo corridore comincia lo stesso percorso ma corre alla velocità costante di  \SI{12.9}{\kilo\meter/\hour}.

Quanto tempo occorre al secondo corridore per raggiungere il primo?

%Dopo quanto tempo il secondo corridore raggiungerà il primo?
%
%Un coureur part du début d'un parcours d'entraînement et court à la vitesse constante de \SI{9.7}{\kilo\meter/\hour}. 
%Cinq minutes plus tard, un autre coureur part du même point, il court à \SI{12.9}{\kilo\meter/\hour} et suit le même chemin. 
%
%Combien de temps faudra-t-il au second coureur pour rattraper le premier ? 

}
\solonly{
Il seocndo corridore raggiungerà il primo dopo aver corso ca. 15 minuti.
%Le second coureur nécessitera ca. 20 minutes pour rattraper le premier.
}






%%%%%%%%%%%%%%%%%%%%%%%%%%%%%%%%%%%%%%%%%%%%%%%%%%%%%%%%%%%%%%%%%%%
\question \exonly{
A \SI{6}{\hour}, un chasse-neige avançant à vitesse constante commence à déblayer une autoroute menant hors de ville. A \SI{8}{\hour}, une voiture emprunte l'autoroute à une vitesse de \SI{48}{\kilo\meter / \hour} et rejoint le chasse-neige $30$ minutes plus tard. 

Trouver la vitesse du chasse-neige.

}
\solonly{
La vitesse du chasse-neige est de \SI{9.6}{\kilo\meter / \hour}
}





\question
\exonly{
Eduardo deve camminare \SI{3}{\kilo\meter} per raggiungere la casa del suo amici Jason.
Al ritorno, Eduardo, prende in prestito la bicicletta di Jason. La sua velocità media in bicicletta supera di  \SI{4}{\kilo\meter/\hour}  quella a piedi e Eduardo impiega in tutto \SI{2}{\hour}  per il tragitto andata e ritorno.

A quale velocità cammina Eduardo?

%Quali sono le velocità nei due tragitti (a piedi e in bicicletta)?
%
%
%Depuis ça maison Édouard doit marcher pendant \SI{3}{\kilo\meter} pour aller chez son copain Jason. \\
%Pour le trajet de retour il emprunte le vélo de Jason. Sa vitesse moyenne à vélo est de \SI{4}{\kilo\meter/\hour} plus rapide qu'en marchant et il a mesuré que l'allez--retour lui à pris \SI{2}{\hour} en tout. \\
%Quelle est la vitesse moyenne de marche de   Édouard?
}

\solonly{
Eduardo cammina ad una velocità di \SI{2}{\kilo\meter/\hour}

}




\question
\exonly{La città A dista 40 km dalla città B e questa dista 160 km dalla città C  Un'automobile parte da B diretta verso C, viaggiando a velocità costante. Contemporaneamente parte da A una seconda automobile che viaggia ad una velocità costante superiore di 20 km/h a quella della prima auto e, passando per B, giunge in C contemporaneamente alla prima auto. 

Determina le velocità delle due auto.}

\solonly{80 e 100 Km/h}

%%%%%%%%%%%%%%%%%%%%%%%%%%%%%%%%%%%%%%%%%%%%%%%%%%%%%%%%%%%%%%%%%%%
\question
\exonly{

Un garçon tond le gazon en 90 minutes, mais sa sœur peut le faire en 60 minutes. Combien leur faudrait-il de temps s'ils tondaient ensemble avec deux tondeuses ? 

}
\solonly{
Il leur faudrait $36$ minutes.

}




%%%%%%%%%%%%%%%%%%%%%%%%%%%%%%%%%%%%%%%%%%%%%%%%%%%%%%%%%%%%%%%%%%
\question
\exonly{%\FR{
En travaillant ensemble Jean, Samuel et Françoise ont eu besoin de 2 heures pour peindre la salle de concerts du village. En travaillant seul Samuel aurait eu besoin de 6 heures pour effectuer le même travail. Françoise, seule, l’aurait fait en 4 heures.

Déterminer en combien de temps Jean, à lui tout seul, aurait pu effectuer le même travail. 
}

\solonly{
Jean nécessite de 12 heures pour peindre la maison tout seul. 
}


%%%%%%%%%%%%%%%%%%%%%%%%%%%%%%%%%%%%%%%%%%%%%%%%%%%%%%%%%%%%%%%%%%
\question \exonly{
Abbiamo a disposizione due tubi da utilizzare per il riempimento di una piscina. Uno di piccola portata con il quale sappiamo di poter riempire la piscina in 8 ore. Il secondo tubo di maggiore portata ci permetterebbe di riempire la piscina in 5 ore.

In quanto tempo potremmo riempire la piscina utilizzandoli contemporaneamente?  
%Utilizzando u
%Avec un tuyau d'arrosage, une piscine peut être remplie en 8 heures. Avec un autre tuyau plus gros, employé seul, on la remplit en 5 heures. 
%
%En combien de temps remplira-t-on la piscine en employant simultanément les deux tuyaux ?

}
\solonly{
$\dfrac{40}{13}\approx \SI{3.08}{\hour}$.

}



%%%%%%%%%%%%%%%%%%%%%%%%%%%%%%%%%%%%%%%%%%%%%%%%%%%%%%%%%%%%%%%%%%%
\question \exonly{}

\solonly{

Le Deep Blue nécessite de $5$ heures pour parcourir \SI{170}{\kilo\meter}, il a donc une vitesse moyenne de \SI{34}{\kilo\meter / \hour}.





}




%%%%%%%%%%%%%%%%%%%%%%%%%%%%%%%%%%%%%%%%%%%%%%%%%%%%%%%%%%%%%%%%%%
\question \exonly{
Un conducente effettua lo stesso tragitto di \SI{120}{\kilo\meter} una volta di giorno e una volta di notte.
Constata così che di notte impiega 36 minuti in meno che di giorno e che la sua velocità media é di \SI{10}{\kilo\meter / \hour} superiore a quelle di giorno.

Qual'é la sua velocità media di giorno?

%Un conducteur effectue le même trajet de \SI{120}{\kilo\meter} une fois de jour et une fois de nuit. Il constate qu’il lui faut 36 minutes de moins pour faire le trajet de nuit et que sa vitesse moyenne est alors de \SI{10}{\kilo\meter / \hour} supérieure à celle de jour. 
%En effet il y a moins de circulation la nuit et par conséquence le trafic est plus fluide.
%
%Quelle est sa vitesse moyenne lors du trajet de jour ?



}

\solonly{

%La vitesse moyenne pendant le jour est de 
\SI{40}{\kilo\meter / \hour}.


}




\end{questions}