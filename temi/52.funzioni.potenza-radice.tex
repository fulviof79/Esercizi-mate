\section{Funzioni potenza e radice}
\subsection{Dominio, immagini, inversa} 
\begin{questions}
	
	
	
	
	
	
	\question
	\exonly{Disegnare la rappresentazione grafica di  $f$ e della sua inversa $f^{-1}$. 
		Se necessario, restringere il dominio di $f$ per renderla biettiva.}
	
	\begin{parts}
		\part
		\exonly{
			\begin{flalign*}
				f:\R & \rightarrow \R \\
				x & \mapsto x^6 &
			\end{flalign*}
		}
		\solonly{
			\begin{flalign*}
				f:[0;\infty[ & \rightarrow [0;\infty[\\
				x & \mapsto x^6 &
			\end{flalign*}
		
		\begin{flalign*}
			f^{-1}:[0;\infty[ & \rightarrow [0;\infty[\\
			x & \mapsto \sqrt[6]{x} &
		\end{flalign*}
	
	oppure
	
				\begin{flalign*}
		f:]-\infty;0] & \rightarrow [0;\infty[\\
		x & \mapsto x^6 &
	\end{flalign*}
	
	\begin{flalign*}
		f^{-1}:[0;\infty[ & \rightarrow ]-\infty;0]\\
		x & \mapsto -\sqrt[6]{x} &
	\end{flalign*}
			
		%	$f^{-1}(x)=\sqrt[6]{x}$  biettiva su \\ $\DD_f=[0;\infty[$ e $\DD_{f^{-1}}=[0;\infty[$
		
	}
		
		\part
		
		\exonly{
		\begin{flalign*}
			f:\R & \rightarrow \R \\
			x & \mapsto (x-1)^2 +2  &
		\end{flalign*}	
			
			
			
		%	$f(x)=(x-1)^2 +2 $
		
	

}
		\solonly{
				\begin{flalign*}
			f:[1;\infty[ & \rightarrow [2;\infty[\\
			x & \mapsto x^6 &
		\end{flalign*}
		
		\begin{flalign*}
			f^{-1}:[2;\infty[ & \rightarrow [1;\infty[\\
			x & \mapsto \sqrt{x-1}+1 &
		\end{flalign*}
		
		oppure
		
		\begin{flalign*}
			f:]-\infty;1] & \rightarrow [2;\infty[\\
			x & \mapsto x^6 &
		\end{flalign*}
		
		\begin{flalign*}
			f^{-1}:[2;\infty[ & \rightarrow ]-\infty;1]\\
			x & \mapsto -\sqrt{x-1}+1 &
		\end{flalign*}
	
		
			
		%	$f^{-1}(x)=\sqrt{x-1}+1$  biettiva su \\  $\DD_f=[1;\infty[$  e $\DD_{f^{-1}}=[2;\infty[$
		
	


		}
		
		\part
		\exonly{
		\begin{flalign*}
			f:\R & \rightarrow \R \\
			x & \mapsto (x-3)^2 -1  &
		\end{flalign*}	
			
			%$f(x)=(x-2)^3-1$
		}
		\solonly{
						
		\begin{flalign*}
			f^{-1}:\R & \rightarrow \R \\
			x & \mapsto \sqrt[3]{x+1}+2 &
		\end{flalign*}	
		}
	\end{parts}
	
	
	
	\begin{profonly}	Src: Swok ex 1-9 impairs pg 202\end{profonly}
\end{questions}

\subsection{Rappresentazione grafica} 
\begin{questions}	
	
	\question 
	\exonly{Rappresentare graficamente le funzioni seguenti:}
	
	\begin{parts}
		\part 
		\exonly{$f(x)=\sqrt{x}$}
		
		\ifprintanswers 
		\begin{tikzpicture}[baseline={($(current bounding box.north)-(0,1.6ex)$)}]
		\begin{axis}[
		AxisDefaults,
		width=0.7\linewidth,
		xmin=-1,xmax=5,ymin=-1,ymax=5,samples=500,]
		\addplot[draw=red,smooth,unbounded coords=jump,domain=0:7]{sqrt(x)}; 
		\addplot [only marks, mark=o] (0,0) ;
		%\filldraw (axis cs:0,0) circle (1.5pt) ;
		\end{axis}
		\end{tikzpicture} 
		\fi
		
		
		\part 
		\exonly{$f(x)=\sqrt{x-2}$}
		
		\ifprintanswers 
		\begin{tikzpicture}[baseline={($(current bounding box.north)-(0,1.6ex)$)}]
		\begin{axis}[
		AxisDefaults,
		width=0.7\linewidth,
		xmin=-1,xmax=5,ymin=-1,ymax=5,samples=500,]
		\addplot[draw=red,smooth,unbounded coords=jump,domain=0:7]{sqrt(x-2)}; 
		\addplot [only marks, mark=o] (2,0) ;
		%\filldraw (axis cs:2,0) circle (1.5pt) ;
		\end{axis}
		\end{tikzpicture} 
		\fi
		
		\part 
		\exonly{$f(x)=\sqrt{x+3}$}
		
		\ifprintanswers 
		\begin{tikzpicture}[baseline={($(current bounding box.north)-(0,1.6ex)$)}]
		\begin{axis}[
		AxisDefaults,
		width=0.7\linewidth,
		xmin=-4,xmax=2,ymin=-1,ymax=5,samples=500,]
		\addplot[draw=red,smooth,unbounded coords=jump,domain=-3:5]{sqrt(x+3)}; 
		\addplot [only marks, mark=o] (-3,0) ;
		%\filldraw (axis cs:-3,0) circle (1.5pt) ;
		\end{axis}
		\end{tikzpicture} 
		\fi
		
		\part 
		\exonly{$f(x)=\sqrt{x+3}-1$}
		\ifprintanswers 
		\begin{tikzpicture}[baseline={($(current bounding box.north)-(0,1.6ex)$)}]
		\begin{axis}[
		AxisDefaults,
		width=0.7\linewidth,
		xmin=-4,xmax=2,ymin=-2,ymax=4,samples=500,]
		\addplot[draw=red,smooth,unbounded coords=jump,domain=-3:5]{sqrt(x+3)-1}; 
		\addplot [only marks, mark=o] (-3,-1) ;
		%\filldraw (axis cs:-3,-1) circle (1.5pt) ;
		\end{axis}
		\end{tikzpicture} 
		\fi
		
		\part
		\exonly{$f(x)=\sqrt[3]{x}-1$}
		\solonly{
		
		\begin{tikzpicture}[baseline={($(current bounding box.north)-(0,1.6ex)$)}]
	\begin{axis}[AxisDefaults,
	width=0.7\linewidth,
	ytick distance={1},xmin=-4,xmax=4,ymin=-5,ymax=4,samples=500,]
	\addplot[draw=red,smooth,unbounded coords=jump,restrict y to domain=-6:5]{cbrt(x)-1} ; 
	\end{axis}
	
	\end{tikzpicture}

 }

				\part
		\exonly{$f(x)=\sqrt[3]{x+2}$}
		\solonly{
		\begin{tikzpicture}[baseline={($(current bounding box.north)-(0,1.6ex)$)}]
\begin{axis}[AxisDefaults,
width=0.7\linewidth,
ytick distance={1},xmin=-4,xmax=4,ymin=-5,ymax=4,samples=500,]
\addplot[draw=red,smooth,unbounded coords=jump,restrict y to domain=-6:5]{cbrt(x+2)} ; 
\end{axis}

\end{tikzpicture}
	 }
 
 				\part
 \exonly{$f(x)=-\sqrt[3]{x+2}$}
 \solonly{
 	\begin{tikzpicture}[baseline={($(current bounding box.north)-(0,1.6ex)$)}]
 	\begin{axis}[AxisDefaults,
 	width=0.7\linewidth,
 	ytick distance={1},xmin=-4,xmax=4,ymin=-5,ymax=4,samples=500,]
 	\addplot[draw=red,smooth,unbounded coords=jump,restrict y to domain=-6:5]{-cbrt(x+2)} ; 
 	\end{axis}
 	
 	\end{tikzpicture}
 }

 				\part
\exonly{$f(x)=\sqrt[3]{-x-2}$}
\solonly{
	\begin{tikzpicture}[baseline={($(current bounding box.north)-(0,1.6ex)$)}]
	\begin{axis}[AxisDefaults,
	width=0.7\linewidth,
	ytick distance={1},xmin=-4,xmax=4,ymin=-5,ymax=4,samples=500,]
	\addplot[draw=red,smooth,unbounded coords=jump,restrict y to domain=-6:5]{cbrt(-x-2)} ; 
	\end{axis}
	
	\end{tikzpicture}
}
		
	\end{parts}
	
	
	\begin{profonly}Dessiner des racines, avec décalages\end{profonly}
	
	
	
	
	
	%\exnewpage
	\question
	\exonly{Determinare la funzioni rappresentata:}
	
	
		\ifprintanswers  \else
		\begin{tikzpicture}[baseline={($(current bounding box.north)-(0,1.6ex)$)}]
		\begin{axis}[
		AxisDefaults,
		width=0.7\linewidth,
		ytick distance={1},ymin=-2]
		\addplot[draw=red,smooth,unbounded coords=jump,restrict y to domain=-4:5,domain=-10:10]{0.5*(x+3)^2-1} node[above left,pos=0.8] {$f(x)$}; 
		\end{axis}
		\end{tikzpicture}
		\fi
		
		\solonly{
			$f(x)=\dfrac{1}{2}(x+3)^2-1$}
	\exnewpage	
	\question
\exonly{Determinare la funzioni rappresentata:}

		\ifprintanswers 	\else
		\begin{tikzpicture}[baseline={($(current bounding box.north)-(0,1.6ex)$)}]
		\begin{axis}[
		AxisDefaults,
		width=0.7\linewidth,
		ytick distance={1},xmin=-2,xmax=2]
		\addplot[draw=red,smooth,unbounded coords=jump,restrict y to domain=-3:3]{2*(x+1)^3} node[below right,pos=0.8] {$g(x)$}; 
		\end{axis}
		\end{tikzpicture} 
		\fi
		\solonly{$g(x)=2(x+1)^3$}

	\question
\exonly{Determinare la funzioni rappresentata:}

		\ifprintanswers     \else 
		\begin{tikzpicture}[baseline={($(current bounding box.north)-(0,1.6ex)$)}]
		
		\begin{axis}[
		AxisDefaults,
		width=0.7\linewidth,
		ytick distance={1},xmin=0,xmax=5,ymin=-2]
		\addplot[draw=red,smooth,unbounded coords=jump,restrict y to domain=-2:3]{2*sqrt(x-2)-1} node[above left,pos=0.8] {$h(x)$}; 
		\addplot [only marks, mark=o] (2,-1) ;
		
		\end{axis}
		\end{tikzpicture}
		\fi
		\solonly{$h(x)=2\sqrt{x-2}-1$}
		
		\exnewpage
		\question
	\exonly{Determinare la funzioni rappresentata:}
	
		\ifprintanswers     \else     
		\begin{tikzpicture}[baseline={($(current bounding box.north)-(0,1.6ex)$)}]
		\begin{axis}[AxisDefaults,
		width=0.7\linewidth,
		ytick distance={1},xmin=-3,xmax=3,ymin=-2,samples=500,]
		\addplot[draw=red,smooth,unbounded coords=jump,restrict y to domain=-2:2]{cbrt(x+1)} node[below right,pos=0.7] {$i(x)$}; 
		\end{axis}
		
		\end{tikzpicture}
		\fi
		\solonly{$i(x)=\sqrt[3]{x+1}$}

	\question
\exonly{Determinare la funzioni rappresentata:}

		\begin{tikzpicture}[baseline={($(current bounding box.north)-(0,1.6ex)$)}]
		\exonly{
			\begin{axis}[AxisDefaults,
			width=0.7\linewidth,
			ytick distance={1},
			xmin=-1,xmax=2,ymin=-2,ymax=2,samples=500,grid=both,minor x tick num=1]
			\addplot[draw=red,smooth,unbounded coords=jump,restrict y to domain=-2:2]{cbrt(2*x-1)} node[below right,pos=0.6] {$l(x)$}; 
			\end{axis}}
		\end{tikzpicture}
		\solonly{$l(x)=\sqrt[3]{2x-1}$}
		
			\question
		\exonly{Determinare la funzioni rappresentata:}
			
		\begin{tikzpicture}[baseline={($(current bounding box.north)-(0,1.6ex)$)}]
		\exonly{
			\begin{axis}[
			AxisDefaults,
			width=0.7\linewidth,
			ytick distance={1},xmin=0,xmax=7,ymin=-2,ymax=3,samples=500,]
			\addplot[draw=red,smooth,unbounded coords=jump,domain=0:7]{-sqrt(x-2)+1} node[below right,pos=0.7] {$m(x)$}; 
			\addplot [only marks, mark=o] (2,1) ;
			%\filldraw (axis cs:2,1) circle (1.5pt) ;
			\end{axis}}
		\end{tikzpicture}
		\solonly{$m(x)=-\sqrt{x-2}+1$}

	
	
\end{questions}

\subsection{Risoluzioni equazioni} 
\begin{questions}
	\question 	
	
	\exonly{
		Determinare graficamente e in seguito calcolare i punti di intersezione tra le funzioni $f$ e $g$.
	}
	
	\begin{parts}
		\part \exonly{$f(x)=\sqrt{3x+4} \qquad g(x)=x-2$} \solonly{$I\left(7;5\right)$}	
		\part \exonly{$f(x)=\sqrt{x-32} \qquad g(x)=16-\sqrt{x}$} \solonly{$I\left(81;7\right)$}
	\end{parts}	
	\begin{profonly}	pg 26 onenote , dessiner raciens, et intersections\end{profonly}
	
	
	\question
	Risolvere le equazioni seguenti:
	
	
	\begin{parts}
		
		\part
		\exonly{$\sqrt{x^2+6}=\sqrt{5x}$}
		\solonly{$S=\{2;3\}$}
		
		\part
		\exonly{$\sqrt{x^2-2}=\sqrt{2x^2-4x+1}$}
		\solonly{$S=\{3\}$}
		
		\part
		\exonly{$2+\sqrt{3x+4}=x$}
		\solonly{$S=\{7\}$}
		
		
	\end{parts}
	
\end{questions}