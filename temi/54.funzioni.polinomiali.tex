\section{Funzioni  polinomiali}
\subsection{Fattorizzazione}

\begin{questions}

	\question
	\exonly{
		Scomporre in fattori e determinare le soluzioni delle seguenti equazioni:
	}
	\begin{parts}

		\part
		\exonly{$x^3-x^2-10x-8=0$}
		\solonly{$(x + 1) (x + 2) (x - 4)=0$ \\ $S=\left\lbrace -2,-1,4\right\rbrace $}

		\part
		\exonly{$x^3 + x^2 -14x -24 = 0$ }
		\solonly{$(x - 4) (x + 2) (x + 3) = 0=0$ \\ $S=\left\lbrace -3,-2,4\right\rbrace $}

		\part
		\exonly{$2 x^3 - 3 x^2 - 17 x + 30 = 0$ }
		\solonly{$(x - 2) (x + 3) (2 x - 5) = 0$ \\$S=\left\lbrace -3,2,\dfrac{5}{2}\right\rbrace $}

		\part
		\exonly{$12x^3+8x^2-3x-2=0$ }
		\solonly{ $(3x+2)(2x-1)(2x+1)=0 $ \\ $S=\left\lbrace -\dfrac{2}{3},-\dfrac{1}{2},\dfrac{1}{2}\right\rbrace $ }

	\end{parts}
	\begin{profonly}	Src: Swok ex 15-18 impairs pg 289\end{profonly}

	%	\question
	%	\exonly{Favre, ex. 27, pg.219}\sol{Sol. Favre}	
\end{questions}

\subsection{Studio della funzione}

\begin{questions}
	\question
	\exonly{
		Realizzare lo studio della funzione (zeri, massimi, minimi, etc.) e rappresentarla graficamente:

	}
	\begin{parts}

		\part
		\exonly{$f_1(x)=x^3-x^2-10x-8$}
		\ifprintanswers
			\documentclass[preview,finale]{standalone}
\usepackage{FulvioCustomIta}

\begin{document}
\begin{tikzpicture}[baseline={($(current bounding box.north)-(0,1.6ex)$)}]
\tikzset{
	every pin/.style={fill=yellow!50!white,rectangle,rounded corners=3pt,font=\tiny},
	small dot/.style={fill=black,circle,scale=0.3}
}
\begin{axis}[
AxisDefaults, 
width=10cm,
ytick distance={10}]
\addplot[draw=red,smooth,unbounded coords=jump,restrict y to domain=-25:20]{x^3-x^2-10*x-8}; 
\addplot[mark=*] coordinates {(-1.52,1.38)} node[small dot,pin={[pin distance=0.5cm]90:{$(-1.52,1.38)$}}]{} ;
\addplot[mark=*] coordinates {(2.19,-24.2)} node[small dot,pin={[pin distance=1cm]90:{$(2.19,-24.2)$}}]{} ;
\end{axis}
\end{tikzpicture} 
\end{document}
			%\begin{tikzpicture}[baseline={($(current bounding box.north)-(0,1.6ex)$)}]
			%  \tikzset{
			%	every pin/.style={fill=yellow!50!white,rectangle,rounded corners=3pt,font=\tiny},
			%	small dot/.style={fill=black,circle,scale=0.3}
			%}
			%\begin{axis}[
			%AxisDefaults, 
			%width=10cm,
			%ytick distance={10}]
			%\addplot[draw=red,smooth,unbounded coords=jump,restrict y to domain=-25:20]{x^3-x^2-10*x-8}; 
			%\addplot[mark=*] coordinates {(-1.52,1.38)} node[small dot,pin={[pin distance=0.5cm]90:{$(-1.52,1.38)$}}]{} ;
			%\addplot[mark=*] coordinates {(2.19,-24.2)} node[small dot,pin={[pin distance=1cm]90:{$(2.19,-24.2)$}}]{} ;
			%\end{axis}
			%\end{tikzpicture} 
		\fi


		\part
		\exonly{$f_2(x)=-x^3 - x^2 +14x +24$ }
		\ifprintanswers
			%\documentclass[preview,finale]{standalone}
%\usepackage{FulvioCustomIta}

%\begin{document}
\begin{tikzpicture}[baseline={($(current bounding box.north)-(0,1.6ex)$)}]
\tikzset{
	every pin/.style={fill=yellow!50!white,rectangle,rounded corners=3pt,font=\tiny},
	small dot/.style={fill=black,circle,scale=0.3}
}
\begin{axis}[
AxisDefaults, 
width=10cm,
ytick distance={10}]
\addplot[draw=red,smooth,unbounded coords=jump,restrict y to domain=-45:5]{x^3+x^2-14*x-24}; 
\addplot[mark=*] coordinates {(-2.5,1.6)} node[small dot,pin={[pin distance=2cm]-90:{$(-2.5,1.6)$}}]{} ;
\addplot[mark=*] coordinates {(1.85,-40.1)} node[small dot,pin={[pin distance=1cm]90:{$(1.85,-40.1)$}}]{} ;
\end{axis}
\end{tikzpicture} 
%\end{document}
			%\begin{tikzpicture}[baseline={($(current bounding box.north)-(0,1.6ex)$)}]
			%\tikzset{
			%	every pin/.style={fill=yellow!50!white,rectangle,rounded corners=3pt,font=\tiny},
			%	small dot/.style={fill=black,circle,scale=0.3}
			%}
			%\begin{axis}[
			%AxisDefaults, 
			%width=10cm,
			%ytick distance={10}]
			%\addplot[draw=red,smooth,unbounded coords=jump,restrict y to domain=-45:5]{x^3+x^2-14*x-24}; 
			%\addplot[mark=*] coordinates {(-2.5,1.6)} node[small dot,pin={[pin distance=2cm]-90:{$(-2.5,1.6)$}}]{} ;
			%\addplot[mark=*] coordinates {(1.85,-40.1)} node[small dot,pin={[pin distance=1cm]90:{$(1.85,-40.1)$}}]{} ;
			%\end{axis}
			%\end{tikzpicture} 
		\fi


		\part
		\exonly{$f_3(x)=2 x^3 - 3 x^2 - 17 x + 30 $ }
		\ifprintanswers
			\documentclass[preview,finale]{standalone}
\usepackage{FulvioCustomIta}

\begin{document}
\begin{tikzpicture}[baseline={($(current bounding box.north)-(0,1.6ex)$)}]
\tikzset{
	every pin/.style={fill=yellow!50!white,rectangle,rounded corners=3pt,font=\tiny},
	small dot/.style={fill=black,circle,scale=0.3}
}
\begin{axis}[
AxisDefaults, 
width=10cm,
ytick distance={10}]
\addplot[draw=red,smooth,unbounded coords=jump,restrict y to domain=-5:45]{2*x^3-3*x^2-17*x+30}; 
\addplot[mark=*] coordinates {(-1.26,42.66)} node[small dot,pin={[pin distance=2cm]-90:{$(-1.26,42.66)$}}]{} ;
\addplot[mark=*] coordinates {(2.26,-0.66)} node[small dot,pin={[pin distance=1cm]90:{$(2.26,-0.66)$}}]{} ;
\end{axis}
\end{tikzpicture} 
\end{document}
			%\begin{tikzpicture}[baseline={($(current bounding box.north)-(0,1.6ex)$)}]
			%\tikzset{
			%	every pin/.style={fill=yellow!50!white,rectangle,rounded corners=3pt,font=\tiny},
			%	small dot/.style={fill=black,circle,scale=0.3}
			%}
			%\begin{axis}[
			%AxisDefaults, 
			%width=10cm,
			%ytick distance={10}]
			%\addplot[draw=red,smooth,unbounded coords=jump,restrict y to domain=-5:45]{2*x^3-3*x^2-17*x+30}; 
			%\addplot[mark=*] coordinates {(-1.26,42.66)} node[small dot,pin={[pin distance=2cm]-90:{$(-1.26,42.66)$}}]{} ;
			%\addplot[mark=*] coordinates {(2.26,-0.66)} node[small dot,pin={[pin distance=1cm]90:{$(2.26,-0.66)$}}]{} ;
			%\end{axis}
			%\end{tikzpicture} 
		\fi



		\part
		\exonly{$f_4(x)=12x^3+8x^2-3x-2$ }
		\ifprintanswers
			\begin{tikzpicture}[baseline={($(current bounding box.north)-(0,1.6ex)$)}]
				\tikzset{
					every pin/.style={fill=yellow!50!white,rectangle,rounded corners=3pt,font=\tiny},
					small dot/.style={fill=black,circle,scale=0.3}
				}
				\begin{axis}[
						AxisDefaults,
						width=10cm,
						ytick distance={1}]
					\addplot[draw=red,smooth,unbounded coords=jump,restrict y to domain=-10:10]{12*x^3+8*x^2-3*x-2};
					\addplot[mark=*] coordinates {(-0.59,0.09)} node[small dot,pin={[pin distance=2cm]-90:{$(-0.59,0.09)$}}]{} ;
					\addplot[mark=*] coordinates {(0.14,-2.23)} node[small dot,pin={[pin distance=1cm]90:{$(0.14,-2.23)$}}]{} ;
				\end{axis}
			\end{tikzpicture}
		\fi










	\end{parts}

	\solnewpage
	\question
	\exonly{
		Realizzare lo studio delle funzioni seguenti (zeri e molteplicità, rappresentazione grafica).}

	\begin{parts}
		\part
		\exonly{$f(x)=x^4-4x^2$}
		\solonly{Zero: $0$ molteplicità $2$, $2$ molt. $1$ et $-2$ molt. $1$}

		\ifprintanswers
			\begin{tikzpicture}[baseline={($(current bounding box.north)-(0,1.6ex)$)}]
				\begin{axis}[AxisDefaults,
						width=12cm,
						ytick distance={1},]
					\addplot[draw=red,smooth,unbounded coords=jump,restrict y to domain=-10:10]{x^4-4*x^2};
				\end{axis}
			\end{tikzpicture}
		\fi

		\begin{profonly}	Src: Swok ex 15 pg 260\end{profonly}

		\part
		\exonly{$f(x)=-x^3+3x^2+10x$}
		\solonly{Zero: $0$ molteplicità $1$, $5$ molt. $1$ et $-2$ molt. $1$}

		\ifprintanswers
			\begin{tikzpicture}[baseline={($(current bounding box.north)-(0,1.6ex)$)}]
				\begin{axis}[AxisDefaults,
						width=12cm,
						ytick distance={10},
					]
					\addplot[draw=red,smooth,unbounded coords=jump,
						restrict y to domain=-20:35,
						domain=-3:6
					]{-x^3+3*x^2+10*x};
				\end{axis}
			\end{tikzpicture}
		\fi

		\begin{profonly}	Src: Swok ex 17 pg 260\end{profonly}
		\solnewpage
		\part
		\exonly{$f(x)=\dfrac{1}{6}(x+2)(x-3)(x-4)$}
		\solonly{Zero: $-2$ molteplicità $1$, $3$ molt. $1$ et $4$ molt. $1$}

		\ifprintanswers
			\begin{tikzpicture}[baseline={($(current bounding box.north)-(0,1.6ex)$)}]
				\begin{axis}[AxisDefaults,
						width=12cm,
						ytick distance={1},
					]
					\addplot[draw=red,smooth,unbounded coords=jump,
						restrict y to domain=-5:5,
						domain=-3:6
					]{(x+2)*(x-3)*(x-4)/6};
				\end{axis}
			\end{tikzpicture}
		\fi

		\begin{profonly}	Src: Swok ex 19 pg 260\end{profonly}

		\part
		\exonly{$f(x)=-x^3-2x^2+4x+8$}
		\solonly{Zero: $2$ molteplicità $1$, $-2$ molt. $2$}

		\ifprintanswers
			\begin{tikzpicture}[baseline={($(current bounding box.north)-(0,1.6ex)$)}]
				\begin{axis}[AxisDefaults,
						width=12cm,
						ytick distance={2},
					]
					\addplot[draw=red,smooth,unbounded coords=jump,
						restrict y to domain=-10:10,
						domain=-3:6
					]{-x^3-2*x^2+4*x+8};
				\end{axis}
			\end{tikzpicture}
		\fi

		\begin{profonly}	Src: Swok ex 21 pg 260\end{profonly}




		\solnewpage
		\part
		\exonly{$f(x)=x^4-6x^2+8$}
		\solonly{Zero: $2$ molteplicità $1$, $-2$ molt. $1$, $\sqrt{2}$ molt. $1$ e $-\sqrt{2}$ molt. $1$.}

		\ifprintanswers
			\begin{tikzpicture}[baseline={($(current bounding box.north)-(0,1.6ex)$)}]
				\begin{axis}[AxisDefaults,
						width=12cm,
						ytick distance={2},
						ymin=-3,
					]
					\addplot[draw=red,smooth,unbounded coords=jump,
						restrict y to domain=-10:10,
						%domain=-3:6
					]{x^4-6*x^2+8};
				\end{axis}
			\end{tikzpicture}
		\fi

		\begin{profonly}	Src: Swok ex 23 pg 260\end{profonly}


		\part
		\exonly{$f(x)=x^2(x+1)^2(x-2)(x-4)$}
		\solonly{Zero: $0$ molteplicità $2$, $-1$ molt. $2$, $2$ molt. $1$ e $4$ molt. $1$.}

		\ifprintanswers
			\begin{tikzpicture}[baseline={($(current bounding box.north)-(0,1.6ex)$)}]
				\begin{axis}[AxisDefaults,
						width=12cm,
						ytick distance={5},
						ymin=-10,
						ymax=20,
						xmax=5,
					]
					\addplot[draw=red,smooth,
						restrict y to domain=-20:40,
						domain=-5:3.2,
					]{x^2*(x+1)^2*(x-2)*(x-4)};
					\addplot[draw=red,smooth,
						restrict y to domain=-20:40,
						domain=3.5:5,
					]{x^2*(x+1)^2*(x-2)*(x-4)};
				\end{axis}
			\end{tikzpicture}
		\fi

		\begin{profonly}	Src: Swok ex 25 pg 260\end{profonly}


		\solnewpage
		\part
		\exonly{$f(x)=12x^3-4x^2-3x+1$}
		\solonly{Zero: $\frac{1}{3}$ molteplicità $1$, $\frac{1}{2}$ molt. $1$ e $-\frac{1}{2}$ molt. $1$}

		\ifprintanswers
			\begin{tikzpicture}[baseline={($(current bounding box.north)-(0,1.6ex)$)}]
				\begin{axis}[AxisDefaults,
						width=12cm,
						ytick distance={0.5},ytick distance={0.5},
						xmin=-1.2,xmax=1.2,
						ymin=-0.5,ymax=1.5,
					]
					\addplot[draw=red,smooth,unbounded coords=jump,
						restrict y to domain=-10:10,
						domain=-3:6
					]{12*x^3-4*x^2-3*x+1};
				\end{axis}
			\end{tikzpicture}
		\fi
	\end{parts}

\end{questions}

\exnewpage
\subsection{Rappresentazione grafica}

\begin{questions}

	\question
	\exonly{
		Determinare la legge di assegnazione della funzione polinomiale di 3° grado rappresentata qui sotto.}
	\solonly{$f(x)=\dfrac{7}{9}(x+1)(x-\dfrac{3}{2})(x-3)$}

	\begin{tikzpicture}[baseline={($(current bounding box.north)-(0,1.6ex)$)}]
		\exonly{
			\begin{axis}[AxisDefaults,
					width=0.7\linewidth,
					xlabel=$t$, xlabel style={at=(current axis.right of origin), anchor=west},
					ylabel=$N(t)$, ylabel style={at=(current axis.above origin), anchor=south},
					ytick distance={1},
					grid=both,
					minor tick num = 1,
					ymin=-5,ymax=5,xmin=-2,xmax=5,
				]
				\addplot[draw=red,thick,smooth,unbounded coords=jump,
					restrict y to domain=-7:6,
					domain=-5:5
				]{7*(x+1)*(x-3/2)*(x-3)/9};
				%\filldraw (axis cs:0,3.5) circle (1.5pt) node[above right] {\tiny $(0;\num{3.5})$};
			\end{axis}

		}
	\end{tikzpicture}



	\begin{profonly}	Src: Swok ex 11 pg 280\end{profonly}

	\question
	\exonly{
		Determinare la legge di assegnazione della funzione polinomiale di 4° rappresentata qui sotto.}
	\solonly{$f(x)=\dfrac{1}{12}x(x-1)(x-3)(x-5)$}

	\begin{tikzpicture}[baseline={($(current bounding box.north)-(0,1.6ex)$)}]
		\exonly{
			\begin{axis}[AxisDefaults,
					width=0.7\linewidth,
					ytick distance={1},
					grid=both,
					minor tick num = 1,
					ymin=-2,ymax=6,xmin=-3,xmax=6,
				]
				\addplot[draw=red,thick,smooth,unbounded coords=jump,
					restrict y to domain=-3:8,
					domain=-3:7
				]{x*(x-1)*(x-3)*(x-5)/12};
				\addplot [only marks, mark=o] (1,-4) node[left] {\scriptsize $(-1;\num{4})$} ;
				%\filldraw (axis cs:-1,4) circle (1.5pt) node[left] {\scriptsize $(-1;\num{4})$};
			\end{axis}

		}
	\end{tikzpicture}



	\begin{profonly}	Src: Swok ex 12 pg 280\end{profonly}
	\exnewpage

	\question
	\exonly{
		Determinare la legge di assegnazione della funzione polinomiale rappresentata qui sotto. In seguito determinare massimi e minimi locali alla calcolatrice.}


	\begin{parts}
		\part

		\solonly{$f(x)=-(x-1)^2(x-3)$
			\newline
			Massimo locale $(2.33;119)$ , Minimo locale $(1;0)$

		}

		\begin{tikzpicture}[baseline={($(current bounding box.north)-(0,1.6ex)$)}]
			\exonly{
				\begin{axis}[AxisDefaults,
						width=0.7\linewidth,
						ytick distance={1},
						grid=both,
						minor tick num = 1,
						ymin=-2,ymax=6,xmin=-3,xmax=6,
					]
					\addplot[draw=red,thick,smooth,unbounded coords=jump,
						restrict y to domain=-3:8,
						domain=-3:7
					]{-(x-3)*(x-1)^2};

				\end{axis}

			}
		\end{tikzpicture}



		\begin{profonly}	Src: Swok ex 13 pg 280\end{profonly}

		%\question 
		%\exonly{Déterminer la fonctions polynomiale dont le graphique est donne sur la figure}
		\part
		\solonly{$f(x)=(x-2)^2(x-4)$
			\newline
			Massimo locale $(2;0)$ , Minimo locale $(3.33;-1.19)$
		}

		\begin{tikzpicture}[baseline={($(current bounding box.north)-(0,1.6ex)$)}]
			\exonly{
				\begin{axis}[AxisDefaults,
						width=0.7\linewidth,
						ytick distance={1},
						grid=both,
						minor tick num = 1,
						ymin=-5,ymax=5,xmin=-2,xmax=6,
					]
					\addplot[draw=red,thick,smooth,unbounded coords=jump,
						restrict y to domain=-6:6,
						domain=-3:7
					]{(x-4)*(x-2)^2};
					\addplot [only marks, mark=o] (1,-3) node[right] {\scriptsize $(1;-3)$} ;
					%\filldraw (axis cs:1,-3) circle (1.5pt) node[right] {\scriptsize $(1;-3)$};
				\end{axis}

			}
		\end{tikzpicture}

		\part
		\solonly{$f(x)=\dfrac{1}{512}(x-2)^2(x+2)^2(x-8)^2$
			\newline Massimi locali $(-0.24;2.06)$, $(5.57;8.42)$ \newline
			Minimi globali $(-2;0)$, $(2;0)$ e $(8;0)$
		}

		\begin{tikzpicture}[baseline={($(current bounding box.north)-(0,1.6ex)$)}]
			\exonly{
				\begin{axis}[AxisDefaults,
						width=0.7\linewidth,
						ytick distance={1},
						grid=both,
						minor tick num = 1,
						ymin=-2,ymax=11,xmin=-4,xmax=10,
					]
					\addplot[draw=red,thick,smooth,unbounded coords=jump,
						restrict y to domain=-2:15,
						domain=-4:11
					]{1/512*(x-2)^2*(x+2)^2*(x-8)^2};
					\addplot [only marks, mark=o] (0,2) node[right] {\scriptsize $(0;2)$};
					%\filldraw (axis cs:0,2) circle (1.5pt) node[right] {\scriptsize $(0;2)$};
				\end{axis}

			}
		\end{tikzpicture}

		\begin{profonly}	Src: Swok ex 14 pg 280\end{profonly}
	\end{parts}


	%\question
	%\exonly{Favre, ex. 24, pg.218}	
	%\solonly{Sol. Favre}
	\solnewpage


	\question
	\exonly{Risolvere graficamente la seguente equazione: $x^3 - 0.6x-3= 0$ .

		% Si cerchi dapprima di stimare una	soluzione determinando l’intersezione tra i grafici di $f (x) = x^3$  e $g(x) = 0.6x + 3$.

		In seguito confrontare il risultato con la soluzione numerica data dalla calcolatrice.}
	%TODO: controllare immagine che da errore in compilazione su OSX !!!!!

	\begin{tikzpicture}[baseline={($(current bounding box.north)-(0,1.6ex)$)}]
		\begin{axis}[AxisDefaults,
			major grid style={ thick,black!50},
				axis line style={very thick, black},
				x=1cm,y=1cm,
				width=0.7\linewidth,
				ytick distance={1},
				grid=both,
				minor tick num = 9,
				ymin=-5,ymax=5,xmin=-5,xmax=5,
				domain=-5:5
			]
			\addplot[draw=blue,thick,smooth,unbounded coords=jump,
				restrict y to domain=-6:6,name path=cubic ]{x^3} node[pos=0.8, left] {$x^3$};
			\ifprintanswers	
			\addplot[draw=red,thick,smooth,unbounded coords=jump,name path=line
			]{0.6*x+3};
			\path [name intersections={of=cubic and line, by=a}];
			\path (a) node[pin={[pin distance=1cm]-80:{$\approx(1.6;3.9)$}}]{} ;
			\fi
		\end{axis}

	\end{tikzpicture}
	\solonly{
		% \begin{tikzpicture}[baseline={($(current bounding box.north)-(0,1.6ex)$)}]
		% 	\begin{axis}[AxisDefaults,
		% 			width=0.7\linewidth,
		% 			ytick distance={1},
		% 			grid=both,
		% 			minor tick num = 1,
		% 			ymin=-3,ymax=5,xmin=-2,xmax=3,
		% 			domain=-3:3
		% 		]
		% 		\addplot[draw=blue,thick,smooth,unbounded coords=jump,
		% 			restrict y to domain=-6:6,name path=cubic ]{(x^3};
		% 		\addplot[draw=red,thick,smooth,unbounded coords=jump,
		% 			restrict y to domain=-6:6,name path=line
		% 		]{0.6*x+3};
		% 		\path [name intersections={of=cubic and line, by=a}];
		% 		\path (a) node[pin={[pin distance=1cm]-80:{$\approx(1.6;4)$}}]{} ;
		% 	\end{axis}

		% \end{tikzpicture}
		Calcolatrice: $\es{(\num{1.5805},\num{3.9483})}$
	}

	\exnewpage
	\question
	\exonly{Dato il grafico di $f$ risolvere graficamente l'equazione $f(x)=3$.

		%Last revision: 14.09.2020
%CHANGELOG: assi più visibili, very thick e black
% major grid più visibile black!50 e thick


\begin{tikzpicture}[baseline={($(current bounding box.north)-(0,1.6ex)$)}]
\begin{axis}[AxisDefaults,
major grid style={ thick,black!50},
axis line style={very thick, black},
x=1.5cm,y=1.5cm,
%width=\linewidth,
ytick distance={1},
grid=both,
minor tick num = 9,
ymin=-5,ymax=7,xmin=-4,xmax=5,
domain=-4:5
]
\addplot[draw=blue,thick,smooth,unbounded coords=jump,
restrict y to domain=-6:8]{(x+3)*(x+1)*(x-2)*(x-1)*(x-4)/12.5}; 
\end{axis}
\end{tikzpicture}
	}

	\exnewpage
	\question
	\exonly{Dato il grafico di $f$ risolvere graficamente l'equazione $ f(x) + \frac{3}{4}x=3$.

		%Last revision: 14.09.2020
%CHANGELOG: assi più visibili, very thick e black
% major grid più visibile black!50 e thick


\begin{tikzpicture}[baseline={($(current bounding box.north)-(0,1.6ex)$)}]
\begin{axis}[AxisDefaults,
major grid style={ thick,black!50},
axis line style={very thick, black},
x=1.5cm,y=1.5cm,
%width=\linewidth,
ytick distance={1},
grid=both,
minor tick num = 9,
ymin=-5,ymax=7,xmin=-4,xmax=5,
domain=-4:5
]
\addplot[draw=blue,thick,smooth,unbounded coords=jump,
restrict y to domain=-6:8]{(x+3)*(x+1)*(x-2)*(x-1)*(x-4)/12.5}; 
\end{axis}
\end{tikzpicture}
	}
	\exnewpage





	\question
	Per ogni grafico sono proposte quattro funzioni da $\R \longrightarrow \R$. Associare la legge di assegnazione con il grafico.


	\begin{parts}
		\part
		\begin{tikzpicture}[baseline={($(current bounding box.north)-(0,1.6ex)$)}]
			\begin{axis}[AxisDefaults,
					width=8cm,
					ymin=-3,ymax=3,xmin=-1,xmax=3,
					grid=none,
					xtick = \empty,
					ytick = \empty,
				]
				\addplot[draw=red,smooth,unbounded coords=jump,
					restrict y to domain=-6:6,
					domain=-1:5
				]{3*x*(x-1)*(x-2)};
			\end{axis}
		\end{tikzpicture}
		\begin{minipage}[t][][c]{0.5\linewidth}
			$f_1(x)=x^2(x-1)^2(x-2)$  \medskip

			$f_2(x)=3x(x-1)(x-2)$ \medskip

			$f_3(x)=x^3(x-1)(x-2)$\medskip

			$f_4(x)=-x(x-1)(x-2)$
		\end{minipage}

		\part
		\begin{tikzpicture}[baseline={($(current bounding box.north)-(0,1.6ex)$)}]
			\begin{axis}[AxisDefaults,
					width=8cm,
					ymin=-3,ymax=3,xmin=-1,xmax=3,
					grid=none,
					xtick = \empty,
					ytick = \empty,
				]
				\addplot[draw=red,smooth,unbounded coords=jump,
					restrict y to domain=-6:6,
					domain=-1:5
				]{5*x*(x-1)^2*(x-2)};
			\end{axis}
		\end{tikzpicture}
		\begin{minipage}[t][][c]{0.5\linewidth}
			$f_1(x)=2x^3(x-1)(x-2)^2$  \medskip

			$f_2(x)=x^2(x-1)(x-2)$ \medskip

			$f_3(x)=5x(x-1)^2(x-2)$\medskip

			$f_4(x)=-2x(x-1)^2(2-x)$
		\end{minipage}


	\end{parts}

\end{questions}
\exnewpage
\subsection{Applicazioni varie}

\begin{questions}

	\question
	\exonly{Determinare gli zeri nonché il massimo e il minimo locale della seguente funzione:

		$f(x)=\num{0.046}x^3-\num{1.57}x^2 +\num{14.3}x-26$


	}





	\solonly{
		Zeri:  $x_1=\num{2.411570}$, $x_2=\num{11.718864}$ e $x_3=\num{20}$ \\
		Massimo locale: $P_{\text{max}}=(\num{6.2966};\num{13.27887}) $\\
		Minimo locale: $P_{\text{min}}=(\num{16.4570};\num{-10.846}) $\\
	}

	\question
	\exonly{
		Un branco di 100 cervi viene introdotto su di una piccola isola.
		%	Un troupeau de 100 cerfs est introduit sur une petite île. 
		Inizialmente il branco cresce rapidamente, ma in seguito le risorse cominciano a scarseggiare e la popolazione diminuisce.
		%Tout d'abord le troupeau grandit rapidement, mais par la suite les ressources en nourriture baissent et la population diminue.
		Supponiamo che il numero di cervi $N(t)$ dopo $t$ anni dalla loro introduzione sull'isola sia dato da

		$N(t) =-t^4 + 21t^2 + 100$, dove $t > 0$.}

	\begin{parts}
		\part
		\exonly{
			Determinare il valore $t$ per il quale $N(t) > 0$ e rappresentare graficamente $N(t)$.
			%Déterminer la valeur de $t$ pour laquelle $N(t) > 0$ et représenter le graphique de $N(t)$.
		}
		\solonly{$0<t<5$}

		\ifprintanswers
			\begin{tikzpicture}[baseline={($(current bounding box.north)-(0,1.6ex)$)}]
				\begin{axis}[AxisDefaults,
						width=\linewidth,
						ytick distance={50},
						ymin=0,
					]
					\addplot[draw=red,smooth,unbounded coords=jump,
						restrict y to domain=0:250,
						domain=0:5
					]{-x^4+21*x^2+100};
				\end{axis}
			\end{tikzpicture}
		\fi
		\part
		\exonly{
			Secondo questo modello la popolazione si estinguerà? Se si, quando?
			%	La population de cerfs va-t-elle s’éteindre? Si oui, quand ?
		}
		\solonly{
			Si, dopo 5 anni.
		}

	\end{parts}




	\question

	\exonly{La tabella qui sotto indica la temperatura $\theta$ in \si{\celsius}

		misurate nell'arco di una giornata (24 ore).
		L'andamento della temperatura in questo intervallo può essere descritto da una funzione polinomiale (di $4\degree$) $f(t)=\theta$.

		\begin{tabular}{|c|c|c|c|c|c|}
			\hline
			$t[\si{\hour}]$          & 0 & 5 & 12 & 19 & 24 \\
			\hline
			$\theta [\si{\celsius}]$ & 0 & 0 & 10 & 0  & 0  \\
			\hline
		\end{tabular}  }



	\begin{parts}
		\part
		\exonly{Determinare la legge di assegnazione di $f(t)$ e tracciarne l'andamento. }
		\solonly{ $f(t)=\dfrac{5}{3528}t(t-5)(t-19)(t-24)$ }

		\part
		\exonly{Determinare la temperatura $\theta $ se $t=9$ }
		\solonly{$ \theta \approx 7.65 \si{\celsius}$ }
	\end{parts}


	\question
	\exonly{La relazione tra la quantità $q$ e il prezzo $p$ ci permettono di descrivere l'offerta e la domanda di un certo bene sul mercato.

		Vorremmo calcolare il prezzo di equilibrio  tra le funzioni di domanda e offerta seguenti:

		Offerta: $q=\sqrt{100-\dfrac{300}{p^2}}$

		Domanda: $q=\dfrac{2400}{p^2}$
	}
	\solonly{$p\approx 15.64$ }

	\question
	\exonly{Determinare il polinomio di Newton $$P_k(x)=c_0+c_1(x-x_0)+\ldots +c_k(x-x_0)\ldots(x-x_{k-1})$$ passante per i punti  $$(1;3) \quad (2;5) \quad (3;1) \quad (6;-3)$$ }
	\solonly{$P(x)=3+2(x-1)-3(x-1)(x-2)+\dfrac{11}{15}(x-1)(x-2)(x-3)$ }


	%\subfile{../ch/BS-MPT-FP-fct-polyn}


	\question

	\exonly{
		Il cubo grande ha un lato di \SI{10}{\centi\metre}. Si vuole costruire un parallelepipedo a base quadrata come in figura:


		%\includegraphics[scale=1]{cubo}

		\begin{tikzpicture}[baseline={($(current bounding box.north)-(0,1.6ex)$)}]
			%\tikzstyle{isometric}=[x={(0.710cm,-0.410cm)},y={(0cm,0.820cm)},z={(-0.710cm,-0.410cm)}]
			%\tikzset{every path/.style={isometric}}
			\tikzset{face/.style={fill=gray!20}}
			\pgfmathsetmacro{\cubex}{3}
			\pgfmathsetmacro{\cubey}{3}
			\pgfmathsetmacro{\cubez}{3}
			\pgfmathsetmacro{\ratio}{0.4}

			\coordinate (FTR) at (0,0,0);
			\coordinate (FTL) at (-\cubex,0,0);
			\coordinate (FBL) at (-\cubex,-\cubey,0);
			\coordinate (FBR) at (0,-\cubey,0);
			\coordinate (BTR) at (0,0,-\cubez);
			\coordinate (BBR) at (0,-\cubey,-\cubez);
			\coordinate (BTL) at (-\cubex,0,-\cubez);
			\coordinate (BBL) at (-\cubex,-\cubey,-\cubez);


			\draw[dotted] (FTR)  -- (FTL)  -- (FBL)  -- (FBR)  -- cycle;
			\draw[dotted] (FTR) -- (BTR)  -- (BTL)  -- (FTL) -- cycle;
			\draw[dotted] (FTR) -- (FBR) -- (BBR)  -- (BTR) -- cycle;
			\dimline[line style = {line width=0.7},extension start length=-0.5cm,extension end length=-0.5cm] {($(FTL)+(-0.5,0,0)$)}{($(FBL)+(-0.5,0,0)$)}{\SI{10}{\centi\metre}};


			\pgfmathsetmacro{\cubex}{1}
			\pgfmathsetmacro{\cubey}{2}
			\pgfmathsetmacro{\cubez}{1}
			%\path (FBR) -- node[midway,right] {$x$}+(0,\cubex,0);
			\dimline[line style = {line width=0.7},
				extension start length=-0.24,
				extension end length=-0.24] {(FBR)}{($ (FBR) + (0,\cubex,0) $)}{$x$};

			\coordinate (FTR) at (0,0,0);
			\coordinate (FTL) at (-\cubex,0,0);
			\coordinate (FBL) at (-\cubex,-\cubey,0);
			\coordinate (FBR) at (0,-\cubey,0);
			\coordinate (BTR) at (0,0,-\cubez);
			\coordinate (BBR) at (0,-\cubey,-\cubez);
			\coordinate (BTL) at (-\cubex,0,-\cubez);
			\coordinate (BBL) at (-\cubex,-\cubey,-\cubez);


			\draw[thick] (FTR)  --(FTL)  -- (FBL)  -- (FBR)  -- cycle;
			\draw[thick] (FTR) -- (BTR)  -- (BTL)  -- (FTL) -- cycle;
			\draw[thick,face] (FTR) -- (FBR) -- (BBR)  -- (BTR) -- cycle;
			\dimline[line style = {line width=0.7},
				extension start length=-0.24,
				extension end length=-0.24] {(FTR)}{(FTL)}{$x$};


		\end{tikzpicture}

		Determinare i valori approssimati di $x$ tali per cui il volume del parallelepipedo sia di \SI{147}{\cubic\centi\metre}.

		SFIDA : Determinare i valori esatti di $x$ tali per cui il volume del parallelepipedo sia di \SI{147}{\cubic\centi\metre}.
	}

	\solonly{$7$ e $\dfrac{3+\sqrt{93}}{2}\approx 6.32$ }

	\question
	\exonly{In una giornata di fine estate la temperatura di \SI{20}{\degreeCelsius}  è stata misurata alle 00h00, alle 09h00 e di nuovo alle 24h00.

		La temperatura minima è stata riscontrata alle 04h00 ed era di \SI{17}{\degreeCelsius}.

		Stabilire un modello polinomiale, $T(x)$ della temperatura in funzione dell'ora per la suddetta giornata e rappresentarla graficamente. }
	\solonly{$T(x)=-\dfrac{3}{400}x(x-9)(x-24)+20$

		\begin{tikzpicture}[baseline={($(current bounding box.north)-(0,1.6ex)$)}]
			\begin{axis}[AxisDefaults,
					width=\linewidth,
					ytick distance={1},
					ymin=0,
					ymax=30,
					ylabel=$T(x)$
				]
				\addplot[ultra thick,draw=red,smooth,unbounded coords=jump,
					restrict y to domain=0:35,
					domain=0:24
				]{-x*(x-9)*(x-24)*3/400+20};
			\end{axis}
		\end{tikzpicture}
	}
\end{questions}