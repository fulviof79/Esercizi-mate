\section{Funzioni armoniche}

\subsection{Cerchio trigonometrico, relazioni fondamentali}

\begin{questions}


	\begin{qblock}
		\question
		\exonly{Sapendo che $\cos \alpha = \dfrac{3}{5}$ determinare:}

		\begin{parts}
			\part
			\exonly{i valori esatti di $\sin \alpha$ e $\tan \alpha$}
			\solonly{$\sin \alpha = \dfrac{4}{5}$ o $\sin \alpha=-\dfrac{4}{5}$ \\
				$\tan \alpha = \dfrac{4}{3}$ o $\tan \alpha=-\dfrac{4}{3}$
			}
			\part
			\exonly{in quale quadrante può trovarsi $\alpha$}
			\solonly{Quadranti $I$ o $IV$}
		\end{parts}
	\end{qblock}



	\begin{qblock}
		\question
		\exonly{Sapendo che $\sin \alpha = \dfrac{3}{5}$ determinare:}

		\begin{parts}
			\part
			\exonly{i valori esatti di $\cos \alpha$ e $\tan \alpha$}
			\solonly{$\cos \alpha = \dfrac{4}{5}$ o $\cos \alpha=-\dfrac{4}{5}$ \\
				$\tan \alpha = \dfrac{3}{4}$ o $\tan \alpha=-\dfrac{3}{4}$
			}

			\part
			\exonly{in quale quadrante può trovarsi $\alpha$}
			\solonly{Quadranti $I$ o $II$}
		\end{parts}
	\end{qblock}



	\begin{qblock}
		\question

		\exonly{Sapendo che $\cos \alpha = \dfrac{1}{3}$ determinare, senza calcolatrice, i valori esatti di $\sin \alpha$ e $\tan \alpha$.}
		\solonly{$\sin \alpha = \dfrac{\sqrt{8}}{3}$ e $\tan \alpha = \sqrt{8}$}

		\exonly{In seguito,e che $\alpha$ si trova nel primo quadrante, determinare i valori esatti di:}

		\begin{parts}
			\part
			\exonly{$\cos \alpha \sin \alpha-\cos \left(180^{\circ}+\alpha\right) \sin \left(180^{\circ}+\alpha\right)$}
			\solonly{$0$}

			\part
			\exonly{$\cos (-\alpha) \cos \left(270^{\circ}+\alpha\right)+2 \cdot \sin \left(90^{\circ}-\alpha\right) \sin \left(180^{\circ}-\alpha\right)$}
			\solonly{$\frac{2 \sqrt{2}}{3}$}

			\part
			\exonly{$\tan (\alpha) \tan \left(90^{\circ}-\alpha\right)-\tan \left(180^{\circ}+\alpha\right) \tan (-\alpha)$}
			\solonly{$9$}

		\end{parts}
	\end{qblock}



	\begin{qblock}
		\question
		\exonly{Determinare, senza calcolatrice, il valore esatto di:}

		\begin{parts}
			\part
			\exonly{$2 \sin \left(60^{\circ}\right) \cos \left(-60^{\circ}\right)-\sin \left(-240^{\circ}\right)$}
			\solonly{$0$}
			\part
			\exonly{$\frac{\tan \left(240^{\circ}\right)}{\tan \left(330^{\circ}\right)}-\sin \left(135^{\circ}\right) \cdot \cos \left(315^{\circ}\right)$}
			\solonly{$-\dfrac{7}{2}$}

			\part
			\exonly{$-3 \cdot \sin \left(x-\frac{3 \pi}{2}\right) \cdot \sin \left(x-\frac{5 \pi}{2}\right)+3 \cdot \cos \left(x+\frac{\pi}{2}\right) \cdot \sin (x+\pi)$}

			\solonly{$3$}
		\end{parts}
	\end{qblock}

	\end{questions}



	\subsection{Funzioni e rappresentazione grafica}

	\begin{questions}

	\begin{qblock}
		\question \label{ex:rappr}
		\exonly{Determinare la legge di assegnazione delle funzioni rappresentate nella forma   $f(x)=a \sin (b x +c)+d$.}
		\begin{parts}
			\part
			 \ifprintanswers   \else 
				 \begin{tikzpicture}[baseline={($(current bounding box.north)-(0,1.6ex)$)}]
					\begin{axis}[
						AxisDefaults,
						SmallAxisLabels,
						TrigDefaults,
						%width=\linewidth,
						height=8cm,
						domain=-3*pi:3*pi,
						ymin=-5,
						ymax=+5,
						minor y tick num = 1,
						%	minor tick num=5,
						xtick distance=pi/2,
						]
						\addplot[draw=black] {4*sin(x+pi)};
					\end{axis}
				 \end{tikzpicture}
			  \fi
			  	
			 \solonly{$f(x)=4\sin(x+\pi)$} 
			 
			\part
			 \ifprintanswers   \else    
				 \begin{tikzpicture}[baseline={($(current bounding box.north)-(0,1.6ex)$)}]
					\begin{axis}[
						AxisDefaults,
						SmallAxisLabels,
						TrigDefaults,
						%width=\linewidth,
						height=8cm,
						domain=-3*pi:3*pi,
						ymin=-5,
						ymax=+5,
						minor y tick num = 1,
						%	minor tick num=5,
						xtick distance=pi/2,
						]
						\addplot[draw=black] {3*sin(2*x+pi/2)};
					\end{axis}
				 \end{tikzpicture}
				 
			  \fi
			  	
			 \solonly{$f(x)=3\sin \left( 2x+\dfrac{\pi}{2}\right) $} 
		\end{parts}
	\end{qblock}
			 


	\begin{qblock}
		\question \label{ex:rappr}
		\exonly{Determinare la legge di assegnazione delle funzioni rappresentate nella forma   $f(t)=A \sin (\omega t+\varphi)+b$.}
		\begin{parts}
			\part
			\ifprintanswers   \else 
				 \begin{tikzpicture}[baseline={($(current bounding box.north)-(0,1.6ex)$)}]
					\begin{axis}[
						AxisDefaults,
						SmallAxisLabels,
						%TrigDefaults,
						%width=\linewidth,
						height=8cm,
						domain=-4:5,
						xlabel=$\frac{t}{\unit{\second}}$,
						ymin=-5,
						ymax=+5,
						minor y tick num = 1,
						%	minor tick num=5,
						xtick distance=1,
						trig format plots=rad,
						]
						\addplot[draw=black] {2*sin(x*pi/2-pi/2)+1};
					\end{axis}
				 \end{tikzpicture}
			 \fi
			 
			  \solonly{$f(x)=2\sin \left( \dfrac{\pi}{2}t-\dfrac{\pi}{2}\right) +1$} 
			  
			\part
			 \ifprintanswers   \else    
				 \begin{tikzpicture}[baseline={($(current bounding box.north)-(0,1.6ex)$)}]
					\begin{axis}[
						AxisDefaults,
						SmallAxisLabels,
						%TrigDefaults,
						%width=\linewidth,
						height=8cm,
						domain=-3:3,
						xlabel=$\frac{t}{\unit{\second}}$,
						ymin=-5,
						ymax=+5,
						minor y tick num = 1,
						minor x tick num=3,
						xtick distance=1,
						trig format plots=rad,
						]
						\addplot[draw=black] {3*sin(2*pi*x+pi/2)};
					\end{axis}
				 \end{tikzpicture}
				 
			  \fi
			  	
			 \solonly{$f(t)=3\sin \left( 2\pi t+\dfrac{\pi}{2}\right) $} 
			 
		\end{parts}
	\end{qblock}



	\begin{qblock}
		\question
		\exonly{Risolvere graficamente $f(x)=2$ per le funzioni rappresentate nell'esercizio (\thesubsection.\ref{ex:rappr})  }
	\end{qblock}

	\end{questions}



	\subsection{Equazioni trigonometriche}

	\begin{questions}

	\begin{qblock}
		\question
		\exonly{Risolvere le equazioni segeuenti indicando tutte le soluzioni (periodo compreso) }

		\begin{parts}
			\part
			\exonly{$\sin(x)=-\frac{\sqrt{2}}{2} $ }
			\solonly{$\es{\frac{5\pi}{4}+2k\pi,\frac{7\pi}{4}+2k\pi \mid k \in \Z}$ }

			\part
			\exonly{$\tan(\theta)=\sqrt{3}$ }
			\solonly{$\es{\frac{\pi}{3}+k\pi \mid k \in \Z}$ }

			\part
			\exonly{$\sin(x)=\frac{\pi}{2}$ }
			\solonly{$S=\emptyset$ }

			\part
			\exonly{$2\cos(2\theta)-\sqrt{3}=0$ }
			\solonly{$\es{\frac{\pi}{12}+k\pi,\frac{11\pi}{12}+k\pi \mid k \in \Z}$ }

			\part
			\exonly{$\sin\left( \theta + \frac{\pi}{4}\right) =\frac{1}{2}$ }
			\solonly{$\es{\frac{7\pi}{12}+2k\pi,\frac{23\pi}{12}+2k\pi \mid k \in \Z}$ }

			\part
			\exonly{$\sin\left( 2x-\frac{\pi}{3}\right) =\frac{1}{2}$ }
			\solonly{$\es{\frac{\pi}{4}k\pi,\frac{7\pi}{12}+k\pi  \mid k \in \Z} $}
		\end{parts}	
	\end{qblock}



	\begin{qblock}
		\question
		\exonly{Risolvere le equazioni seguenti nell'intervallo  $[0\degree ; 360 \degree[$}


		\begin{parts}
			\part
			\exonly{$\cos \left( x\right) =\frac{\sqrt{3}}{2}$}
			\solonly{$ \es{30\degree}{330\degree}$}

			\part
			\exonly{$\tan \left( x\right) =-\tan 20^{\circ}$}
			\solonly{$ \es{160\degree}{340\degree}$}

			\part
			\exonly{$\sin \left( x\right) =\sin 150^{\circ}$}
			\solonly{$ \es{30\degree}{150\degree}$}

			\part
			\exonly{$\sin \left( x\right) =-\sqrt{3} \cos \left( x\right) $}
			\solonly{$ \es{120\degree}{300\degree}$ }

			\part
			\exonly{$\sin ^{2} \left( x\right) +\frac{1}{2} \sin\left(x\right) =0$}
			\solonly{$ \es{0\degree}{180\degree}{210\degree}{330\degree}$}

		\end{parts}
	\end{qblock}



	\begin{qblock}
		\question
		\exonly{Risolvere le seguenti equazioni nell'intervallo $[0,2\pi[$}

		\begin{parts}
			\part
			\exonly{$\sin \left( x\right) +\frac{1}{\sqrt{3}} \cos \left( x\right) =0$}

			\solonly{$ \es{\frac{5\pi}{6}}{\frac{11\pi}{6}}$}

			\part
			\exonly{$1-\sin \left( x\right) =\sqrt{3} \cos \left( x\right) $}
			\solonly{$ \es{\frac{\pi}{2}}{\frac{11\pi}{6}}$}

			\part
			\exonly{$2 \cos ^{2} \left( t\right) +3 \cos\left(  t\right) +1=0$}
			\solonly{$ \es{\pi}{\frac{2\pi}{3}}{\frac{4\pi}{3}}$} 

			\part
			\exonly{$2 \sin ^{2} \left( u\right) +\sin \left( u\right) -3=0$}
			\solonly{$\es{\frac{\pi}{2}}$}

			\part
			\exonly{$ \sin \left( x- \dfrac{\pi}{4}\right) =\dfrac{1}{\sqrt{2}}$}
			\solonly{$\es{\frac{\pi}{2}+2k\pi,\pi + 2k\pi | k \in \Z}$}


			\part
			\exonly{$ \cos \left( 2x+ \dfrac{\pi}{6}\right) =-\dfrac{1}{2}$}
			\solonly{$\es{\frac{\pi}{4}+k\pi,\dfrac{7\pi}{12} + k\pi | k \in \Z}$}
		\end{parts}
	\end{qblock}

	\end{questions}



	\subsection{Applicazioni}

	\begin{questions}

	\begin{qblock}
		\question
		\exonly{
			Il processo ritmico della respirazione consiste in un'alternanza di periodi di inspirazione e espirazione.
			Un ciclo completo dura normalmente $5$ secondi.

			Se $F(t)$ descrive il flusso d'aria al tempo $t$ (in secondi) e se il flusso massimo é di \num{0.6} litri al secondo.
		}

		\begin{parts}
			\part
			\exonly{ modellizzare $F(t)=a \sin (bt)$}
			\solonly{$F(t)=0.6 \sin\left(\dfrac{2\pi}{5}t\right)$}

			\part
			\exonly{Determinare in quali istanti $t$ il flusso é di \num{0.2} litri al secondo.}

			\solonly{$\num{0.27}+ 5k$ e $\num{2.23}+5k$ , $k\in \Z$}
		\end{parts}
	\end{qblock}



	\begin{qblock}
		\question 
		\exonly{
			Il modello $f(t) = a \sin (bt + c) + d$ viene talvolta usato per simulare le variazioni di temperatura durante la giornata.   

			$t$ é il tempo in ore, $f(t)$ la temperatura in \si{\celsius} e $t=0$ corrisponde a mezzanotte.

			Supponiamo che $f(t)$ sia decrescente a partire da mezzanotte.

			Determinare i valori di $a$, $b$, $c$ et $d$ rappresentare graficamente $f(t)$ per $0 \leq t \leq 24$ nelle situazioni seguenti:
		}

		\begin{parts}
			\part
			\exonly{La temperatura massima é di \SI{10}{\celsius} e la temperatura minima di \SI{-10}{\celsius} viene registrata alle   $4$ del mattino.}

			\solonly{$f(t)=10\sin\left(\dfrac{\pi}{12}(t-10)\right)=\\ =10\sin\left(\dfrac{\pi}{12}t-\dfrac{5\pi}{6}\right)$

				oppure

				$f(t)=-10\sin\left(\dfrac{\pi}{12}(t+2)\right)=\\ =10\sin\left(\dfrac{\pi}{12}t+\dfrac{\pi}{6}\right)$

			}

			\ifprintanswers 
				 \begin{tikzpicture}[baseline={($(current bounding box.north)-(0,1.6ex)$)}]
					\begin{axis}[
						AxisDefaults,
						SmallAxisLabels,
						trig format plots=rad,
						width=\linewidth,
						ymin=-11,
						ymax=+11,
						xtick distance=2,
						ytick distance =5,
						minor tick num=0,
						]
						\addplot[draw=red,smooth,unbounded coords=jump,domain=0:24]{10*sin(x*pi/12-5*pi/6)}; 
					\end{axis}
				\end{tikzpicture}
			\fi

			\part
			\exonly{
			La temperatura a mezzanotte é di \SI{15}{\degreeCelsius} e le temperature massime e minime sono rispettivamente di  \SI{20}{\degreeCelsius} e \SI{10}{\degreeCelsius}.}

			\solonly{$f(t)=5\sin\left(\dfrac{\pi}{12}t-\pi\right)+15$  o \\\medskip $f(t)=-5\sin\left(\dfrac{\pi}{12}t\right)+15$}

			\ifprintanswers 
				 \begin{tikzpicture}[baseline={($(current bounding box.north)-(0,1.6ex)$)}]
					\begin{axis}[
						AxisDefaults,
						SmallAxisLabels,
						trig format plots=rad,
						width=\linewidth,
						domain=-1:25,
						ymin=-0,
						ymax=+21,
						xtick distance=2,
						ytick distance =5,
						]		
						\addplot[draw=red,smooth,unbounded coords=jump,domain=0:24]{-5*sin(x*pi/12)+15}; 
					\end{axis}

				\end{tikzpicture}
			\fi

			\part
			\exonly{La temperatura varia tra\SI{10}{\degreeCelsius} e \SI{30}{\degreeCelsius} e la temperatura media di \SI{20}{\degreeCelsius} viene registrata per la prima volta alle
			$9$ del mattino.}

			\solonly{$f(t)=10\sin\left(\dfrac{\pi}{12}(t-9)\right)+20$}

		\end{parts}

		\begin{profonly}	Src: Swok-trigo ex 55 pg 419	\end{profonly}
	\end{qblock}



	\begin{qblock}
		\question
		\exonly{La ruota panoramica di Londra ha un diametro di \SI{120}{\metre}, un’ altezza di \SI{135}{\metre} e compie un giro in $30$ minuti.
		}

		\begin{parts}
			\part
			\exonly{Determina l’altezza $H(t)$ in funzione del tempo della cabina che parte dal basso all’istante $t = 0$.}
			\solonly{$t$ in minuti: $H(t)=60\sin(\dfrac{2\pi}{30} t-\dfrac{\pi}{2})+75$ }

			\part
			\exonly{Rappresenta graficamente $H(t)$}
			\ifprintanswers 
				 \begin{tikzpicture}[baseline={($(current bounding box.north)-(0,1.6ex)$)}]
					\begin{axis}[
						AxisDefaults,
						SmallAxisLabels,
						trig format plots=rad,
						width=\linewidth,
						domain=-9:45,
						ymin=0,
						ymax=140,
						xtick distance=10,
						ytick distance =10,
						minor y tick num=1,
						xlabel=$t$,
						ylabel=$H(t)$,
						extra x ticks={-7.5},
						]		
						\addplot[draw=red,smooth,unbounded coords=jump]{60*sin(x*pi/15-pi/2)+75}; 
						\draw[dashed] (axis cs:-7.5,0) -- (axis cs:-7.5,75);
					\end{axis}

				\end{tikzpicture}
			\fi

			\part
			\exonly{Quando si troverà a  \SI{100}{\metre} di altezza?}
			\solonly{\SI{9.55}{\minute} e \SI{20.45}{\minute}}

			\part
			\exonly{Dopo 1 ora e 40 minuti quante volte si sarà trovata a \SI{40}{\metre} di altezza?}

			\solonly{7 volte}
			\part
			\exonly{Quanto tempo impiega per tornare al punto più basso la cabina che si trova a \SI{110}{\metre} da terra e sta salendo?}
			\solonly{\SI{19.53}{\minute}}
		\end{parts}

		\begin{profonly}	Src: Swok-trigo exemple 11 pg 413	\end{profonly}
	\end{qblock}



	\begin{qblock}
		\question
		\exonly{Una massa appesa ad una molla oscilla verticalmente con una frequenza
		di \SI{1.3}{\hertz} scostandosi dalla posizione iniziale di \SI{5.3}{\centi\metre} al massimo. }

		\begin{parts}
			\part
			\exonly{Sapendo che all'istante $t = 3$ \si{\second} si trova \SI{2}{\centi\metre} al di sotto della posizione iniziale e sta scendendo determina la sua legge oraria. }
			\solonly{$f(t)=5.3\cdot \sin(2\pi \cdot 1.3 \cdot t -2.1237)$ }
			\part
			\exonly{Quanto dura un'oscillazione completa? }
			\solonly{\SI{0.77}{\second} }
			\part
			\exonly{Quanto tempo impiega dal punto più basso per tornare alla posizione di
			riposo? }
			\solonly{\SI{0.19}{\second} }
			\part
			\exonly{Dove si trova all’istante $t = 35$ \si{\second}? }
			\solonly{\SI{4.51}{\centi\metre} }
			\part
			\exonly{Quante oscillazioni complete esegue in \SI{3}{\minute}? }
			\solonly{234 }
		\end{parts}
	\end{qblock}



	\begin{qblock}
		\question

		\exonly{Il valore di picco di una tensione alternata è di $U = \SI{300}{\volt}$, la sua frequenza è di \SI{50}{\hertz}. 

			Il valore istantaneo a $t = \SI{0}{\second}$ è di $U = \SI{50}{\volt}$ e sta aumentando.
		}

		\begin{parts}
			\part
			\exonly{Determina la funzione $U(t)$}
			\solonly{
				$U(t)=300\sin(100\pi t +0.16745)$ \si{\volt}
			}

			\part
			\exonly{La tensione $U(t)$ viene applicata ai capi di una resistenza $R = \SI{5}{\kilo\ohm}$ determina
			l’intensità della corrente $I(t)$ che scorre nella resistenza. }
			\solonly{
				$I(t)=0.06 \sin(100\pi t +0.16745)$  \si{\ampere}
			}

			\part
			\exonly{Qual’è il valore istantaneo dell’intensità di corrente all’istante $t = \SI{0}{\second}$?
			}

			\solonly{
				\SI{0.01}{\ampere}
			}
			\part
			\exonly{Determina i valori di $U$ e $I$ all’istante $t =\SI{3.5}{\second}$. }
			\solonly{$U(3.5)=\SI{50}{\volt}$\\
				$I(3.5)=\SI{0.01}{\ampere}$
			}

			\part

			\exonly{Quante oscillazioni complete compie in \SI{3}{\second}?
			}
			\solonly{$150$ }
		\end{parts}
	\end{qblock}



	\begin{qblock}
		\question
		\exonly{
			Sono date le intensità di corrente:

			\begin{align*}
				I_1(t) & =\SI{1.5}{A} \cdot \sin{(100\pi \si{s^{-1}} \cdot t + \frac{\pi}{3})} \\
				I_2(t) & =\SI{0.8}{A}\cdot \sin{(100\pi \si{s^{-1}} \cdot t + \frac{\pi}{6})} \\
				I_3(t) & =\SI{0.5}{A}\cdot \sin{(100\pi \si{s^{-1}} \cdot t + \frac{\pi}{4})}
			\end{align*}
		}

		\begin{parts}
			\part
			\exonly{Aiutandoti con la calcolatrice (Desmos, GeoGebra, \ldots ) traccia un grafico accurato di \begin{equation*}
					I(t)=I_1(t) + I_2(t) - I_3(t)
			\end{equation*} }

			\ifprintanswers 
				\begin{tikzpicture}[baseline={($(current bounding box.north)-(0,1.6ex)$)}]
					\begin{axis}[
						AxisDefaults,
						SmallAxisLabels,
						trig format plots=rad,
						width=\linewidth,
						domain=-0.02:0.06,
						ymin=-4,
						ymax=4,
						xtick distance=0.01,
						%ytick distance =10,
						xmin=-0.02,
						xmax=0.06,
						%minor y tick num=1,
						xlabel=$t$,
						ylabel=$I(t)$,
						%extra x ticks={-7.5},
						]		
						\addplot[draw=red,smooth,unbounded coords=jump]{1.731 *sin(100*pi*x+0.889)}; 

					\end{axis}

				\end{tikzpicture}
			\fi

			\part 
			\exonly{In base al grafico ottenuto determina la funzione $I(t)$ }
			\solonly{$I(t)=1.731\sin\left(100\pi\left(t+0.00283\right)\right)$ \\
				$I(t)=1.731\sin\left(100\pi t +0.899\right)$
			}

			\part 
			\exonly{Verifica la correttezza del risultato utilizzando i numeri complessi }
			\solonly{$1.73\angle 0.89$ }
		\end{parts}		
	\end{qblock}



\end{questions}

