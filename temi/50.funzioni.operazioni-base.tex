\section{Funzioni e operazioni}

\subsection{Riepilogo funzioni dei 1$^{\circ}$  et 2$^{\circ}$ grado}
%%%%%%%%%%%%%%%%%%%%%%%%%%%%%%%%%%%%%%%%%%%%%%%%%%%%%%%%%%%%%%%%%%%%%%
%\subfile{../ch/BS-MPT-FP-rappel-fct-1-2}
\begin{questions}

    \question
    \exonly{Siano le tre funzioni seguenti :

        $f(x)=2x+2$

        $g(x)=-\frac{1}{2}x-8$

        $h(x)=x^2+8x+7$ }

    \begin{parts}
        \part
        \exonly{
            Determinare algebricamente i punti di intersezione con gli assi  delle tre funzioni.
            %		Déterminer algébriquement les points d'intersection avec les axes des trois fonctions. 
        }

        \solonly{
            $f(x):$
            $I_x(-1;0) \quad I_y(0;2)$

            $g(x):$
            $I_x(-16;0) \quad I_y(0;-8)$

            $h(x):$
            $I_{x_1}(-7;0) \quad I_{x_2}(-1;0) \quad I_y(0;7)$

        }

        \part
        \exonly{
            Determinare algebricamente la pendenza delle funzioni $f$ e $g$. Cosa si può dire?
            %		Déterminer algébriquement la pente de la fonction $f$ et de la fonction $g$. Que peut-on dire?
        }

        \solonly{
            $f(x):$ pendenza$=2$

            $g(x):$ pendenza$=-\frac{1}{2}$

            Rette perpendicolari
        }

        \part
        \exonly{
            Determinare algebricamente il vertice della funzione $h$.
            %		Déterminer algébriquement le sommet de la fonction $h$. 
        }
        \solonly{$S(-4;-9)$}

        \part
        \exonly{
            Determinare algebricamente i punti di intersezione tra le funzioni $f$ e $g$.
            %		Déterminer algébriquement les points d'intersection entre la fonction $f$ et $g$. 
        }
        \solonly{$I(-4;-6)$}
        \part
        \exonly{
            Determinare algebricamente i punti di intersezione tra le funzioni $f$ e $h$.
            %		Déterminer algébriquement les points d'intersection entre la fonction $f$ et $h$. 
        }
        \solonly{$I_1(-1;0)$ et $I_2(-5;-8)$}

        \part
        \exonly{
            Disegnare le tre funzioni nello stesso sistema di riferimento.
            %	Dessiner chacune des trois fonctions dans le même repère. 
        }
        \ifprintanswers
            \begin{tikzpicture}[baseline={($(current bounding box.north)-(0,1.6ex)$)}]
                \begin{axis}[
                        AxisDefaults,
                        TinyAxisLabels,
                        width=0.8\linewidth,
                        ytick distance={2},
                        ymin=-10,
                        ymax=10,
                        domain=-10:10,
                        restrict y to domain=-11:11,
                    ]
                    \addplot[draw=red]{2*x+2} node[pos=0.8,above] {$f(x)$};
                    \addplot[draw=blue]{-x/2-8}node[pos=0.8,above] {$g(x)$};
                    \addplot[draw=violet,smooth	]{x^2+8*x+7} node[pos=0.2,right] {$h(x)$};
                \end{axis}
            \end{tikzpicture}
        \fi
    \end{parts}


\end{questions}

\subsection{Dominio, immagini, iniettività, suriettività, biettività}

\begin{questions}
    \question
    \exonly{Consideriamo la funzione $f: A \mapsto B$
        con legge di assegnazione $f(x)=x^2+4x-5$.

        Quali di queste affermazioni sono vere?
    }
    \solonly{Affermazioni vere $\blacksquare$ }

    \begin{checkboxes}

        \choice $f$ può essere resa suriettiva se $B=[-10;\infty[$

        \CorrectChoice $f$ può essere resa suriettiva se $B=[-9;\infty[$

        \CorrectChoice $f$ é iniettiva se $A=[-2;\infty[$

        \choice $f$ é biettiva per $A=[-\infty;-2[$ e $B=]-\infty;-9[$

        \CorrectChoice $f$ é biettiva per $A=[-\infty;-2[$ e $B=[-9;\infty[$

        \CorrectChoice $f$ é biettiva per $A=[-2;\infty[$ e $B=[-9;\infty[$

        \CorrectChoice il dominio naturale di $f$ è $\DD=\R$

        \choice $f$ é iniettiva per $A=\R$

        \choice $f$ é biettiva per $A=\R$ e $B=\R$

    \end{checkboxes}

    \question
    \exonly{Consideriamo la funzione $f: A \mapsto B$
        con legge di assegnazione $f(x)=(x+2)^3-9$.

        Quali di queste affermazioni sono vere?
    }
    \solonly{Affermazioni vere $\blacksquare$ }

    \begin{checkboxes}

        \CorrectChoice $f$  può essere resa suriettiva se  $B=[0;\infty[$
        \CorrectChoice $f$  può essere resa suriettiva se  $B=[-9;\infty[$

        \CorrectChoice $f$ é iniettiva per $A=[-2;\infty[$

                    \choice $f$ é biettiva per $A=]-\infty;-2[$ e $B=[-9;\infty[$
                    \CorrectChoice $f$ é biettiva per $A=]-\infty;-2[$ e $B=]-\infty;-9[$
        \CorrectChoice il dominio naturale di $f$ $\DD=\R$
        \CorrectChoice $f$ é iniettiva per $A=\R$
        \CorrectChoice $f$ é biettiva per $A=\R$ e $B=\R$

    \end{checkboxes}
\end{questions}

\subsection{Operazioni sulle funzioni}

\begin{questions}

    \question
    \exonly{
        Date le funzioni  $f(x)=x^2+2$ e \\ $g(x)=2x^2-1$.
        Calcolare:
        %	Soit les fonctions $f(x)=x^2+2$ et \\ $g(x)=2x^2-1$. 
        %Calculer:
    }

    \begin{parts}
        \part \exonly{$(f+g)(x)$ }
        \solonly{$3x^2+1$}
        \part \exonly{$(f-g)(x)$ }
        \solonly{$-x^2+3$}
        \part \exonly{$(fg)(x)$ }
        \solonly{$2x^4+3x^2-2$}
        \part
        \exonly{ $\left(\dfrac{f}{g}\right)(x)$ }

        \solonly{$=\dfrac{x^2+2}{2x^2-1}$}

        \part \exonly{
            Il dominio di $(f+g)(x)$
            %	Le domaine de définition de $(f+g)(x)$
        }
        \solonly{$\DD=\R $}
        \part \exonly{
            Il dominio  di $(\dfrac{f}{g})(x)$
            %	Le domaine de définition de $(\dfrac{f}{g})(x)$ 
        }
        \solonly{$\DD=\R \setminus \left\lbrace \pm \dfrac{ \sqrt{2}}{2}\right\rbrace $}
    \end{parts}

    \begin{profonly}	Src: Swok ex 3 pg 226/227	\end{profonly}





    \question
    \exonly
    {
        Date le funzioni $f(x)=\sqrt{x+5}$ e \\ $g(x)=\sqrt{x+5}$.
        Calcolare:
        %	Soit les fonctions $f(x)=\sqrt{x+5}$ et \\ $g(x)=\sqrt{x+5}$. 
        %Calculer:
    }

    \begin{parts}
        \part $(f+g)(x)$\solonly{$=2\sqrt{x+5}$}

        \part $(f-g)(x)$\solonly{$=0$}

        \part $(fg)(x)$\solonly{$=x+5$}

        \part $\left(\dfrac{f}{g}\right)(x)$\solonly{$=\dfrac{\sqrt{x+5}}{\sqrt{x+5}}$}



        \part \exonly{
            Il dominio di  $(f+g)(x)$}\solonly{$\DD=\left[- 5 ; \infty \right[ $
        }

        \part \exonly{
            Il dominio di  $\left(\dfrac{f}{g}\right)(x)$ }\solonly{$\DD=\left] -5 ; \infty \right[ $
        }

    \end{parts}
    \begin{profonly}	Src: Swok ex 5 pg 226/227	\end{profonly}


    %\question \begin{profonly}	Src: Swok ex 7 pg 226/227	\end{profonly}

    \question
    \exonly{Date le funzioni  $f(x)=x^2-3x$ e \\ $g(x)=\sqrt{x+2}$.
        Determinare:
    }
    \begin{parts}
        \part $(f \circ g)(x)$\solonly{$=x+2-3\sqrt{x+2}$}
        \part \exonly{il dominio di $(f \circ g)(x)$}
        \solonly{$\DD=\left[ -2 ; \infty \right[ $}
        \part $(g \circ f)(x)$\solonly{$=\sqrt{x^2-3x+2}$}
        \part \exonly{il dominio $(g \circ f)(x)$}\solonly{$\DD=\left] -\infty ;1 \right] \cup \left[ 2; \infty \right[ $}
    \end{parts}

    \begin{profonly}	Src: Swok ex 21 pg 227	\end{profonly}



    \question
    \exonly{Date le funzioni $f(x)=x^2-4$ e \\ $g(x)=\sqrt{3x}$.
        Determinare:
    }
    \begin{parts}
        \part $(f \circ g)(x)$\solonly{$=3x-4$}
        \part \exonly{il dominio di $(f \circ g)(x)$}\solonly{$\DD=\left[ 0 ; \infty \right[ $}
        \part $(g \circ f)(x)$\solonly{$=\sqrt{3x^2-12}$}
        \part \exonly{il dominio di $(g \circ f)(x)$}\solonly{$\DD=\left] -\infty ;-2 \right] \cup \left[ 2; \infty \right[ $}
    \end{parts}


    \begin{profonly}	Src: Swok ex 23 pg 227	\end{profonly}

    \solnewpage
    \question
    \begin{parts}
        \part
        \exonly{
            Determinare il dominio naturale $\DD_f$ e l'insieme delle immagini $Im_f$.}


        \part
        \exonly{
            Determinare se la funzione $f: \DD_f \mapsto Im_f$ é biettiva.}
        \begin{checkboxes}
            \choice $f(x)=x^2-9$ %\solonly{$\DD=\left]0;\infty\right[$}
            \CorrectChoice $f(x)=3x-7$

            \CorrectChoice $f(x)=\sqrt{x}$
            \choice $f(x)=\sqrt{4-x^2}$
        \end{checkboxes}

        \part
        \exonly{
            In caso contrario restringere il dominio e determinarne le immagine in modo da renderla biettiva.}
    \end{parts}




    \begin{profonly}	Src: Swok ex 1-12 pg 236	\end{profonly}

    \question
    \exonly{Determinare la funzione inversa (o reciproca) di $f$ sul loro dominio.}

    \begin{parts}

        \part
        \exonly{$f(x)=3x+5$} \solonly{$f^{-1}(x)=\dfrac{x-5}{3}$}

        \part
        \exonly{$f(x)=\dfrac{3x+2}{2x-5}$}

        \solonly{$f^{-1}(x)=\dfrac{5x+1}{2x-3}$}

        \part
        \exonly{$f(x)=2-3x^2$ , $x\leq 0$}
        \solonly{$f^{-1}(x)=-\sqrt{\dfrac{2-x}{3}}$}

        \part
        \exonly{$f(x)=\sqrt{3-x}$} \solonly{$f^{-1}(x)=3-x^2$ , $x \geq 0$}


    \end{parts}

    \begin{profonly}	Src: Swok ex 17,21,23,27 pg 236	\end{profonly}





\end{questions}

\subsection{Funzioni pari e dispari}
\begin{questions}
    \question
    \exonly{Determinare se $f$ é una funzione pari, dispari o nessuna delle due cose.}

    \begin{parts}
        \part
        \exonly{$f(x)=5x^3+2x$}
        \solonly{Dispari}

        \part
        \exonly{$f(x)=3x^4+2x^2-5$}
        \solonly{Pari}

        \part
        \exonly{$f(x)=8x^3-3x^2$}
        \solonly{Né pari né dispari}

        \part
        \exonly{$\sqrt{x^2+1}$ }
        \solonly{Pari}

    \end{parts}
\end{questions}