\section{Calcolo differenziale e integrale}
\subsection{Calcolo delle derivate di base}
\begin{questions}
	\question
	\exonly{
		Determinare la derivata delle seguenti funzioni reali:
	}

	\begin{parts}
		\part
		\exonly{ $f(x)=3x^2-5x+2$ }
		\solonly{$6x-5$}
		\part
		\exonly{$f(x)=2x^4-3x^2+5$ }
		\solonly{$8x^3-6x^2$ }
		\part
		\exonly{$f(x)=(x+3)^2$ }
		\solonly{$2(x+3)=2x+6$ }
		\part
		\exonly{$f(x)=(2x-3)(3x+4)$ }
		\solonly{ $12x-1$  }
		\part
		\exonly{$f(x)=(x^2-3x+2)(2x-5)$ }
		\solonly{ $ 6 x^21- 22 x +9$  }

	\end{parts}

	\exonly{	Verificare la correttezza dei risultati utilizzando la calcolatrice o geogebra. }


	\question
	\exonly{	Determinare la derivata delle seguenti funzioni reali:}

	\exonly{	\begin{multicols}{2}}
			\begin{parts}
				\part
				\exonly{$ f(x)=3x-7 $}
				\solonly{$f'(x)=3$}
				\part
				\exonly{$ f(x)=3x+2 $}
				\solonly{$f'(x)=3$}
				\part
				\exonly{$ f(x)=-2x+4 $}
				\solonly{$f'(x)=-2$}
				\part
				\exonly{$ f(x)=-\frac{5}{3}x+\frac{4}{3} $}
				\solonly{$f'(x)=-\frac{5}{3}$}
				\part
				\exonly{$ f(x)=2x^2+3x+4 $}
				\solonly{$f'(x)=4x+3$}

				\part
				\exonly{$ f(x)=-3x^2-7x+5 $}
				\solonly{$f'(x)=-6x-7$}
				\part
				\exonly{$ f(x)=-3x^2-7x-2 $}
				\solonly{$f'(x)=-6x-7$}
				\part
				\exonly{$ f(x)=4x^3-2x^2+5 $}
				\solonly{$f'(x)=12x^2-4x$}
				\part
				\exonly{$ f(x)=2x^4-3x^2+3 $}
				\solonly{$f'(x)=8x^3-6x$}
				\part
				\exonly{$ f(x)=x^4-3x^3+x^2-5x+1 $}
				\solonly{$f'(x)=4x^3-9x^2-5$}
			\end{parts}
			\exonly{\end{multicols}}





	\question
	\exonly{	Determinare la derivata delle seguenti funzioni reali:}

	\exonly{\begin{multicols}{2}}
			\begin{parts}
				\part
				\exonly{$ f(x)=\frac{1}{x}$}
				\solonly{$f'(x)=-\frac{1}{x^2}$}
				\part
				\exonly{$ f(x)=\frac{3}{x^2} $}
				\solonly{$f'(x)=-\frac{6}{x^3}$}


				\part
				\exonly{$ f(x)=\sqrt{x} $}
				\solonly{$f'(x)=\frac{1}{2\sqrt{x}}$}
				\part
				\exonly{$ f(x)=\sqrt{x^3} $}
				\solonly{$f'(x)=\frac{3}{2}\sqrt{x}$}
			\end{parts}
			\exonly{\end{multicols}	}


	\question
	\exonly{
		Determinare la derivata delle seguenti funzioni reali:

		\begin{multicols}{2}
			\begin{parts}
				\setlength\itemsep{3mm}
				\part $ f(x)=\dfrac{x}{x-2} $
				\part $ f(x)=\dfrac{x+1}{x-1} $
				\part $ f(x)=\dfrac{3x-1}{2x+3} $
				\part $ f(x)=\dfrac{2x+1}{3x-2} $
				\part $ f(x)=\dfrac{2x^2+1}{5x-3} $
				\part $ f(x)=\dfrac{2x^2+3x-5}{x^3-2x+1} $
				\part $ f(x)=3x-\frac{1}{x}+\frac{2}{x^2} $
				\part $ f(x)=\dfrac{1}{2x+3}+3x^2-4$
			\end{parts}
		\end{multicols}
	}
	\solonly{Geogebra CAS}


	\question
	\exonly{
		Determinare la derivata delle seguenti funzioni reali:
		\medskip
		\begin{multicols}{2}
			\begin{parts}
				\setlength\itemsep{3mm}
				\part $ f(x)=2x+\sqrt{x} $
				\part $ f(x)=\sqrt{2x+3} $
				\part $ f(x)=\sqrt{4x^2} $
				\part $ f(x)=\sqrt[3]{4x^2} $
				\part $ f(x)=\sqrt[3]{2x^2-3x+5}$
				\part $ f(x)=\sqrt{4x^2} $
				\part $ f(x)=2\cdot \sqrt[3]{x}+5 \cdot \sqrt{x} $
				\part $ f(x)=\dfrac{\sqrt{x}}{x} $
				\part $ f(x)=\dfrac{x}{\sqrt{x}} $
				\part $ f(x)=\dfrac{x^2+3x-2}{\sqrt[3]{3x-2}} $
				\part $ f(x)=\dfrac{\sqrt{x+3}}{\sqrt{2x-1}}$
			\end{parts}
		\end{multicols}
	}
	\solonly{Geogebra CAS}

	\question
	\exonly{
		Determinare la derivata delle seguenti funzioni reali:
		\medskip
		\begin{multicols}{2}
			\begin{parts}
				\setlength\itemsep{3mm}
				\part $ f(x)=\sin(2x+3) $
				\part $ f(x)=\cos(3x^2-2x+5) $
				\part $ f(x)=\sin(x) +3 \cos(x) $
				\part $ f(x)=\sin(3x) +3 \cos(-2x) $
				\part $ f(x)=\sin(x)\cdot \cos(x) $
				\part $ f(x)=\sin(3x-5)\cdot \cos(2x-1) $
				\part $ f(x)=\dfrac{2\sin(x)}{\cos{x}} $
				\part $ f(x)=\sin(3x+2) + \cos(4x-3) $
				\part $ f(x)=\dfrac{\sin(2x+3)}{\sqrt{x-1}} $
				\part $ f(x)=\sqrt{\sin(2x+1)}$
				\part $ f(x)=\sin^2(x) + \cos^2(x) $
				\part $ f(x)=\sin^2(2x+3)  $
			\end{parts}
		\end{multicols}
	}
	\solonly{Geogebra CAS}

	\question
	\exonly{
		Determinare la derivata delle seguenti funzioni reali:
		\medskip
		\begin{multicols}{2}
			\begin{parts}
				\setlength\itemsep{3mm}
				\part $ f(x)=2x $
				\part $ f(x)=10^{3x} $
				\part $ f(x)=5\times 7^{x} $
				\part $ f(x)=/x $
				\part $ f(x)=2\cdot (3x-5) $
				\part $ f(x)=-2x+3 $
				\part $ f(x)=10^{x^2} $
				\part $ f(x)=x^2 $
				\part $ f(x)=(2x+1) \cdot (x+2 )$
				\part $ f(x)=10^{2x+1} \cdot 2^{x+2} $
			\end{parts}
		\end{multicols}
	}
	\solonly{Geogebra CAS}


	\question
	\exonly{
		Determinare la derivata delle seguenti funzioni reali:
		\medskip
		\begin{multicols}{2}
			\begin{parts}
				\setlength\itemsep{3mm}
				\part $ f(x)=\ln(2x) $
				\part $ f(x)=\lg(3x) $
				\part $ f(x)=4\ln(3x) $
				\part $ f(x)=\log_5(3x^2-5) $
				\part $ f(x)=2\log_2(2x-5) $
				\part $ f(x)=\ln(3x-5)\cdot \ln(2x+1) $
				\part $ f(x)=\lg(x^2-3x+2) $
				\part $ f(x)=\log_5(\frac{3}{4}x-\frac{5}{6}) $
			\end{parts}
		\end{multicols}

	}
	\solonly{Geogebra CAS}

	\question
	\exonly{
		Determinare la derivata delle seguenti funzioni reali:
		\medskip
		\begin{multicols}{2}
			\begin{parts}
				\setlength\itemsep{3mm}
				\part $ f(x)=\dfrac{1}{x} $
				\part $ f(x)=2x \cdot \ln(3x+1) $
				\part $ f(x)=\dfrac{x}{3x} $
				\part $ f(x)=\sqrt{10^{3x}} $
				\part $ f(x)=\dfrac{1}{\ln{x}} $
				\part $ f(x)=\dfrac{\ln(2x-5)}{\lg(3x+1)} $
				\part $ f(x)=\sqrt[3]{x^2+2x-1} $
				\part $ f(x)=10^x \cdot 2^x $
				\part $ f(x)=3^{2x-1} \cdot (x+2) $
				\part $ f(x)=\dfrac{x+2}{x+2} $
			\end{parts}
		\end{multicols}
	}
	\solonly{Geogebra CAS}


\end{questions}


\exnewpage
\solnewpage
\subsection{Analisi di funzioni}
\begin{questions}


	\question
	\exonly{
		È data la funzione:

		\begin{align*}
			f:\mathbb{R} & \rightarrow \mathbb{R} \\
			x            & \mapsto 2x^2-3x-5
		\end{align*}

		\begin{parts}
			\part Determinarne la derivata $f'(x)$.
			\part Rappresentare graficamente $f(x)$ e $f'(x)$ sullo stesso diagramma.
			\part Determinare le coordinate del vertice di $f(x)$.
			\part Determinare la pendenza di $f(x)$ nel punto di ascissa $x=2$.
			\part Determinare la pendenza di $f(x)$ nel punto di ascissa $x=-1$.
			\part Determinare in che punti $f(x)$ ha una pendenza di $-1$.
		\end{parts}
	}

	\question
	\exonly{
		Sono date le seguenti funzioni reali

		\begin{align*}
			f(x) & = \frac{1}{2} x^2 -\frac{3}{2} x + \frac{4}{2} \\
			g(x) & = -2x^2+x+3
		\end{align*}

		\begin{parts}
			\part Determinare i punti di intersezione tra $f(x)$ e $g(x)$.
			\part Determinare la pendenza di $f(x)$ e di $g(x)$ nel punto di ascissa $x=-4$
			\part Determinare in che punti $f(x)$ e $g(x)$ hanno pendenza $1$.
			\part Determinare in che punti $f(x)$ e $g(x)$ hanno la stessa pendenza.
		\end{parts}

	}


	\question

	\exonly{	È data la funzione reale:

		\begin{align*}
			f:\mathbb{R} & \longrightarrow\mathbb{R} \\
			x            & \longmapsto 3x^2-5x-2
		\end{align*}
	}

	\begin{parts}
		\part \exonly{Determinarne le intersezioni con gli assi. }
		\part \exonly{Determinarne gli estremi locali. }
		\part \exonly{Tracciare uno schizzo del grafico utilizzando i dati trovati. }
	\end{parts}
	\exonly{
		Verificare la correttezza dei risultati tracciando il grafico con la calcolatrice, Geogebra o Desmos.
	}


	\exnewpage
	\question
	\exonly{
		È data la funzione reale:

		\begin{align*}
			f:\mathbb{R} & \longrightarrow\mathbb{R} \\
			x            & \longmapsto 2x^3-5x^2-2x
		\end{align*}
	}

	\begin{parts}
		\part \exonly{Determinarne le intersezioni con gli assi. }
		\part \exonly{Determinarne gli estremi locali. }
		\part \exonly{Tracciare uno schizzo del grafico utilizzando i dati trovati. }
	\end{parts}
	\exonly{
		Verificare la correttezza dei risultati tracciando il grafico con la calcolatrice, Geogebra o Desmos.
	}

	\question
	\exonly{
		È data la funzione reale:

		\begin{align*}
			f:\mathbb{R} & \longrightarrow\mathbb{R} \\
			x            & \longmapsto x^4-3x^2
		\end{align*}
	}

	\begin{parts}
		\part \exonly{Determinarne le intersezioni con gli assi. }
		\solonly{$I_x=\left\lbrace (-\sqrt{3};0),(\sqrt{3},0)\right\rbrace $ $I_y(0;0)$ }
		\part \exonly{Determinarne gli estremi locali. }
		\solonly{Min $\in \left\lbrace \left( -\frac{\sqrt{6}}{2};-\frac{9}{4}\right), (0;0),\left( \frac{\sqrt{6}}{2};-\frac{9}{4}\right) \right\rbrace $ Max$\left( 0;0\right) $ }
		\part \exonly{Tracciare uno schizzo del grafico utilizzando i dati trovati. }
	\end{parts}
	\exonly{
		Verificare la correttezza dei risultati tracciando il grafico con la calcolatrice, Geogebra o Desmos.
	}

	\question
	\exonly{
		È data la funzione reale:

		\begin{align*}
			f:\mathbb{R} & \longrightarrow\mathbb{R} \\
			x            & \longmapsto 2x^3+3x
		\end{align*}
	}

	\begin{parts}
		\part \exonly{Determinarne le intersezioni con gli assi. }
		\solonly{$(0;0)$ }
		\part \exonly{Determinarne gli estremi locali. }
		\solonly{$\emptyset$ }
		\part \exonly{Tracciare uno schizzo del grafico utilizzando i dati trovati. }
	\end{parts}

	\exonly{
		Verificare la correttezza dei risultati tracciando il grafico con la calcolatrice, Geogebra o Desmos.
	}


	\question
	\exonly{
		È data la funzione reale:

		\begin{align*}
			f:\mathbb{R} & \longrightarrow\mathbb{R} \\
			x            & \longmapsto (x^2-2x)(x-2)
		\end{align*}

	}

	\begin{parts}
		\part \exonly{ Determinarne le intersezioni con gli assi. }
		\solonly{$I_x=\left\lbrace (0;0),(2,0)\right\rbrace $ $I_y(0;0)$ }
		\part \exonly{ Determinarne gli estremi locali. }

		\solonly{Min $\left( 2;0\right) $ Max$\left( \frac{2}{3};\frac{32}{27}\right) $ }

		\part \exonly{Tracciare uno schizzo del grafico utilizzando i dati trovati. }
	\end{parts}

	\exonly{
		Verificare la correttezza dei risultati tracciando il grafico con la calcolatrice, Geogebra o Desmos.
	}

	\exnewpage
	\question
	\exonly{	Il codominio delle seguenti funzioni è $\mathbb{R}$. Analizzarle (intersezioni con gli assi, massimi e minimi locali, asintoti) e disegnarne qualitativamente il grafico .
	}

	\begin{parts}
		\part \exonly{$f(x)=2x-3$ }
		\solonly{ $I_x(\frac{3}{2},0)$, $I_y(0,-3)$ }
		\part \exonly{$g(x)=3x^2-2x-4$ }
		\solonly{ $I_x=\left\lbrace \frac{1 \pm \sqrt{13}}{3}\right\rbrace $, $I_y(0,-4)$ \newline Min $\left( \frac{1}{3};-\frac{13}{3}\right)$}
		\part \exonly{$h(x)=-2x^3+5x^2-3x$ }
	\end{parts}



	%\vspace*{\stretch{1}}

	\question
	\exonly{	Il codominio delle seguenti funzioni è $\mathbb{R}$. Analizzarle (intersezioni con gli assi, massimi e minimi locali, asintoti) e disegnarne qualitativamente il grafico.
	}

	\begin{parts}
		\setlength\itemsep{3mm}
		\part \exonly{$ f(x)=\dfrac{1}{x} $ }
		\solonly{ Asintoto verticale $x=0$ \newline  Asintoto orizzontale $y=0$ }
		\part \exonly{$ g(x)=\dfrac{x+2}{3x-4} $ }
		\solonly{ Asintoto verticale $x=\frac{4}{3}$ \newline Asintoto orizzontale $y=\frac{1}{3}$ \newline $I_x(-2,0)$, $I_y(0,-\frac{1}{2})$ }
		\part \exonly{$ h(x)=\dfrac{3x-2}{5x-7} $ }
		\solonly{ Asintoto verticale $x=\frac{7}{5}$  \newline Asintoto orizzontale $y=\frac{3}{5}$ \newline $I_x(1,0)$, $I_y(0,\frac{2}{7})$}
	\end{parts}




	%\vspace*{\stretch{1}}


	\question
	\exonly{	Il codominio delle seguenti funzioni è $\mathbb{R}$. Analizzarle (intersezioni con gli assi, massimi e minimi locali, asintoti) e disegnarne qualitativamente il grafico.}

	\begin{parts}
		\setlength\itemsep{3mm}
		\part \exonly{$ f(x)=\dfrac{x^2}{x^2-4} $ }
		\solonly{
			Asintoti verticali $x_1=-2$ e $x_2=2$  \newline
			Asintoto orizzontale $y=1$ \newline
			$I_y(0,3)$ \newline
			Max$(-0.28;4.19)$}

		\part \exonly{$ g(x)=\dfrac{2x^2+3}{4x^2+2x+1} $ }
		\solonly{

			Asintoto orizzontale $y=\dfrac{1}{2}$ \newline
			$I_y(0,3)$}

		\part \exonly{$ h(x)=\dfrac{2x^2+3x-9}{6x^2+5x-6} $ }
		\solonly{
			Asintoti verticali $x_1=-\frac{3}{2}$ e $x_2=\frac{2}{3}$ \newline
			Asintoto orizzontale $y=\frac{1}{3}$ \newline
			$I_x\in\left\lbrace (-3;0),\left(\frac{3}{2};0\right)\right\rbrace $, $I_y\left(0,\frac{3}{2}\right)$ \newline
			Min$(-0.31;1.4)$}
	\end{parts}




	%\vspace*{\stretch{1}}

	\question
	\exonly{	Il codominio delle seguenti funzioni è $\mathbb{R}$. Analizzarle (intersezioni con gli assi, massimi e minimi locali, asintoti) e disegnarne qualitativamente il grafico. }

	\begin{parts}
		\setlength\itemsep{3mm}
		\part \exonly{$ f(x)=\dfrac{2x^2+4}{5x-3} $ }
		\solonly{
			Asintoto verticale $x=\frac{3}{5}$  \newline
			Asintoto obliquo $y=\frac{2}{5}x+\frac{6}{25}$ \newline
			$I_y\left(0;-\frac{4}{3}\right) $ \newline
		}

		\part \exonly{$ g(x)=\dfrac{2x-1}{3x^2+2x-4} $ }

		\solonly{
			Asintoti verticali $x_{1,2}=\frac{-1\pm\sqrt{13}}{3}$  \newline
			Asintoto orizzontale $y=0$ \newline
			$I_x\left(\frac{1}{2};0\right) $ \newline $I_y(0;\frac{1}{4})$
		}
		%	\part \exonly{$ h(x)=\dfrac{4x-2}{x^2+2x+5} $ }
	\end{parts}


	%\vspace*{\stretch{1}}


\end{questions}

\exnewpage
\subsection {Applicazione delle derivate}
\begin{questions}


	\question
	\exonly{
		L'equazione oraria di una palla lanciata verticalmente è:

		\begin{equation*}
			x(t)=-\frac{1}{2}g \cdot t^2 + \SI{3}{\frac{m}{s}} \cdot t+ \SI{1}{m}
		\end{equation*}
	}
	\begin{parts}
		\part
		\exonly{Determinare la velocità in funzione del tempo $v(t)=\dot{x}(t) $. }
		\solonly{Con $g=9.81$ : $v(t)=-\num{9.81}t+3$ }
		\part
		\exonly{ Determinare l'altezza massima raggiunta. }
		\solonly{$x(0.3)=\SI{1.46}{\metre}$}

		\part
		\exonly{ Determinare l'accelerazione in funzione del tempo $a(t)=\ddot{x}(t) $. }
		\solonly{$a(t)=-g \approx 9.81$}
	\end{parts}



	\question

	\exonly{
		Determinare il volume massimo di una piramide a base quadrata di area laterale $\SI{47}{cm^2} $.

	}

	\solonly{Volunme massimo $\num{33.317}\approx \SI{33.2}{\cubic\centi\metre}$ }



	\question
	\exonly{
		Determinare l'area massima che può avere la superficie del rettangolo costruito sotto la curva
		$$y=-2x^2+5x$$

		\begin{tikzpicture}[dot/.style={circle,minimum width=.8pt,inner sep=1pt}, scale=1]
			\begin{axis}[
					enlarge x limits=false,
					enlarge y limits=false,
					axis x line=center,
					axis y line=center,
					xtick={-5,-4.5,...,3.5},
					ytick={-4,-3.5,...,2.5},
					xticklabels={,,},
					yticklabels={,,},
					xlabel={$x$},
					ylabel={$y$},
					xlabel style={below right},
					ylabel style={above left},
					xmin=-1,
					xmax=4,
					ymin=-1,
					ymax=3,
					%		grid=major,
				]

				\addplot[domain=-1:4,samples=201,]{-x^2+3*x};
				\draw[-,thick] (0.9,0) --(0.9,-0.9^2+3*0.9) -- (3-0.9,-0.9^2+3*0.9) -- ((3-0.9,0);
			\end{axis}
		\end{tikzpicture}

	}

	\solonly{Area massima $\num{3.007}\approx 3$ }

	\question
	\exonly{
		Determinare l'area massima che può avere la superficie del rettangolo costruito sotto la curva
		\[ y=-x^2+3x+2 \]
	}

	\solonly{Area massima $\num{6.745}\approx \num{6.75}$ }
	\question
	\exonly{
		Dato un settore circolare di raggio $r=OA=OB$ e angolo $\alpha=\angle AOB<90\degree$.\\
		Determinare un punto $P$  tale per cui la somma delle distanze dai lati del settore $KP+HP$ sia massima.



		\begin{tikzpicture}
			\coordinate[label=below:$O$] (O) at (0,0);
			\coordinate[label=below:$A$] (A) at (5,0);
			\coordinate[label=above left:$B$] (B) at (57:5);
			\coordinate[label=right:$P$] (P) at (34:5);
			\coordinate[label=below:$H$] (H) at ($(O)!(P)!(A)$) ;
			\coordinate[label=left:$K$] (K) at ($(O)!(P)!(B)$) ;
			\draw pic["$\alpha$",draw, angle eccentricity=1.4, blue] {angle=A--O--B};
			\draw pic["$x$",draw, angle eccentricity=1.4, blue,angle radius=1cm] {angle=A--O--P};

			\draw (O) -- (A);
			\draw[] (O) --(B);
			\draw (A) arc (0:57:5);
			\fill (P) circle (1pt);
			\draw (P) -- (H);
			\draw (P) -- (K);
			\draw[dashed] (O)--(P);
		\end{tikzpicture}

	}

	\begin{parts}
		\part\exonly{Descrivere la somma delle distanze  $f(x)=KP+HP$ in funzione di $r$, $\alpha$ e $x$. }
		\solonly{ $f(x)=r\sin(\alpha -x) + r\sin x $}

		\part
		\exonly{Determinare per quale valore di $x$ la distanza $f(x)$ é massima. }
		\solonly{$x=\dfrac{\alpha}{2}$ }


	\end{parts}

	\solonly{Punto medio dell'arco. }

	\question
	\exonly{
		Utilizzare il metodo di Newton per risolvere le seguenti equazioni:

		\begin{parts}
			\part $ 2x^3-4x^2-3x+2=0 $
			\part $ 2x+\cos(x)=3 $
			\part $ 2x^2-x=\sin(2x+4)+3 $
		\end{parts}
	}
\end{questions}

%\setcounter{section}{4} % Sincronizzazione con serie Reto
\exnewpage
%%%%%%%%%%%%%%%%%%%%%%%%%%%%%%%%%%%%%%%%%%%%%%%%%%%%%%%%%%%%%%%%%%%%%%
\subsection{Calcolo integrali indefiniti di base}
%%%%%%%%%%%%%%%%%%%%%%%%%%%%%%%%%%%%%%%%%%%%%%%%%%%%%%%%%%%%%%%%%%%%%%

\begin{questions}

	\question
	\exonly{
		Calcolare i seguenti integrali indefiniti:
	}
	\begin{multicols}{2}

		\begin{parts}
			\part
			\exonly{$\displaystyle\int \left(2x^4-3x^3+2x^2-4x+3 \right) \dif{x} $ }
			\solonly{$\dfrac{2}{5}x^5-\dfrac{3}{4}x^4+\dfrac{2}{3}x^3-2x^2+3x+C$ }
			\part
			\exonly{$ \displaystyle\int \left( cos(2x-4)+3sin(4x+2)\right) \dif{x}= $ }
			\solonly{$\dfrac{\sin(2x-4)}{2}-\dfrac{3\cos(4x+2)}{4}+C$ }

			\part
			\exonly{ $ \displaystyle\int \left( e^{3x-5}+10^{4x-2} \right) \dif{x}= $ }
			\solonly{$\dfrac{e^{3x-5}}{3}+\dfrac{10^{4x-2}}{4\ln(10)}+C$ }
			\part
			\exonly{$ \displaystyle\int \left( x \cdot \cos{(x^2+3)} \right) \dif{x}= $ }
			\solonly{$\dfrac{\sin(x^2+3)}{2}+C$ }
			\part
			\exonly{$ \displaystyle\int \dfrac{2x}{x^2-1} \dif{x}= $ }
			\solonly{$\ln(|x^2-1|)+C$ }
			\part
			\exonly{$ \displaystyle\int \dfrac{4x+2}{x^2+x+3} \dif{x}= $ }
			\solonly{$2\ln(|x^2+x+3|)+C$ }
			\part
			\exonly{$ \displaystyle\int \dfrac{4x}{3x^2-2} \dif{x}= $ }
			\solonly{$\dfrac{2}{3}\ln(|3x^2-2|)+$ }
			\part
			\exonly{ $ \displaystyle\int \dfrac{2x}{\sqrt{x^2-1}} \dif{x}= $ }
			\solonly{$2\sqrt{x^2-1}+C$ }
		\end{parts}
	\end{multicols}
\end{questions}

\subsection{Calcolo aree e lunghezze}
\begin{questions}

	\question
	\exonly{
		Data la funzione {$\mathbb{R} \longrightarrow \mathbb{R}$}:

		\begin{align*}
			f(x) & = -3x^2+5x-1 \\
		\end{align*}

		Determinare l'area della superficie compresa tra il grafico di $f$ e l'asse delle ascisse.

	}

	\solonly{$\int_{0.23}^{1.43} f(x) \dif{x} \approx 0.868$}

	\question
	\exonly{	Sono date le funzioni {$\mathbb{R} \longrightarrow \mathbb{R}$}:

		\begin{align*}
			f(x) & = 3x^2+5x-1 \\
			g(x) & = 2x+5
		\end{align*}

		Determinare l'area della superficie chiusa compresa tra i loro grafici.
	}
	\solonly{ $\dfrac{27}{2}\approx \num{13.5}$}

	\question
	\exonly{	Sono date le funzioni {$\mathbb{R} \longrightarrow \mathbb{R}$}:

		\begin{align*}
			f(x) & = -\frac{3}{2}x+\frac{9}{2} \\
			g(x) & = \frac{3}{x}
		\end{align*}

		Determinare l'area della superficie chiusa compresa tra i loro grafici.
	}
	\solonly{ $\frac{9}{4}-\ln(8)\approx \num{0.1705}$}

	\question
	\exonly{	Sono date le funzioni {$\mathbb{R} \longrightarrow \mathbb{R}$}:

		\begin{align*}
			f(x) & = 3\sin(2x)+1 \\
			g(x) & = 2x^2-2x-2
		\end{align*}

		Determinare l'area della superficie chiusa compresa tra i loro grafici.

		Se necessario utilizzare il metodo di Newton. }

	\solonly{ Area$\approx\num{8.21}$ }

	\question
	\exonly{	Trova il perimetro e  l'area dell'ellisse di equazione:

		\begin{equation*}
			\dfrac{x^2}{9}+y^2=1
		\end{equation*}}
	\solonly{Perimetro$\approx \num{13.36}$ Area$=3\pi$ }

	\question
	\exonly{Determinare la lunghezza del grafico della funzione $f(x)=\dfrac{2}{3}\left(x-1\right)^{3/2}$ con $x\in[1;4]$.
	}
	\solonly{$\dfrac{14}{3}$ }

	\question
	\exonly{	Trova perimetro e area dell'ellisse di equazione (aiutati con la calcolatrice per calcolare l'integrale):

		\begin{equation*}
			2x^2+x+3y^2+2y=5
		\end{equation*} }

	\solonly{ Area $\approx 7$ Perimetro $\approx 9.45$ }

\end{questions}

\subsection{Applicazioni integrale}
\begin{questions}

	\question
	\exonly{	Un veicolo sta viaggiando con una velocità di $\SI{50}{\frac{km}{h}}$ e accelera per $\SI{5}{s}$ con un'accelerazione $a=\SI{1.5}{m\cdot s^{-2}}\cdot t$.

		\begin{parts}
			\part traccia il grafico della velocità $v(t)$ durante la fase di accelerazione
			\part traccia il grafico della posizione $x(t)$ durante la fase di accelerazione
		\end{parts} }

	\solonly{TODO }

	\question
	\exonly{Determina il valore medio assunto dalla funzione

		\begin{align*}
			f: [0;4] & \rightarrow \mathbb{R}                  \\
			x        & \mapsto 2x^4 - 16x^3 + 45x^2 - 54x + 26
		\end{align*}


	}

	\solonly{$f_{medio}=0.95$}
	\question

	\exonly{
		Determinare il valore RMS di una corrente sinusoidale $i(t)=\sin(t)$.


	}

	\solonly{$i_{RMS}=\dfrac{1}{\sqrt{2}}$}




\end{questions}