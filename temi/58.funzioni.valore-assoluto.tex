\section{Valore assoluto}
\subsection{Risoluzione grafica} 
\begin{questions}
	 
	\question
	\exonly{Dato il grafico di $f(x)$ risolvere graficamente $|f(x)|=3$ .
		
		%Last revision: 14.09.2020
%CHANGELOG: assi più visibili, very thick e black
% major grid più visibile black!50 e thick


\begin{tikzpicture}[baseline={($(current bounding box.north)-(0,1.6ex)$)}]
\begin{axis}[AxisDefaults,
major grid style={ thick,black!50},
axis line style={very thick, black},
x=1.5cm,y=1.5cm,
%width=\linewidth,
ytick distance={1},
grid=both,
minor tick num = 9,
ymin=-5,ymax=7,xmin=-4,xmax=5,
domain=-4:5
]
\addplot[draw=blue,thick,smooth,unbounded coords=jump,
restrict y to domain=-6:8]{(x+3)*(x+1)*(x-2)*(x-1)*(x-4)/12.5}; 
\end{axis}
\end{tikzpicture}
	}
	
	\exnewpage
	\question
\exonly{Dato il grafico di $f(x)$ risolvere graficamente $|f(x)-2|=3$ .
	
	%Last revision: 14.09.2020
%CHANGELOG: assi più visibili, very thick e black
% major grid più visibile black!50 e thick


\begin{tikzpicture}[baseline={($(current bounding box.north)-(0,1.6ex)$)}]
\begin{axis}[AxisDefaults,
major grid style={ thick,black!50},
axis line style={very thick, black},
x=1.5cm,y=1.5cm,
%width=\linewidth,
ytick distance={1},
grid=both,
minor tick num = 9,
ymin=-5,ymax=7,xmin=-4,xmax=5,
domain=-4:5
]
\addplot[draw=blue,thick,smooth,unbounded coords=jump,
restrict y to domain=-6:8]{(x+3)*(x+1)*(x-2)*(x-1)*(x-4)/12.5}; 
\end{axis}
\end{tikzpicture}
}
\exnewpage
	\question
\exonly{Dato il grafico di $f(x)$ risolvere graficamente  $|f(x)|<2$ .
	
	%Last revision: 14.09.2020
%CHANGELOG: assi più visibili, very thick e black
% major grid più visibile black!50 e thick


\begin{tikzpicture}[baseline={($(current bounding box.north)-(0,1.6ex)$)}]
\begin{axis}[AxisDefaults,
major grid style={ thick,black!50},
axis line style={very thick, black},
x=1.5cm,y=1.5cm,
%width=\linewidth,
ytick distance={1},
grid=both,
minor tick num = 9,
ymin=-5,ymax=7,xmin=-4,xmax=5,
domain=-4:5
]
\addplot[draw=blue,thick,smooth,unbounded coords=jump,
restrict y to domain=-6:8]{(x+3)*(x+1)*(x-2)*(x-1)*(x-4)/12.5}; 
\end{axis}
\end{tikzpicture}
}

\exnewpage
	\question
\exonly{Dato il grafico di $f(x)$ risolvere graficamente  $|f(x)-2|<1$ .
	
	%Last revision: 14.09.2020
%CHANGELOG: assi più visibili, very thick e black
% major grid più visibile black!50 e thick


\begin{tikzpicture}[baseline={($(current bounding box.north)-(0,1.6ex)$)}]
\begin{axis}[AxisDefaults,
major grid style={ thick,black!50},
axis line style={very thick, black},
x=1.5cm,y=1.5cm,
%width=\linewidth,
ytick distance={1},
grid=both,
minor tick num = 9,
ymin=-5,ymax=7,xmin=-4,xmax=5,
domain=-4:5
]
\addplot[draw=blue,thick,smooth,unbounded coords=jump,
restrict y to domain=-6:8]{(x+3)*(x+1)*(x-2)*(x-1)*(x-4)/12.5}; 
\end{axis}
\end{tikzpicture}
}
	
	\exnewpage
		\question
	\exonly{Dato il grafico di $f(x)$ risolvere graficamente  $|f(x)-1|<4$ .
		
		%Last revision: 14.09.2020
%CHANGELOG: assi più visibili, very thick e black
% major grid più visibile black!50 e thick


\begin{tikzpicture}[baseline={($(current bounding box.north)-(0,1.6ex)$)}]
\begin{axis}[AxisDefaults,
major grid style={ thick,black!50},
axis line style={very thick, black},
x=1.5cm,y=1.5cm,
%width=\linewidth,
ytick distance={1},
grid=both,
minor tick num = 9,
ymin=-5,ymax=7,xmin=-4,xmax=5,
domain=-4:5
]
\addplot[draw=blue,thick,smooth,unbounded coords=jump,
restrict y to domain=-6:8]{(x+3)*(x+1)*(x-2)*(x-1)*(x-4)/12.5}; 
\end{axis}
\end{tikzpicture}
	}
	
	\exnewpage
		\question
	\exonly{Dato il grafico di $f(x)$ risolvere graficamente  $|f(x)-4|>1$ .
		
		%Last revision: 14.09.2020
%CHANGELOG: assi più visibili, very thick e black
% major grid più visibile black!50 e thick


\begin{tikzpicture}[baseline={($(current bounding box.north)-(0,1.6ex)$)}]
\begin{axis}[AxisDefaults,
major grid style={ thick,black!50},
axis line style={very thick, black},
x=1.5cm,y=1.5cm,
%width=\linewidth,
ytick distance={1},
grid=both,
minor tick num = 9,
ymin=-5,ymax=7,xmin=-4,xmax=5,
domain=-4:5
]
\addplot[draw=blue,thick,smooth,unbounded coords=jump,
restrict y to domain=-6:8]{(x+3)*(x+1)*(x-2)*(x-1)*(x-4)/12.5}; 
\end{axis}
\end{tikzpicture}
	}
	\exnewpage
		\question
	\exonly{Dato il grafico di $f(x)$ risolvere graficamente  $|f(x)+1|\geq 1$ .
		
		%Last revision: 14.09.2020
%CHANGELOG: assi più visibili, very thick e black
% major grid più visibile black!50 e thick


\begin{tikzpicture}[baseline={($(current bounding box.north)-(0,1.6ex)$)}]
\begin{axis}[AxisDefaults,
major grid style={ thick,black!50},
axis line style={very thick, black},
x=1.5cm,y=1.5cm,
%width=\linewidth,
ytick distance={1},
grid=both,
minor tick num = 9,
ymin=-5,ymax=7,xmin=-4,xmax=5,
domain=-4:5
]
\addplot[draw=blue,thick,smooth,unbounded coords=jump,
restrict y to domain=-6:8]{(x+3)*(x+1)*(x-2)*(x-1)*(x-4)/12.5}; 
\end{axis}
\end{tikzpicture}
	}
	
	\exnewpage
		\question
	\exonly{Dato il grafico di $f(x)$ risolvere graficamente  $|f(x)-x|<1$ .
		
		%Last revision: 14.09.2020
%CHANGELOG: assi più visibili, very thick e black
% major grid più visibile black!50 e thick


\begin{tikzpicture}[baseline={($(current bounding box.north)-(0,1.6ex)$)}]
\begin{axis}[AxisDefaults,
major grid style={ thick,black!50},
axis line style={very thick, black},
x=1.5cm,y=1.5cm,
%width=\linewidth,
ytick distance={1},
grid=both,
minor tick num = 9,
ymin=-5,ymax=7,xmin=-4,xmax=5,
domain=-4:5
]
\addplot[draw=blue,thick,smooth,unbounded coords=jump,
restrict y to domain=-6:8]{(x+3)*(x+1)*(x-2)*(x-1)*(x-4)/12.5}; 
\end{axis}
\end{tikzpicture}
	}
	
\end{questions}

\exnewpage
\subsection{Equazioni}
\begin{questions}
\question
\exonly{Risolvere le seguenti equazioni: }

\begin{parts}
	\part
	\exonly{$|2x-5|=-10 $}
	\solonly{$S=\emptyset$ }
	
	\part
	\exonly{$2|4x-2|=10 $}
	\solonly{$\es{-\frac{3}{4},\frac{7}{4}}$ }
	
	\part
	\exonly{$|x+2|=-10 $}
	\solonly{$S=\emptyset$ }
	
	\part
	\exonly{$|x^2-1|=16 $}
	\solonly{$\es{-\sqrt{17},\sqrt{17}}$ }

	\part
	\exonly{$(x-5)|x-7|=0 $}
	\solonly{$\es{5,7}$ }
	\part
	\exonly{$|x^2-x|-6=0 $}
	\solonly{$\es{-2,3}$ }
\end{parts}
	
\question
\exonly{Risolvere le seguenti equazioni: }

\begin{parts}
	\part
	\exonly{$-3|2x-5|=-18 $}
	\solonly{$\es{-\frac{1}{2},\frac{11}{2}}$} 

	\part
	\exonly{$|x-7|=x+3$ }
	\solonly{$\es{2}$ }
	
	\part
	\exonly{$2|x-3|+3x=5x-8$ }
	\solonly{$S=\emptyset$ }
	
	\part
	\exonly{$ \dfrac{x+1}{2}=\dfrac{1}{|x-2|}$ }
	\solonly{$\es{0,1,\frac{\sqrt{17}+1}{2}}$ }
	
	\part
	\exonly{$|x^2-9|=-4x-4 $ }
	\solonly{$\es{-5,2-\sqrt{17}}$ }
	
	\part
	\exonly{$|3+2|x-2||=5 $ }
	\solonly{$\es{1,3}$ }
	
		
	\part
	\exonly{$|3-2|x-2||=5 $ }
	\solonly{$\es{-2,6}$ }

\end{parts}	
	



\question
\exonly{Risolvere le seguenti equazioni e verificare con il \textbf{CAS} }
\begin{parts}
	\part
	\exonly{$|x+1|=|2x-4| $}
	\solonly{$\es{1,5} $}
	
	\part
\exonly{$|x+1|-2=|2x-4| $}
	\solonly{$\es{\dfrac{5}{3},3} $}
	
		\part
	\exonly{$|x-5|-2=|2x-4| $}
	\solonly{$\es{\dfrac{7}{3},1} $}
	
		\part
	\exonly{$|3-2|x-2||=2 $ }
	\solonly{$\es{-\frac{1}{2},\frac{3}{2},\frac{5}{2},\frac{9}{2}}$ }
	
\end{parts}

\question
\exonly{\textbf{CAS} Risolvere le seguenti equazioni}
\begin{parts}
	\part
	\exonly{$\dfrac{3+|x^2-1|}{3|x-6|}=x-1 $}
	\solonly{$\es{\frac{5}{4};4;\frac{21+\sqrt{313}}{4}}$ }
	\part
	\exonly{$\sqrt{|x-3|-4}=|x-3|-4 $}
	\part
	\exonly{$ \sqrt{x+6}-|2-x|=x-14$ }
	\part 
	\exonly{$\left(x^3 -\sqrt[3]{x+6} \right)^2=x^2+2$}
	\part
	\exonly{$\sqrt{|2x-4|-4}=\dfrac{|2x-4|}{5}$ }

	
\end{parts}

\exonly{\newpage}
\question 
\exonly{Il rimepimento di pacchetti di zucchero da \SI{500}{\gram} è automatizzato. Una prima macchina riempie i pacchetti che vengono successivamente pesati da una seconda macchina. Questa scarta i pacchetti che non rispettano la tolleranza massima di \SI{5}{\gram} imposta sui pacchetti.

Esprimere, con un unico vincolo, la condizione che la seconda macchina deve testare per scartare il pacchetto pesato. 

Risolvere e indicare l'intervallo entro il quale la macchina scarta il pacchetto.
}

\solonly{
	Sia $x$ la massa (in grammi) del pacchetto pesato.

$|x-500|> 5$

La macchina scarta pacchetti nell'intervallo $[0;455[ \cup ]505,\infty[$

}

\question 
\exonly{
Una macchina costruisce barre di metallo di un diametro di \SI{8}{\milli\metre} con un errore dichiarato di massimo \SI{0.5}{\milli\metre}.

Esprimere, con un solo vincolo, la condizione da testare per verificare che la barra di metallo si nei limiti tollerati. Indicare l'intervallo tollerato.


}

\solonly{
Sia $x$ il diametro della barra.

$|x-8|\leq 0.5$

Intervallo tollerato $[7.5;8.5]$ \si[]{\milli\metre}

}

\question 
\exonly{
	\textbf{CAS} Per un angolo in radianti, la funzione $f(x)=\sin$ può essere approssimata da $g(x)=x$. Se si tollera un errore massimo di \num[]{0.01}, per quale intervallo è possibile usare l'approssimazione? Esprimere l'intervallo in radianti e in gradi.

}

\solonly{
$|\sin(x)-x|\leq 0.01$

In radianti: $S=[-0.392;0.392]$

In gradi: $S=[-22.46;22.46]$


}




\end{questions}