\section{Valore assoluto}
\subsection{Risoluzione grafica}
\begin{questions}

	\begin{qblock}
		\question
		\exonly{Dato il grafico di $f$ risolvere graficamente $|f(x)|=3$ .

			%Last revision: 14.09.2020
%CHANGELOG: assi più visibili, very thick e black
% major grid più visibile black!50 e thick


\begin{tikzpicture}[baseline={($(current bounding box.north)-(0,1.6ex)$)}]
\begin{axis}[AxisDefaults,
major grid style={ thick,black!50},
axis line style={very thick, black},
x=1.5cm,y=1.5cm,
%width=\linewidth,
ytick distance={1},
grid=both,
minor tick num = 9,
ymin=-5,ymax=7,xmin=-4,xmax=5,
domain=-4:5
]
\addplot[draw=blue,thick,smooth,unbounded coords=jump,
restrict y to domain=-6:8]{(x+3)*(x+1)*(x-2)*(x-1)*(x-4)/12.5}; 
\end{axis}
\end{tikzpicture}
		}
		\solonly{ ~}
	\end{qblock}

	\begin{qblock}
		\question
		\exonly{Dato il grafico di $f$ risolvere graficamente $|f(x)-2|=3$ .

			%Last revision: 14.09.2020
%CHANGELOG: assi più visibili, very thick e black
% major grid più visibile black!50 e thick


\begin{tikzpicture}[baseline={($(current bounding box.north)-(0,1.6ex)$)}]
\begin{axis}[AxisDefaults,
major grid style={ thick,black!50},
axis line style={very thick, black},
x=1.5cm,y=1.5cm,
%width=\linewidth,
ytick distance={1},
grid=both,
minor tick num = 9,
ymin=-5,ymax=7,xmin=-4,xmax=5,
domain=-4:5
]
\addplot[draw=blue,thick,smooth,unbounded coords=jump,
restrict y to domain=-6:8]{(x+3)*(x+1)*(x-2)*(x-1)*(x-4)/12.5}; 
\end{axis}
\end{tikzpicture}
		}

		\solonly{ ~}

	\end{qblock}


	\begin{qblock}
		\question
		\exonly{Dato il grafico di $f$ risolvere graficamente  $|f(x)|<2$ .

			%Last revision: 14.09.2020
%CHANGELOG: assi più visibili, very thick e black
% major grid più visibile black!50 e thick


\begin{tikzpicture}[baseline={($(current bounding box.north)-(0,1.6ex)$)}]
\begin{axis}[AxisDefaults,
major grid style={ thick,black!50},
axis line style={very thick, black},
x=1.5cm,y=1.5cm,
%width=\linewidth,
ytick distance={1},
grid=both,
minor tick num = 9,
ymin=-5,ymax=7,xmin=-4,xmax=5,
domain=-4:5
]
\addplot[draw=blue,thick,smooth,unbounded coords=jump,
restrict y to domain=-6:8]{(x+3)*(x+1)*(x-2)*(x-1)*(x-4)/12.5}; 
\end{axis}
\end{tikzpicture}
		}
		\solonly{ ~}
	\end{qblock}


	\begin{qblock}
		\question
		\exonly{Dato il grafico di $f$ risolvere graficamente  $|f(x)-2|<1$ .

			%Last revision: 14.09.2020
%CHANGELOG: assi più visibili, very thick e black
% major grid più visibile black!50 e thick


\begin{tikzpicture}[baseline={($(current bounding box.north)-(0,1.6ex)$)}]
\begin{axis}[AxisDefaults,
major grid style={ thick,black!50},
axis line style={very thick, black},
x=1.5cm,y=1.5cm,
%width=\linewidth,
ytick distance={1},
grid=both,
minor tick num = 9,
ymin=-5,ymax=7,xmin=-4,xmax=5,
domain=-4:5
]
\addplot[draw=blue,thick,smooth,unbounded coords=jump,
restrict y to domain=-6:8]{(x+3)*(x+1)*(x-2)*(x-1)*(x-4)/12.5}; 
\end{axis}
\end{tikzpicture}
		}
		\solonly{ ~}
	\end{qblock}


	\begin{qblock}
		\question
		\exonly{Dato il grafico di $f$ risolvere graficamente  $|f(x)-1|<4$ .

			%Last revision: 14.09.2020
%CHANGELOG: assi più visibili, very thick e black
% major grid più visibile black!50 e thick


\begin{tikzpicture}[baseline={($(current bounding box.north)-(0,1.6ex)$)}]
\begin{axis}[AxisDefaults,
major grid style={ thick,black!50},
axis line style={very thick, black},
x=1.5cm,y=1.5cm,
%width=\linewidth,
ytick distance={1},
grid=both,
minor tick num = 9,
ymin=-5,ymax=7,xmin=-4,xmax=5,
domain=-4:5
]
\addplot[draw=blue,thick,smooth,unbounded coords=jump,
restrict y to domain=-6:8]{(x+3)*(x+1)*(x-2)*(x-1)*(x-4)/12.5}; 
\end{axis}
\end{tikzpicture}
		}
		\solonly{ ~}
	\end{qblock}


	\begin{qblock}
		\question
		\exonly{Dato il grafico di $f$ risolvere graficamente  $|f(x)-4|>1$ .

			%Last revision: 14.09.2020
%CHANGELOG: assi più visibili, very thick e black
% major grid più visibile black!50 e thick


\begin{tikzpicture}[baseline={($(current bounding box.north)-(0,1.6ex)$)}]
\begin{axis}[AxisDefaults,
major grid style={ thick,black!50},
axis line style={very thick, black},
x=1.5cm,y=1.5cm,
%width=\linewidth,
ytick distance={1},
grid=both,
minor tick num = 9,
ymin=-5,ymax=7,xmin=-4,xmax=5,
domain=-4:5
]
\addplot[draw=blue,thick,smooth,unbounded coords=jump,
restrict y to domain=-6:8]{(x+3)*(x+1)*(x-2)*(x-1)*(x-4)/12.5}; 
\end{axis}
\end{tikzpicture}
		}
		\solonly{ ~}
	\end{qblock}


	\begin{qblock}
		\question
		\exonly{Dato il grafico di $f$ risolvere graficamente  $|f(x)+1|\geq 1$ .

			%Last revision: 14.09.2020
%CHANGELOG: assi più visibili, very thick e black
% major grid più visibile black!50 e thick


\begin{tikzpicture}[baseline={($(current bounding box.north)-(0,1.6ex)$)}]
\begin{axis}[AxisDefaults,
major grid style={ thick,black!50},
axis line style={very thick, black},
x=1.5cm,y=1.5cm,
%width=\linewidth,
ytick distance={1},
grid=both,
minor tick num = 9,
ymin=-5,ymax=7,xmin=-4,xmax=5,
domain=-4:5
]
\addplot[draw=blue,thick,smooth,unbounded coords=jump,
restrict y to domain=-6:8]{(x+3)*(x+1)*(x-2)*(x-1)*(x-4)/12.5}; 
\end{axis}
\end{tikzpicture}
		}
		\solonly{ ~}
	\end{qblock}


	\begin{qblock}
		\question
		\exonly{Dato il grafico di $f$ risolvere graficamente  $|f(x)-x|<1$ .

			%Last revision: 14.09.2020
%CHANGELOG: assi più visibili, very thick e black
% major grid più visibile black!50 e thick


\begin{tikzpicture}[baseline={($(current bounding box.north)-(0,1.6ex)$)}]
\begin{axis}[AxisDefaults,
major grid style={ thick,black!50},
axis line style={very thick, black},
x=1.5cm,y=1.5cm,
%width=\linewidth,
ytick distance={1},
grid=both,
minor tick num = 9,
ymin=-5,ymax=7,xmin=-4,xmax=5,
domain=-4:5
]
\addplot[draw=blue,thick,smooth,unbounded coords=jump,
restrict y to domain=-6:8]{(x+3)*(x+1)*(x-2)*(x-1)*(x-4)/12.5}; 
\end{axis}
\end{tikzpicture}
		}

		\solonly{ ~}
	\end{qblock}


	\begin{qblock}
		\question
		\exonly{Dato il grafico di $f(x)=x^3$ risolvere graficamente  $|4x^3-4|<2$ .

			\begin{tikzpicture}[baseline={($(current bounding box.north)-(0,1.6ex)$)}]
				\begin{axis}[AxisDefaults,
						major grid style={ thick,black!50},
						axis line style={very thick, black},
						x=1cm,y=1cm,
						%width=\linewidth,
						ytick distance={1},
						grid=both,
						minor tick num = 9,
						ymin=-5,ymax=7,xmin=-4,xmax=5,
						domain=-4:5
					]
					%\addplot[draw=blue,thick,restrict y to domain=-6:8]{x**3}; 
					\addplot {x^3};
				\end{axis}
			\end{tikzpicture}
		}
		\solonly{ ~}
	\end{qblock}


	\begin{qblock}
		\question
		\exonly{Dato il grafico di $f(x)=x^3$ risolvere graficamente  $|-2x^3+4|<2$ .

			\begin{tikzpicture}[baseline={($(current bounding box.north)-(0,1.6ex)$)}]
				\begin{axis}[AxisDefaults,
						major grid style={ thick,black!50},
						axis line style={very thick, black},
						x=1cm,y=1cm,
						%width=\linewidth,
						ytick distance={1},
						grid=both,
						minor tick num = 9,
						ymin=-5,ymax=7,xmin=-4,xmax=5,
						domain=-4:5
					]
					%\addplot[draw=blue,thick,restrict y to domain=-6:8]{x**3}; 
					\addplot {x^3};
				\end{axis}
			\end{tikzpicture}
		}

		\solonly{ ~}
	\end{qblock}


	\begin{qblock}
		\question
		\exonly{Dato il grafico di $f(x)=x^3$ risolvere graficamente  $|x|^3-2x-4=0$ .

			\begin{tikzpicture}[baseline={($(current bounding box.north)-(0,1.6ex)$)}]
				\begin{axis}[AxisDefaults,
						major grid style={ thick,black!50},
						axis line style={very thick, black},
						x=1cm,y=1cm,
						%width=\linewidth,
						ytick distance={1},
						grid=both,
						minor tick num = 9,
						ymin=-5,ymax=9,xmin=-4,xmax=5,
						domain=-4:5
					]
					%\addplot[draw=blue,thick,restrict y to domain=-6:8]{x**3}; 
					\addplot {x^3};
				\end{axis}
			\end{tikzpicture}
		}

		\solonly{ ~}
	\end{qblock}

\end{questions}

\exnewpage
\subsection{Equazioni}
\begin{questions}

	\begin{qblock}
		\question
		\exonly{Risolvere le seguenti equazioni: }

		\begin{parts}
			\part
			\exonly{$|2x-5|=-10 $}
			\solonly{$S=\emptyset$ }

			\part
			\exonly{$2|4x-2|=10 $}
			\solonly{$\es{-\frac{3}{4},\frac{7}{4}}$ }

			\part
			\exonly{$|x+2|=-10 $}
			\solonly{$S=\emptyset$ }

			\part
			\exonly{$|x^2-1|=16 $}
			\solonly{$\es{-\sqrt{17},\sqrt{17}}$ }

			\part
			\exonly{$(x-5)|x-7|=0 $}
			\solonly{$\es{5,7}$ }
			\part
			\exonly{$|x^2-x|-6=0 $}
			\solonly{$\es{-2,3}$ }
		\end{parts}
	\end{qblock}


	\begin{qblock}
		\question
		\exonly{Risolvere le seguenti equazioni: }

		\begin{parts}
			\part
			\exonly{$-3|2x-5|=-18 $}
			\solonly{$\es{-\frac{1}{2},\frac{11}{2}}$}

			\part
			\exonly{$|x-7|=x+3$ }
			\solonly{$\es{2}$ }

			\part
			\exonly{$2|x-3|+3x=5x-8$ }
			\solonly{$S=\emptyset$ }

			\part
			\exonly{$ \dfrac{x+1}{2}=\dfrac{1}{|x-2|}$ }
			\solonly{$\es{0,1,\frac{\sqrt{17}+1}{2}}$ }

			\part
			\exonly{$|x^2-9|=-4x-4 $ }
			\solonly{$\es{-5,2-\sqrt{17}}$ }

			\part
			\exonly{$|3+2|x-2||=5 $ }
			\solonly{$\es{1,3}$ }


			\part
			\exonly{$|3-2|x-2||=5 $ }
			\solonly{$\es{-2,6}$ }

		\end{parts}
	\end{qblock}


	\begin{qblock}
		\question
		\exonly{Risolvere le seguenti equazioni e verificare con il \textbf{CAS} }
		\begin{parts}
			\part
			\exonly{$|x+1|=|2x-4| $}
			\solonly{$\es{1,5} $}

			\part
			\exonly{$|x+1|-2=|2x-4| $}
			\solonly{$\es{\dfrac{5}{3},3} $}

			\part
			\exonly{$|x-5|-2=|2x-4| $}
			\solonly{$\es{\dfrac{7}{3},1} $}

			\part
			\exonly{$|3-2|x-2||=2 $ }
			\solonly{$\es{-\frac{1}{2},\frac{3}{2},\frac{5}{2},\frac{9}{2}}$ }

		\end{parts}
	\end{qblock}


	\begin{qblock}
		\question
		\exonly{\textbf{CAS} Risolvere le seguenti equazioni}
		\begin{parts}
			\part
			\exonly{$\dfrac{3+|x^2-1|}{3|x-6|}=x-1 $}
			\solonly{$\es{\frac{5}{4};4;\frac{21+\sqrt{313}}{4}}$ }
			\part
			\exonly{$\sqrt{|x-3|-4}=|x-3|-4 $}
			\part
			\exonly{$ \sqrt{x+6}-|2-x|=x-14$ }
			\part
			\exonly{$\left(x^3 -\sqrt[3]{x+6} \right)^2=x^2+2$}
			\part
			\exonly{$\sqrt{|2x-4|-4}=\dfrac{|2x-4|}{5}$ }


		\end{parts}
	\end{qblock}

\end{questions}
\subsection{Applicazioni}

\begin{questions}


	\begin{qblock}
		\question

		\exonly{
			Nel piano cartesiano sottostante sono rappresentati i grafici della funzione $f$ e $g$ con leggi:
			\[
				f(x)=|2x-1| \qquad g(x)=|x+2|
			\]

			\begin{tikzpicture}[baseline={($(current bounding box.north)-(0,1.6ex)$)}]
				\begin{axis}[AxisDefaults,
						width=12cm,
						ytick distance={1},
						ymax=7,
						ymin=-1
					]
					\fill[ color=gray!10] (-1/3,0) -- (-1/3,5/3) -- (3,5) -- (3,0)-- (-1/3,0) ;
					\draw[dashed] (-1/3,0) -- (-1/3,5/3) (3,0)-- (3,5);
					\draw[] (-1/3,0) --(3,0);

					\addplot[draw=red,
						domain=-5:5
					]{abs(x+2)} node[pos=0.1,above] {$g$};
					\addplot[draw=blue,
						domain=-5:5,restrict y to domain=-1:8,
					]{abs(2*x-1)} node[pos=0.2,above right] {$f$};
					\addplot [only marks, mark=o] (-1/3,5/3) node[left] {$C$} ;
					\addplot [only marks, mark=o] (3,5) node[right] {$D$} ;
					\addplot [only marks, mark=o] (-1/3,0) node[above left,xshift=0.2cm] {$A(x_A;y_A)$} ;
					\addplot [only marks, mark=o] (3,0) node[above right] {$B(x_B;y_B)$} ;

				\end{axis}
			\end{tikzpicture}
		}

		\begin{parts}
			\part
			\exonly{Quale vincolo tra $f$ e $g$ ha come soluzione l'intervallo  $[x_A;x_B]$ ?}

			\solonly{
				$f(x)\geq g(x)$

			}

			\part
			\exonly{
				Calcolare le coordinate dei punti $A$, $B$, $C$ e $D$
			}

			\solonly{
				$A(-\frac{1}{3};0) \quad B(3;0) \quad C(-\frac{1}{3};\frac{3}{5}) \quad D(3;5)$
			}
		\end{parts}
	\end{qblock}



	\begin{qblock}
		\question
		\exonly{


			Il riempimento di pacchetti di zucchero da \SI{500}{\gram} è automatizzato. Una prima macchina riempie i pacchetti che vengono successivamente pesati da una seconda macchina. Questa scarta i pacchetti che non rispettano la tolleranza massima di \SI{5}{\gram} imposta sui pacchetti.

			Esprimere, con un unico vincolo, la condizione che la seconda macchina deve testare per scartare i pacchetti fuori tolleranza.

			Risolvere e indicare l'intervallo entro il quale la macchina scarta il pacchetto.
		}

		\solonly{
		Sia $x$ la massa (in grammi) del pacchetto pesato.

		$|x-500|> 5$

		La macchina scarta pacchetti nell'intervallo $[0;455[ \cup ]505,\infty[$

		}
	\end{qblock}


	\begin{qblock}
		\question
		\exonly{
			Una macchina costruisce barre di metallo di un diametro di \SI{8}{\milli\metre} con un errore dichiarato di massimo \SI{0.1}{\milli\metre}.

			Esprimere, con un solo vincolo, la condizione da testare per verificare che la barra di metallo sia nei limiti tollerati. Indicare l'intervallo tollerato.


		}

		\solonly{
			Sia $x$ il diametro della barra.

			$|x-8|\leq 0.1$

			Intervallo tollerato $[7.9;8.1]$ \si[]{\milli\metre}

		}
	\end{qblock}


	\begin{qblock}
		\question
		\exonly{
			\textbf{CAS} Per un angolo in radianti, la funzione $f(x)=\sin(x)$ può essere approssimata da $g(x)=x$. Se si tollera un errore massimo di \num[]{0.01}, per quale intervallo è possibile usare l'approssimazione? Esprimere l'intervallo in radianti e in gradi.

		}

		\solonly{
		$|\sin(x)-x|\leq 0.01$

		In radianti: $S=[-0.392;0.392]$

		%In gradi: $S=[-22.46;22.46]$
		}
	\end{qblock}

\end{questions}