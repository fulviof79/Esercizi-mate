\section{Esponenziali e logaritmi}

\subsection{Richiamo regole con le potenze}
\begin{questions}


	\begin{qblock}
		\question
		\exonly{
			Semplificare:
		}

		\begin{parts}
			\part
			\exonly{
			$(5x^2y^{-3})(4x^{-5}y^4)$
			}
			\solonly{$\dfrac{20y}{x^3}$\\
				\begin{profonly}Src: Swok pg 41 ex 29	 		\end{profonly}}

			\part
			\exonly{$\left(\dfrac{1}{3}x^4y^{-3}\right)^{-2}$ }
			\solonly{
				$\dfrac{9y^6}{x^8}$
				\begin{profonly}Src: Swok pg 41 ex 23	 		\end{profonly}}

			\part
			\exonly{$\left(\dfrac{3x^5y^4}{x^0y^{-3}}\right)^2$ }
			\solonly{
			$9x^{10}y^{14}$
			\begin{profonly}Src: Swok pg 41 ex 31	 		\end{profonly}}

			\part
			\exonly{$\left(\dfrac{-8x^3}{y^{-6}} \right)^{\sfrac{2}{3}}$ }
			\solonly{
				$4x^2y^4$
				\begin{profonly}Src: Swok pg 41 ex 41	 		\end{profonly}}

		\end{parts}
	\end{qblock}

\end{questions}


\subsection{Funzioni esponenziali}

\begin{questions}


	\begin{qblock}
		\question
		\exonly{
			Rappresentare  graficamente le funzioni seguenti:
		}
		\exonly{
			\begin{parts}
				\part \label{ex:1}
				\exonly{$2^{x}$ }


				\begin{profonly}Src: Swok pg 317 ex 9 a	\end{profonly}

				\part \label{ex:2}
				\exonly{$-2^{x}$ }

				\begin{profonly}Src: Swok pg 317 ex 9 b	\end{profonly}


				\part \label{ex:3}
				\exonly{$2^{3-x}$ }

				\begin{profonly}Src: Swok pg 317 ex 9 j	\end{profonly}



			\end{parts}}

		\solonly{Rappresentazioni grafiche:}

		\ifprintanswers
			\begin{tikzpicture}[baseline={($(current bounding box.north)-(0,1.6ex)$)}]
				\begin{axis}[
						AxisDefaults,
						SmallAxisLabels,
						width=\linewidth,
					]
					\addplot[draw=red,smooth,unbounded coords=jump,restrict y to domain=0:5]{2^x} node[right, pos=0.9] {$(a)$};
					\addplot[draw=red,smooth,unbounded coords=jump,restrict y to domain=-5:5]{-2^x}
					node[below right, pos=0.9] {$(b)$};
					\addplot[draw=red,smooth,unbounded coords=jump,restrict y to domain=0:5]{2^(3-x)}
					node[above, pos=0.9] {$(c)$};
				\end{axis}
			\end{tikzpicture}
		\fi
	\end{qblock}




	
	\begin{qblock}
		\question
		\exonly{Siano $a>b>1>c>0$. Rappresentare graficamente le funzioni seguenti:
		$f_1(x)=a^x$, $f_2(x)=b^x$, $f_3(x)=c^x$, $f_4(x)=c^{x+2}$ e $f_5(x)=b^x-2$. }
	
		\solonly{Esempio con  $a=3$, $b=2$ e $c=\frac{1}{2}$:}
	
		\ifprintanswers
			\begin{tikzpicture}[baseline={($(current bounding box.north)-(0,1.6ex)$)}]
				\begin{axis}[
						AxisDefaults,
						SmallAxisLabels,
						width=\linewidth,
					]
					\addplot[draw=red,smooth,unbounded coords=jump,restrict y to domain=0:5]{3^x}
					node[ left, pos=0.9] {$f_1$};
					\addplot[draw=red,smooth,unbounded coords=jump,restrict y to domain=-5:5]{2^x}
					node[below right, pos=0.9] {$f_2$};
					\addplot[draw=red,smooth,unbounded coords=jump,restrict y to domain=0:5]{(0.5)^x}
					node[above, pos=0.9] {$f_3$};
					\addplot[draw=red,smooth,unbounded coords=jump,restrict y to domain=0:5]{(0.5)^(x+2)}
					node[above, pos=0.2] {$f_4$};
					\addplot[draw=red,smooth,unbounded coords=jump,restrict y to domain=-5:5]{2^(x)-2}
					node[below right, pos=0.9] {$f_5$};
				\end{axis}
			\end{tikzpicture}
		\fi
	\end{qblock}




	
	\begin{qblock}
		\question
	
		\exonly{
			Secondo la legge di raffreddamento di Newton, la velocità di raffreddamento é proporzionale alla differenza di temperature tra l'oggetto e l'ambiente in cui si trova.
	
			Un ferro da stiro passa da \SI{52}{\celsius} a \SI{38}{\degreeCelsius} in \SI{30}{\minute} in una stanza ad una temperatura  constante di \SI{24}{\celsius}.
	
			La temperatura $T$ del ferro dopo  $t$ ore é modellizzata da
			$$ T(t)=28\cdot 2^{-2t}+24$$
		}
	
		\begin{parts}
	
			\part \exonly{
				Indicare il ruolo di ciascun parametro del modello e da quali dati sperimentali è stato ricavato.
			}
	
			\part
			\exonly{Eseguire uno schizzo qualitativo del grafico della funzione $T(t)$}
	
	
		\end{parts}
	\end{qblock}

	
	\begin{qblock}
		\question
		\exonly{
			Il numero di batteri di una cultura passa da  $600$ a $1800$ tra le  $7h$ e le  $9h$ del mattino.
			Supponendo una crescita esponenziale, la popolazione di batteri   $t$ ore dopo le  $7h$ del mattino é data dal modello
	
			\[P(t)=600 \cdot 3^{\sfrac{t}{2}}\]
		}
	
	
	
	
		\begin{parts}
			\part \exonly{Indicare il ruolo di ciascun parametro del modello e da quali dati sperimentali è stato ricavato.}
	
	
			\part
			\exonly{Eseguire uno schizzo qualitativo del grafico della funzione $P(t)$}
	
	
		\end{parts}
	
	
		\begin{profonly}
			Swokowski pg 318 ex 15
			Swokowski pg 317 ex 17
			Swokowski pg 317 ex 19
		\end{profonly}
	\end{qblock}



\begin{qblock}
		\question
		\exonly{
			Determinare il modello $f$ che rappresenti al meglio i dati rappresentati qui sotto.
	
	
	
			\begin{center}
				\begin{tikzpicture}
					\begin{axis}[
							AxisDefaults,
							SmallAxisLabels,
							height=10cm, %height and width of the graph
							width=0.8\linewidth,
							restrict y to domain= 0:11,
							ymin=-1, ymax=10, %set the min and max y tick
							xmin=-3
						]
	
						\addplot[draw=red,  only marks] coordinates {
								(-3.00,24.00) (-2.00,12.00) (-1.00,6.00) (0.00,3.00) (1.00,1.50) (2.00,0.75) (3.00,0.38) (4.00,0.19) (5.00,0.09) (6.00,0.05) (7.00,0.02) (8.00,0.01) (9.00,0.01)
							};
						\addplot[mark=none] coordinates {(-1,6)} node[pin=170:{$(-1;6)$}]{} ;
						\addplot[mark=none] coordinates {(0,3)} node[pin=10:{$(0;3)$}]{} ;
					\end{axis}
				\end{tikzpicture}
			\end{center}
	
	
	
		}
	
		\solonly{
			$f(x)=3\cdot 2^{-x}=3\cdot \left(\dfrac{1}{2}\right)^x$
	
	
		}
\end{qblock}

	
	\begin{qblock}
		\question
		\exonly{
			Determinare il modello $f$ che rappresenti al meglio i dati rappresentati qui sotto.
	
	
	
			\begin{center}
				\begin{tikzpicture}
					\begin{axis}[
							AxisDefaults,
							SmallAxisLabels,
							height=10cm, %height and width of the graph
							width=0.8\linewidth,
							restrict y to domain= 0:30,
							ymin=-1, ymax=30, %set the min and max y tick
							xmin=-3,
							ytick distance=5,
						]
	
						\addplot[draw=red,  only marks] coordinates {
								(-3.00,0.05) (-2.00,0.09) (-1.00,0.19) (0.00,0.38) (1.00,0.75) (2.00,1.50) (3.00,3.00) (4.00,6.00) (5.00,12.00) (6.00,24.00) (7.00,48.00) (8.00,96.00) (9.00,192.00)
							};
						\addplot[mark=none] coordinates {(3,3)} node[pin=160:{$(3;3)$}]{} ;
						\addplot[mark=none] coordinates {(5,12)} node[pin=110:{$(5;12)$}]{} ;
					\end{axis}
				\end{tikzpicture}
			\end{center}
	
		}
	
		\solonly{
	
			$f(x)=3\cdot 4^{\frac{x-3}{2}}=3\cdot 2^{x-3}$
		}
	\end{qblock}

\end{questions}


\subsection{Equazioni esponenziali - introduzione}

\begin{questions}
	
	
	\begin{qblock}
		\question
		\exonly{Risolvere le seguenti equazioni senza l'ausilio della calcolatrice:  }
	
		\begin{parts}
			\part
			\exonly{$7^{x+6}=7^{3x-4}$ }
	
	
			\solonly{$S=\left\lbrace 5\right\rbrace $
				\begin{profonly}Src: Swok pg 317 ex 1	\end{profonly}
			}
	
	
			\part
			\exonly{$3^{2x+3}=3^{(x^2)}$ }
			\solonly{$S=\left\lbrace -1;3\right\rbrace $
				\begin{profonly}Src: Swok pg 317 ex 3	\end{profonly}
			}
	
			\part
			\exonly{$2^{-100x}=(\num{0.5})^{x-4}$ }
			\solonly{$S=\left\lbrace -\frac{4}{99}\right\rbrace $
				\begin{profonly}Src: Swok pg 317 ex 5	\end{profonly}
			}
	
			\solonly{\begin{profonly} Evtl :
					Src: Swok pg 317 ex 2
					Src: Swok pg 317 ex 4
					Src: Swok pg 317 ex 6
					Src: Swok pg 317 ex 7
					Src: Swok pg 317 ex 8
				\end{profonly}}
		\end{parts}
	\end{qblock}


\end{questions}

\subsection{Funzioni logaritmiche}
\begin{questions}


	
	% \begin{qblock}
		\question
		\exonly{Rappresentare graficamente:}
	
	
		\begin{parts}
			\part
			\exonly{$f(x)=\log_4(x)$}
	
			\ifprintanswers
				\begin{tikzpicture}[baseline={($(current bounding box.north)-(0,1.6ex)$)}]
					\begin{axis}[
							AxisDefaults,
							SmallAxisLabels,
							width=\linewidth,
						]
						\addplot[draw=red,samples=1000,domain=0:1]{log2(x)/log2(4)} ;
						\addplot[draw=red,samples=100,domain=1:6]{log2(x)/log2(4)};
					\end{axis}
				\end{tikzpicture}
			\fi
	
	
			\part
			\exonly{$f(x)=\log_4(x+2)$}
			\ifprintanswers
				\begin{tikzpicture}[baseline={($(current bounding box.north)-(0,1.6ex)$)}]
					\begin{axis}[
							AxisDefaults,
							SmallAxisLabels,
							width=\linewidth,
						]
						\addplot[draw=red,samples=1000,domain=-2:-1,restrict y to domain=-5:0]{log2(x+2)/log2(4)} ;
						\addplot[draw=red,samples=100,domain=-1:4]{log2(x+2)/log2(4)};
					\end{axis}
				\end{tikzpicture}
			\fi
			%	\columnbreak
	
			\part
			\exonly{$f(x)=\log_4(x)+2$}
	
	
			\ifprintanswers
				\begin{tikzpicture}[baseline={($(current bounding box.north)-(0,1.6ex)$)}]
					\begin{axis}[
							AxisDefaults,
							SmallAxisLabels,
							width=\linewidth,
						]
						\addplot[draw=red,samples=1000,domain=0:1]{log2(x)/log2(4)+2} ;
						\addplot[draw=red,samples=100,domain=1:6]{log2(x)/log2(4)+2};
					\end{axis}
				\end{tikzpicture}
			\fi
	
			\part
			\exonly{$f(x)=\log_2(x-1)+1$}
	
	
			\ifprintanswers
				\begin{tikzpicture}[baseline={($(current bounding box.north)-(0,1.6ex)$)}]
					\begin{axis}[
							AxisDefaults,
							SmallAxisLabels,
							width=\linewidth,
							ymin=-3,ymax=3,xmin=0,xmax=4
						]
						\addplot[draw=red,samples=1000,domain=1:2]{log2(x-1)+1} ;
						\addplot[draw=red,samples=100,domain=2:4]{log2(x-1)+1};
					\end{axis}
				\end{tikzpicture}
			\fi
	
		\end{parts}
	% \end{qblock}

	
	\begin{qblock}
		\question
		\exonly{Determinare le funzioni $f1(x)$, $f_2(x)$, $f_3(x)$ e $f_4(x)$}
	
		\ifprintanswers \else
			\begin{tikzpicture}[baseline={($(current bounding box.north)-(0,1.6ex)$)}]
				\begin{axis}[
						AxisDefaults,
						SmallAxisLabels,
						width=0.7\linewidth,
						ymin=-6,ymax=6,xmin=-6,xmax=6]
					\addplot[draw=red,samples=500,domain=0:1]{log2(x)/log2(3)};
					\addplot[draw=red,samples=100,domain=1:6]{(log2(x)/log2(3))} node[pos=0.5,above] {$f_1$} ;
					\addplot[draw=red,samples=500,domain=2:3]{(log2(x-2))}node[pos=0.8,below right] {$f_2$} ;
					\addplot[draw=red,samples=100,domain=3:6]{log2(x-2)}  ;
					\addplot[draw=red,samples=500,domain=-1:0]{(log2(-x))} ;
					\addplot[draw=red,samples=100,domain=-6:-1]{log2(-x)} node[pos=0.1,above] {$f_3$} ;
					\addplot[draw=red,samples=500,domain=1:2]{(log2(x-1)+3)} ;
					\addplot[draw=red,samples=100,domain=2:6]{(log2(x-1)+3)} node[pos=0.8,below] {$f_4$} ;
				\end{axis}
			\end{tikzpicture}
		\fi
	
		\solonly{$f_1(x)=\log_3(x)$
	
			$f_2(x)=\log_2(x-2)$
	
			$f_3(x)=\log_2(-x)$
	
			$f_4(x)=\log_2(x-1)+3$}
	
		\solonly{\begin{profonly}
				SRC: Swokowski pg 343 ex 37 et suivants
			\end{profonly}}
	\end{qblock}


	
	\begin{qblock}
		\question
		\exonly{Determinare le funzioni seguenti}
	
		\ifprintanswers \else
			\begin{tikzpicture}[baseline={($(current bounding box.north)-(0,1.6ex)$)}]
				\begin{axis}[
						AxisDefaults,
						SmallAxisLabels,
						width=\linewidth,
						ymode=log,
						ymin=1/10,
						ymax=1000,
						ytick distance =10,
						xmin=-5,xmax=10,
						domain=-5:10,
					]
					\addplot[draw=red,smooth]{2^x} node[above left,pos=0.5] {$f$};
					\addplot[draw=blue,smooth,domain=-5:3]{3^(-x)} node[above right,pos=0.2] {$g$};
					\addplot[draw=orange,smooth]{2*x+10} node[above left,pos=0.5] {$h$};
				\end{axis}
			\end{tikzpicture}
		\fi
	
		\solonly{$f(x)=2^x$, $g(x)=3^{-x}$, $h(x)=2x+10$}
	\end{qblock}

\end{questions}

\subsection{Forma esponenziale e forma logaritmica}

\begin{questions}
	
	\begin{qblock}
		\question
		\exonly{Esprimere in forma logaritmica:}
	
	
		
		\begin{multicols}{2}
			\begin{parts}
		
				\part
				\exonly{$3^5=243$}
				\solonly{$\log_3(243)=5$}
		
				\part
				\exonly{$3^{-4}=\frac{1}{81}$}
				\solonly{$\log_3(\frac{1}{81})=-4$}
		
				\part
				\exonly{$c^p=d$}
				\solonly{$\log_c(d)=p$}
		
				\part
				\exonly{$7^x=100p$}
				\solonly{$\log_7(100p)=x$}
		
				\part
				\exonly{$3^{-2x}=\frac{P}{F}$}
				\solonly{$\log_3(\frac{P}{F})=-2x$}
		
				\part
				\exonly{$\num{0.9}^t=\frac{1}{2}$}
				\solonly{$\log_{\num{0.9}}(\frac{1}{2})=t$}
		
			\end{parts}
		
		\end{multicols}
		\solonly{\begin{profonly}
				Swokowski pg 342 ex 2
			\end{profonly}}
	\end{qblock}




	
	\begin{qblock}
		\question
		\exonly{Esprimere in forma esponenziale}
	
		
		\begin{multicols}{2}
			\begin{parts}
		
				\part
				\exonly{$\log_3(81)=4$}
				\solonly{$3^4=81$}
		
				\part
				\exonly{$\log_4(\frac{1}{256})=-4$}
				\solonly{$4^{-4}=\frac{1}{256}$}
		
				\part
				\exonly{$\log_v(w)=t$}
				\solonly{$v^t=w$}
		
				\part
				\exonly{$\log_6(2x-1)=3$}
				\solonly{$6^3=2x-1$}
		
				\part
				\exonly{$\log_4(p)=5-x$}
				\solonly{$4^{5-x}=p$}
		
				\part
				\exonly{$\log_{a}(343)=\frac{3}{4}$}
				\solonly{$a^{\frac{3}{4}}=343$}
		
			\end{parts}
		\end{multicols}
	
		\solonly{
			\begin{profonly}
				Swokowski pg 342 ex 4
			\end{profonly}
		}
	\end{qblock}

\end{questions}


\subsection{Regole di calcolo dei logaritmi}

\begin{questions}

	
	\begin{qblock}
		\question
		\exonly{Calcolare senza l'ausilio della calcolatrice}
	
	
		
		\begin{multicols}{2}
			\begin{parts}
				\part
				\exonly{$\log_8(1)$}
				\solonly{$0$}
		
				\part
				\exonly{$\log_9(9)$}
				\solonly{$1$}
		
				\part
				\exonly{$\log_5(0)$}
				\solonly{$\notin \R$}
		
				\part
				\exonly{$\log_6(6^7)$}
				\solonly{$7$}
		
				\part
				\exonly{$5^{\log_5(4)}$}
				\solonly{$4$}
				\part
				\exonly{$\log_2(128)$}
				\solonly{$7$}
			\end{parts}
		\end{multicols}
	
	
	
		\solonly{\begin{profonly}SRC:Swokowski pg 342 ex 14	\end{profonly}	}
	
		\solonly{\begin{profonly}EVTL: Swokowski pg 342 ex 15	\end{profonly}	}
	\end{qblock}



	
	\begin{qblock}
		\question
		\exonly{Calcolare senza l'ausilio della calcolatrice}
	
	
		
		\begin{multicols}{2}
			\begin{parts}
				\part \exonly{$\log_5(1)$ }
				\solonly{$0$}
				\part
				\exonly{$\log_3(3)$ }
				\solonly{$1$}
				\part
				\exonly{$\log_7(7^2)$}
				\solonly{$2$}
				\part
				\exonly{$\log_2(-4)$ }
				\solonly{$\notin \mathbb{R}$}
				\part
				\exonly{$3^{\log_3(8)}$ }
				\solonly{$8$}
				\part
				\exonly{$\log_5(125)$}
				\solonly{$3$}
				\part
				\exonly{$\log_4\left(\frac{1}{16}\right)$ }
				\solonly{$-2$}
				\part
				\exonly{$\log_2(\sqrt{2})$}
				\solonly{$\frac{1}{2}$}
			\end{parts}
		\end{multicols}
	\end{qblock}

	
	\begin{qblock}
		\question
		\exonly{Determinare, senza l'ausilio della calcolatrice, i valori delle espressioni seguenti:}
	
		
		\begin{multicols}{2}
			\begin{parts}
				\part
				\exonly{$\log 7 + \log \left(\dfrac{1}{7}\right)$}
				\solonly{$0$}
		
				\part
				\exonly{$\log (40) + \log (25) $ }
				\solonly{$3$}
		
				\part
				\exonly{$\log (6000) - \log (60)$ }
				\solonly{$2$}
		
				\part
				\exonly{$3\cdot \log (2) + 3 \cdot \log (5)$}
				\solonly{$3$}
		
			\end{parts}
		\end{multicols}
	\end{qblock}

	
	\begin{qblock}
		\question
		\exonly{Selezionare le affermazioni vere}
	
	
	\begin{multicols}{2}
		
			\begin{checkboxes}
				\choice $\log_b (2x) = 2 \cdot \log_b x$
				\correctchoice $\log_a(u+v) \neq  \log_a u + \log_a v$
				\correctchoice $\log (10^a) =a$
				\correctchoice $\log_a \left(\dfrac{1}{x}\right)=-\log_a (x)$
				\choice $\log(10a) \neq 1 + \log (a)$
				\choice $\log_a(m^3) = m \cdot \log_a (3)$
				\choice $\log (b) + \log (2b) = \log (3b)$
				\correctchoice $\log_b(-x) \neq - \log_b (x)$
				\choice $\log_b (\sqrt{x}) = \sqrt{\log_b (x)}$
				\choice $\dfrac{\log (a)}{\log (b)}= \log \left(\dfrac{a}{b}\right)$
			\end{checkboxes}
	\end{multicols}
	\end{qblock}



\begin{qblock}
		\question
		\exonly{Scomporre le espressioni seguenti:}
	
		
		\begin{multicols}{2}
			\begin{parts}
				\part
				\exonly{$\log \left(\dfrac{1}{x}\right)$}
				\solonly{$- \log (x)$}
		
				\part
				\exonly{$\log \left(\dfrac{1}{k^6}\right)$ }
				\solonly{$-6 \cdot \log (k)$}
		
				\part
				\exonly{$\log \left(\dfrac{2a}{bc}\right)$ }
				\solonly{$\log (2) + \log (a) - \log (b) - \log (c)$}
		
				\part
				\exonly{$\log(3a^2(a+b))$}
				\solonly{$\log (3) + 2 \log (a) + \log (a+b)$}
		
				\part
				\exonly{$\log \left(\sqrt{n^3}\right)$ }
				\solonly{$\dfrac{3}{2}\cdot\log (n)$}
		
				\part
				\exonly{$\log \left(\sqrt{\dfrac{1+x}{1-x}}\right)$ }
				\solonly{$\dfrac{1}{2}\left(   \log(1+x) - \log (1-x)\right) $}
			\end{parts}
		\end{multicols}
	
\end{qblock}



	
	\begin{qblock}
		\question
		\exonly{Riscrivere utilizzando un solo logaritmo:}
	
		
			\begin{parts}
				\part
				\exonly{$\log_3 \left(2z \right)-\log_3 (x)$ }
				\solonly{$\log_3 \left( \dfrac{2z}{x}\right) $}
		
				\part
				\exonly{$5\log_6 (x)-\dfrac{1}{2}\log_6(3x-4)-3\log_6(5x+1)$}
				\solonly{$\log_6 \left(\dfrac{x^5}{\sqrt{3x-4}\cdot (5x+1)^3}\right)$}
		
				\part
				\exonly{	$3\log_2 (x) + \dfrac{1}{2}\log_2 (3-x)-5\log_2(x+3)$}
				\solonly{$\log_2 \left( \dfrac{x^3  \cdot \sqrt{3-x}}{(x+3)^5}\right)$}
		
				\part
				\exonly{$3\log \left(\dfrac{y}{x^2}\right)-3\log (y) + \dfrac{1}{3} \log (x^6 y^3)$}
				\solonly{$\log \left(\dfrac{y}{x^4} \right)$}
		
				\part
				\exonly{$\ln (y^3) + \dfrac{1}{3}\ln (x^3y^6)-5 \ln (y)$ }
				\solonly{$\ln (x)$}
			\end{parts}
	
		\solonly{\begin{profonly}
				EVTL: Swokowski pg 351 ex 1 - 15
			\end{profonly}}
	\end{qblock}


\end{questions}

\subsection{Equazioni esponenziali}

\begin{questions}
	
	\begin{qblock}
		\question
		\exonly{
			Risolvere le equazioni seguenti. Specificare l'insieme delle soluzioni esatte e dare un valore approssimato (al centesimo).  }
	
	
		\begin{multicols}{2}
			\begin{parts}
				\part
				\exonly{$3^{x+4}=2^{1-3x}$ }
				\solonly{$S=\left\lbrace \dfrac{\log (\frac{2}{81})}{\log 24}\right\rbrace $  $\approx -\num{1.16}$
					\begin{profonly}Src: Swok pg 361 ex 11	\end{profonly}
				}
	
	
				\part
				\exonly{$4^{2x+3}=5^{x-2}$ }
				\solonly{$S=\left\lbrace \dfrac{\log (1600)}{\log \frac{5}{16}}\right\rbrace $  $\approx -\num{6.34}$
					\begin{profonly}Src: Swok pg 361 ex 12	\end{profonly}
				}
	
				\part
				\exonly{$2^{2x-3}=5^{x-2}$ }
					\solonly{$S=\left\lbrace \dfrac{\log (\frac{8}{25})}{\log \frac{4}{5}}\right\rbrace $  $\approx \num{5.11}$
						\begin{profonly}Src: Swok pg 361 ex 13	\end{profonly}
					}
	
					\part
					\exonly{$3^{2-3x}=4^{2x+1}$ }
				\solonly{$S=\left\lbrace \dfrac{\log (\frac{9}{4})}{\log 432}\right\rbrace $  $\approx \num{0.13}$
					\begin{profonly}Src: Swok pg 361 ex 14	\end{profonly}
				}
	
				\part
				\exonly{$3^{-x}=8$ }
				\solonly{$S=\left\lbrace -\dfrac{\log 8}{\log 3} \right\rbrace $  $\approx \num{-1.89}$
					\begin{profonly}Src: Swok pg 361 ex 15	\end{profonly}
				}
	
				\part
				\exonly{$3^{-x^2}=5$ }
				\solonly{$S=\emptyset $}
				\begin{profonly}Src: Swok pg 361 ex 16	\end{profonly}
	
				\part
				\exonly{	$5^x-5^{-x}=6$  }
				%\begin{prof}$ x= \log_5(3 + \sqrt{10})=\dfrac{\log(3+\sqrt{10})}{\log 5}$\end{prof}
				\solonly{$S=\left\lbrace \log_5(3 + \sqrt{10})\right\rbrace =\left\lbrace \dfrac{\log(3+\sqrt{10})}{\log 5}\right\rbrace $}
	
				\part
				\exonly{$2^x-6 \cdot 2^{-x}=6$ } %\begin{prof}$x=\dfrac{\log(3 \pm \sqrt{15})}{\log 2}$\end{prof}
				\solonly{$S=\left\lbrace \log_2(3 + \sqrt{15})\right\rbrace=
						\left\lbrace \dfrac{\log(3 + \sqrt{15})}{\log 2}\right\rbrace $}
	
	
			\end{parts}
		\end{multicols}
	\end{qblock}


\end{questions}

\subsection{Equazioni logaritmiche}

\begin{questions}
	
	\begin{qblock}
		\question
		\exonly{Risolvere le equazioni seguenti dopo averne determinato l'insieme di definizione (dominio)}
	
		
		\begin{multicols}{2}
			\begin{parts}
				\part
				\exonly{$\log_6(2x-3)=\log_6(12)-\log_6(3)$}
				\solonly{$\DD=]\frac{3}{2};\infty[ \quad S=\left\lbrace \frac{7}{2} \right\rbrace $}
		
				\part
				\exonly{$\log_4(3x+2)=\log_4(5)+\log_4(3)$}
				\solonly{$\DD=]-\frac{2}{3};\infty[ \quad S=\left\lbrace \frac{13}{3} \right\rbrace $}
		
				\part
				\exonly{$2\log_3(x)=3\log_3(5)$}
				\solonly{$\DD=]0;\infty[ \quad S=\left\lbrace \sqrt{125} \right\rbrace $}
		
				\part
				\exonly{$3\log_2(x)=2\log_2(3)$}
				\solonly{$\DD=]0;\infty[ \quad S=\left\lbrace \sqrt[3]{9} \right\rbrace $}
		
				\part
				\exonly{$\log_5(x-2)=\log_5(3x+7)$}
				\solonly{$\DD=]2;\infty[ \quad S=\emptyset$}
		
		
				\part
				\exonly{$\log_3(x+4)=\log_3(1-x)$}
				\solonly{$\DD=]-4;1[ \quad S=\left\lbrace -\frac{3}{2} \right\rbrace $}
		
				\part
				\exonly{$\log(x)-\log(x+1)=3\log(4)$}
				\solonly{$\DD=]0;\infty[ \quad S=\emptyset$}
		
				\part
				\exonly{$\log_2(x+7)+\log_2(x)=3$}
				\solonly{$\DD=]0;\infty[ \quad S=\left\lbrace 1 \right\rbrace $}
		
				\part
				\exonly{$\log_3(x-2)+\log_3(x-4)=2$}
				\solonly{$\DD=]4;\infty[ \quad S=\left\lbrace \frac{6 + \sqrt{40}}{2} \right\rbrace $}
		
				\part
				\exonly{$\log(57x)=2+\log(x-2)$}
				\solonly{$\DD=]2;\infty[ \quad S=\left\lbrace \frac{200}{43} \right\rbrace $}
		
			\end{parts}
		\end{multicols}
	
		\solonly{\begin{profonly}
				Swokowski pg 351-352 ex 17 - 32
			\end{profonly}}
	\end{qblock}

\end{questions}


\subsection{Equazioni esponenziali e logaritmiche}

\begin{questions}
	
	\begin{qblock}
		\question
		\exonly{Risolvere le equazioni seguenti dopo averne determinato l'insieme di definizione (dominio)}
	
		
		\begin{multicols}{2}
			\begin{parts}
				\part
				\exonly{$2^{-x}=8$ }
				\solonly{$S=\left\lbrace -3\right\rbrace $}
		
				\part
				\exonly{$3^{x+3}=18^{x-3}$  }
				\solonly{	$S=\left\lbrace 3 \cdot \dfrac{\log 54}{\log 6}\right\rbrace $}
		
				\part
				\exonly{$4^{(2^x)}=2$  }
				\solonly{$S=\left\lbrace -1\right\rbrace $}
		
				\part
				\exonly{$\log_x (5) = \log_5 (x)$ }
				%$\log^2 5 = \log^2 x \iff \pm \log 5 = \log x \iff x= 5^{\pm1}$ $x_1=5$ et $x_2=\dfrac{1}{5}$
				\solonly{$S=\left\lbrace 5; \dfrac{1}{5} \right\rbrace $}
		
				\part
				\exonly{$\log(x-2)+\log(x+1)=1$  }
		
				%\begin{prof}$(x+1)(x-2)=10^1 $ donc $x_1=4$ et $x_2=-3\notin ED$\end{prof}
		
				\solonly{$S=\left\lbrace 4 \right\rbrace $}
		
				\part
				\exonly{$2\log (x) - \log(2x+3)=0$  }
		
				%\begin{prof}$x^2-2x-3=0$ donc $x_1=3$ et $x_2=-1\notin ED$\end{prof}
				\solonly{$S=\left\lbrace 3  \right\rbrace $}
		
				\part
				\exonly{$1=\log_2 (x^2) - \log_2 (4-x)$  }
				%\begin{prof}$x^2+2x-8=0$ donc $x_1=-4$ et $x_2=2$\end{prof}
				\solonly{$S=\left\lbrace -4;2\right\rbrace $}
				\part
				\exonly{	$
						\begin{cases}
							x+y=15 \\
							\log x + \log (-y)=2
						\end{cases}$ }
				%\begin{prof}$y= -\dfrac{100}{x}$ donc $x^2-15x-100=0$ $(10;-5)$ et $(-5;20) \notin ED$\end{prof}
				\solonly{$S=\left\lbrace (20;-5)\right\rbrace $}
				\part
				\exonly{$\log(35-x^3 )=3 \log (5-x)$  }
		
				%\begin{prof}$0=15 \cdot (x+3)(x+2)$ donc $x_1=2$ et $x_2=3$\end{prof}
				\solonly{$S=\left\lbrace 2;3\right\rbrace $}
		
				\part
				\exonly{$x^{6+\log (x)}=10^{-8}$  }
				%\begin{prof}$6\log x +\log^2 x = -8$ donc $y=\log x$ $x_1=10^{-4}$ et $x_2=10^{-2}$\end{prof}
				\solonly{$S=\left\lbrace 10^{-4};10^{-2}\right\rbrace $}
		
		
				%	\part
				%	$\log_y \sqrt{3}=\sqrt{2}$ \begin{prof}$y= 3^{\frac{1}{2 \sqrt{2}}}$\end{prof}
				%	
		
			\end{parts}
		\end{multicols}
	\end{qblock}



\end{questions}
\solnewpage
\subsection{Applicazioni, modellizzazione}


\begin{questions}
	
	\begin{qblock}
		\question
		\exonly{
			La crescita in altezza di un albero é frequentemente descritta con una funzione logistica.
			Supponiamo che l'altezza $h$ (in metri) di un albero di $t$ anni sia descritta dalla funzione
	
			$$h(t)=\dfrac{36}{1+200 e^{-\num{0.2}t}} $$}
	
		\begin{parts}
			\part
			\exonly{
				Quale sarà l'altezza di un albero di $10$ anni?
			}
			\solonly{\SI{1.28}{\meter}}
	
			\part
			\exonly{
				A che età la sua altezza sarà di \SI{15}{\metre}?}
	
			\solonly{A \num{24.8} anni}
		\end{parts}
	\end{qblock}

	
	\begin{qblock}
		\question
	
		\exonly{
			La massa di un elefante africano (in  \SI{}{\kilogram}) in funzione dell’età (in anni) é data approssimativamente dalla formula
	
			$$m(t)=2600\cdot\left(1-0.51\cdot e^{-0.075\cdot t}\right)^3$$}
	
		\begin{parts}
			\part
			\exonly{Determinare la massa di un elefante alla nascita}
			\solonly{\SI{305.9}{\kilogram}}
	
			\part
			\exonly{
				Determinare l’età di un elefante di  \SI{1800}{\kilogram}}
			\solonly{\num{19.82} ans}
		\end{parts}
	\end{qblock}
	
	
	\begin{qblock}
		\question
		\exonly{
			La popolazione (in milioni di abitanti) degli Stati Uniti $t$ anni dopo il 1980 può essere approssimata dalla funzione
	
			$$N(t)=227\cdot e^{0.007\cdot t}.$$
	
			In quale lasso di tempo la popolazione raddoppia?
			In quale anno la popolazione sarà raddoppiata rispetto al 1980?
		}
		\solonly{
			\begin{profonly}
				$2=e^{0.007\cdot t} \iff \ln 2 = 0.007 \cdot t \iff t=\dfrac{\ln 2}{0.007}\approx 99$
			\end{profonly}
	
			Dopo 99 anni. Nel 2079 }
	\end{qblock}

	
	\begin{qblock}
		\question
		\exonly{
			Varie masse vengono sospese ad una molla di  \SI{70}{\centi\meter} . Le lunghezze corrispondenti della molla vengono  misurate e notate nella tabella  qui sotto.
	
		
		\begin{center}
				\begin{Tabular}[1.5]{|l|c|c|c|c|}
					\hline
					Massa in \si{\kilo\gram} & \num{0} & \num{0.5} & \num{1} & \num{2.5} \\
					\hline
					Lunghezza in  \si{\centi\meter} & \num{70} & \num{71.5} & \num{73} & \num{77.5} \\
					\hline
				\end{Tabular}
		\end{center}
	
			Vogliamo modellizzare la funzione $L(x)$ che ci darà la lunghezza della molla (in \SI{}{\centi\meter}) in funzione della massa sospesa (in \SI{}{\kilogram}). Con che tipo di modello abbiamo a che fare? Lineare (o affine) o esponenziale?
	
			Quale lunghezza possiamo prevedere se sospendiamo una massa di \SI{4}{\kilogram} ?
		}
	
		\solonly{
			Modello lineare. $L(4)=\SI{82}{\centi\meter}$
			\begin{profonly}
				Pour chaque augmentation de $\frac{1}{2}$ de $kg$ on augmente la longueur de  $\num{1.5}$ $cm$
	
				$L(x)= \dfrac{\frac{3}{2}}{\frac{1}{2}}x + 70 = 3x+70$
	
				$L(4)=82$ $cm$
			\end{profonly}
		}
	\end{qblock}


	
	\begin{qblock}
		\question
		\exonly{
			Una tazza di caffè viene servita in un salone la cui temperatura é di
			\SI{20}{\degreeCelsius}.
			Una persona decide di misurare la temperatura del caffè ogni \SI{4}{\minute}.
			I dati sono stati raccolti nella tabella qui sotto:
	
	
	\begin{center}
				\begin{Tabular}[1.5]{|l|c|c|c|c|}
					\hline
					Tempo in  $\SI{}{\minute}$ & \num{0} & \num{4} & \num{8} & \num{12} \\
					\hline
					Temperatura in \SI{}{\degreeCelsius} & \num{95} & \num{80.75} & \num{69.2} & \num{59.9} \\
					\hline
				\end{Tabular}
	\end{center}
	
			Vogliamo modellizzare la funzione $T(t)$, la temperatura del caffè in \SI{}{\degreeCelsius}, in funzione del tempo trascorso $t$ in \SI{}{\minute}.
	
			Si tratta di un modello lineare (o affine) o esponenziale?
	
			Quale sarà la temperatura del caffè dopo \SI{24}{\minute}?
		}
	
		\solonly{	Modello Esponenziale $T(24)=\SI{41.18}{\degreeCelsius}$ }
	
	
		\begin{profonly}
			Cela ne peut pas être linéaire puisque pour une variation de  $\SI{4}{\minute}$ on à des diminutions différentes de température : entre $0$ et $4$ diminution $\SI{14.2}{\degreeCelsius}$ mais entre $4$ et $8$ diminution de $\SI{12.1}{\degreeCelsius}$  et entre $8$ et $12$ de $\SI{10.3}{\degreeCelsius}$ .
	
			Le modèle est exponentiel puisque le facteur de décroissance est constant:
	
			$0.81 \approx \dfrac{60.8}{75}=\dfrac{48.7}{60.8}$
	
			$T(t)= 75 \cdot 0.81^\frac{t}{4}+20$  les élèves auront probablement $T'(t)=95 \cdot 0.85^\frac{t}{4}$
	
			$T(24)=\SI{41.2}{\degreeCelsius}$
	
			ou $T'(24)=\SI{35.82}{\degreeCelsius}$
	
			Discuter du problème de l'asymptote.
		\end{profonly}
	\end{qblock}

	
	\begin{qblock}
		\question
		\exonly{
			Uno stagno contiene 1000 trote.
			Novanta giorni più tardi se ne stimano 600.
			Determinare il modello esponenziale $q(t)$ utilizzabile per stimare il numero di trote dopo $t$ giorni.
	
			In seguito determinare dopo quanti giorni resteranno solo 100 trote.
		}
	
		\solonly{Dopo  406 giorni circa}
	\end{qblock}


	
	\begin{qblock}
		\question
		\exonly{
			Il numero di batteri in una coltura cresce esponenzialmente. L'esperimento inizia in laboratorio al tempo $t=0$.
			Trenta minuti dopo l'inizio dell'esperienza si registrano 1500 batteri e quaranta minuti dopo l'inizio dell'esperimento se ne stimano 2000.
		}
	
		\begin{parts}
			\part
	
			\exonly{Stabilire il modello $P(t)$ che stima il numero di batteri $t$ minuti dopo l'inizio dell'esperimento. }
			\solonly{$P(t)=1500 \left( \frac{4}{3}\right) ^{\frac{t-30}{10}}$ }
			\part
			\exonly{
				Quanti batteri c'erano inizialmente (al $t=0$)?
			}
	
			\solonly{
				\begin{profonly}
					$632.81=\dfrac{1500}{(\frac{4}{3})^3}$
				\end{profonly}
				$632.81\approx 633$ batteri.
			}
	
			\part
			\exonly{
				Quanto tempo é necessario per raddoppiare la popolazione?
			}
	
			\solonly{\SI{24.1}{\minute} }
		\end{parts}
	
	\end{qblock}


	
	\begin{qblock}
		\question
		\exonly{
			Il \DTMDate{2000-01-01} la popolazione della città $A$ era di $\num{62500}$ abitanti con un tasso di decrescita del $2\%$ annuo.
	
			La popolazione della città $B$ era di $\num{45070}$ il \DTMDate{1995-01-01} e di
			$\num{53530}$ il \DTMDate{2000-01-01}.
	
			I tassi di crescita e decrescita delle popolazioni sono supposti costanti nel tempo.
		}
	
		\begin{parts}
			\part
			\exonly{
				Determinare le funzioni $A(t)$ e $B(t)$ rappresentanti il numero di abitanti de lle due città in funzione del tempo $t$ misurato in anni ($t=0$ il \DTMDate{2000-01-01}).
			}
	
	
			\solonly{
			$A(t)=62500 \cdot \num{0.98}^t$ \\
			$B(t)=53530 \cdot \num{1.19}^{\sfrac{t}{5}}$
			}
	
	
			\part
			\exonly{
				Quale sarà il numero di abitanti della città $A$ il \DTMDate{2050-01-01}? E per la città $B$?
			}
	
			\solonly{
				$A(50)\approx \num{22760}$\\
				$B(50)\approx \num{304837}$
			}
	
	
	
			\part
			\exonly{
				In quale momento le due popolazioni saranno identiche?
			}
	
			\solonly{
				Settembre 2002
				\begin{profonly}
					$t=2.81$
				\end{profonly}
			}
	
		\end{parts}
	\end{qblock}


	
	\begin{qblock}
		\question
		\exonly{
			Si é osservata una diminuzione del 7\% della superficie di una foresta ogni 3 anni.
	
			Oggi la superficie é di  $\num{250000} m^2$.
		}
	
		\begin{parts}
			\part
			\exonly{
				Determinare il modello esponenziale  $S(t)$ rappresentante la superficie della foresta in funzione del tempo $t$ in anni.}
	
			\solonly{$S(t)=250000 \cdot (1-0.07)^{\sfrac{t}{3}}$}
	
			\part
			\exonly{
				Quale sarà la superficie della foresta tra 7 anni?
			}
	
			\solonly{$S(7)\approx \num{211057}$}
	
			\part
			\exonly{
				Quale superficie occupava la foresta 5 anni fa?
			}
	
			\solonly{$S(-5)\approx \num{282142}$}
	
			\part
			\exonly{
				Tra quanti anno  resteranno solo  $10000 m^2$ di foresta?}
	
			\solonly{Tra  \num{133} anni circa}
	
		\end{parts}
	\end{qblock}

	
	\begin{qblock}
		\question
		\exonly{
			In un paese si é constatata la diminuzione del potere di acquisto del 20\% ogni 4 anni circa.
		}
	
		\begin{parts}
			\part
			\exonly{
				Modellizzare questa evoluzione usando una funzione del tipo
				$P(t)=P_0 \cdot \alpha^{t/T}$
			}
	
			\solonly{$P(t)=P_0 \cdot 0.8^{\sfrac{t}{4}}$}
	
			\part
			\exonly{
				Se definiamo un potere d'acquisto di riferimento al 100\% ad una certa data, quale sarà il potere d'acquisto dopo 10 anni?
			}
	
			\solonly{$P(10)=57\%$}
	
		\end{parts}
	\end{qblock}


	
	\begin{qblock}
		\question
		\exonly{
			Un capitale di  \SI{35000}{\CHF} viene depositato ad un tasso del  $\num{2.75}\%$ annuale (interesse composto, capitalizzato annualmente).
	
			Qual'é stata la durata del deposito a queste condizioni se il capitale finale é di \SI{44679.11}{\CHF}?}
	
		\solonly{Approssimativamente  $9$ anni}
	\end{qblock}



	
	\begin{qblock}
		\question
		\exonly{
			Un nostro avo ha depositato la modica somma di  \SI{1}{\CHF} su di un conto ad un tasso di interesse del $5\%$ capitalizzato annualmente.
			Oggi ritiriamo la somma depositata sotto forma di lingotti d'oro da \SI{1}{\kilogram}.
	
			Determinare il numero di lingotti che riceveremmo sapendo che il deposito risale a $500$ anni fa e stimando il prezzo attuale di \SI{1}{\kilogram} d'oro a \SI{25000}{\CHF}
		}
	
		\solonly{\num{1572930} lingotti }
	\end{qblock}


	\solonly{\begin{profonly}
			Note: : La plus grande réserve d’or du monde, la «Federal Reserve Bank» de New York dispose de 9 millions de kg d'or.
		\end{profonly}}




\begin{qblock}
		\question
		\exonly{
			Un maglio colpisce un pezzo di metallo di uno spessore iniziale di un centimetro con una frequenza di 10 colpi al minuto.
			Ad ogni colpo lo spessore del pezzo diminuisce dell'1\%.
	
			Quanti colpi saranno necessari per ridurre lo spessore di un quarto?
	
			Il pezzo é considerato finito se il suo spessore non eccede i 5 millimetri.
			Quanto tempo sarà necessario ad ottenere un pezzo finito?
		}
		\solonly{
			29 colpi \\ 414 \si{\second}
		}
\end{qblock}

	
	\begin{qblock}
		\question
		\exonly{Il tempo di dimezzamento (o emivita) dell'isotopo radioattivo $239$ del Plutonio (${}^{239}Pu$) è di $24000$ anni. }
	
		\begin{parts}
			\part
			\exonly{
				Dopo $2000$ anni quanto rimarrà di un campione iniziale di \SI{100}{\gram}?
			}
	
			\solonly{
			$Q(t)=100 \cdot \left(\dfrac{1}{2}\right)^{\dfrac{t}{24000}}$
	
			$Q(2000)=\SI{94.4}{\gram}$}
	
	
			\part
	
			\exonly{Quanto tempo è necessario per passare da \SI{100}{\gram} a \SI{10}{\gram} di ${}^{239}Pu$. }
			\solonly{$\approx\num{79726.3}$ anni}
	
			\part
			\exonly{Definire un modello $T(x)$ che rappresenti il tempo (in anni) necessario affinché rimanga solo una percentuale  $x$  di ${}^{239}Pu$}
	
			\solonly{$T(x)=-\dfrac{24000}{\log(2)}\log\left( \dfrac{x}{100}\right)$}
	
		\end{parts}
	\end{qblock}


	
	\begin{qblock}
		\question
		\exonly{
			In acustica si definisce il livello di intensità acustica ($D$) come rapporto in \si{\deci\bel}
			fra il flusso di energia $I$ e il flusso $I_0$ della soglia di udibilità, pari a $10^{-12} \text{W/m}^2$.
	
			$$D=10\cdot \log\left(\frac{I}{I_0}\right)$$
		}
	
		\begin{parts}
			\part
			\exonly{Una conversazione normale ha un livello di intensità acustica di \SI{60}{\deci\bel}. Calcolare il flusso di energia $I$ in $\text{W/m}^2$ }
			\solonly{$10^{-6}$ $\text{W/m}^2$ }
	
			\part
			\exonly{Misuriamo due suoni con intenistà sonore $D_1$ e $D_2$. Qual'è il rapporto tra i due flussi di energia  $I_1$ e $I_2$ misurati se:}
			\begin{subparts}
				\subpart
				\exonly{$D_2 = 30 + D_1$ }
				\solonly{$\frac{I_2}{I_1}=10^3=1000$ }
				%			
				\subpart
				\exonly{$D_2 = 7 + D_1$ }
				\solonly{$\frac{I_2}{I_1}=10^{\num{0.7}}\approx 5.02$ }
			\end{subparts}
	
	
	
		\end{parts}
	\end{qblock}

	
	\begin{qblock}
		\question
	
		\exonly{
			Secondo la legge di raffreddamento di Newton, la velocità di raffreddamento é proporzionale alla differenza di temperature tra l'oggetto e l'ambiente in cui si trova.
	
			Un ferro da stiro passa da \SI{52}{\celsius} a \SI{38}{\degreeCelsius} in \SI{30}{\minute} in una stanza ad una temperatura  constante di \SI{24}{\celsius}.
	
			La temperatura $T$ del ferro dopo  $t$ ore é modellizzata da
			$$ T(t)=28\cdot 2^{-2t}+24$$
		}
	
		\begin{parts}
	
			\part \exonly{Supponendo che $t=0$ corrisponda alle ore  $13:00$, determinare, con una precisione di un decimo di grado, la temperatura alle ore  14h00, 15h30 e 16h00 }
			\solonly{$T(1)=\SI{31}{\degreeCelsius}$ \\ $T(\num{2.5})=\SI{24.875}{\degreeCelsius}$ \\
				$T(3)=\SI{24.4}{\degreeCelsius}$
			}
	
			\part
			\exonly{Rappresentare graficamente  $T(t)$ nell'intervallo  $0\leq t \leq 4$}
	
			\ifprintanswers
				\begin{tikzpicture}[baseline={($(current bounding box.north)-(0,1.6ex)$)}]
					\begin{axis}[
							AxisDefaults,
							SmallAxisLabels,
							width=\linewidth,
							ymin=0,
							ymax=52,
							ytick distance={10},
							xlabel=$t$
							%sytick={0,10,...,24,...,52}
						]
						\addplot[draw=red,smooth,unbounded coords=jump,domain=-1:4,restrict y to domain=0:52]{28*2^(-2*x)+24};
						\draw[dashed] (0,24) -- (4,24);
					\end{axis}
				\end{tikzpicture}
			\fi
	
			\solonly{
	
	
				\begin{profonly}
					Src: Swokowski pg 318 ex 24
				\end{profonly}}
	
	
		\end{parts}
	\end{qblock}

	
	\begin{qblock}
		\question
		\exonly{
			Il numero di batteri di una cultura passa da  $600$ a $1800$ tra le  $7h$ e le  $9h$ del mattino.
			Supponendo una crescita esponenziale, la popolazione di batteri   $t$ ore dopo le  $7h$ del mattino é data dal modello
	
			\[P(t)=600 \cdot 3^{\sfrac{t}{2}}\]
		}
	
	
	
	
		\begin{parts}
			\part \exonly{Determinare approssimativamente il numero di batteri alle  $8h$ e alle  $11h$ del mattino. }
			\solonly{$P(1)\approx 1039$ , $P(4)=5400$}
	
			\part
			\exonly{Rappresentare graficamente  $P(t)$ per $0\leq t \leq 4$}
	
			\ifprintanswers
				\begin{tikzpicture}[baseline={($(current bounding box.north)-(0,1.6ex)$)}]
					\begin{axis}[
							AxisDefaults,
							SmallAxisLabels,
							width=\linewidth,
							xlabel=$t$,
							ymin=0,
							ytick distance={500},
							%sytick={0,10,...,24,...,52}
						]
						\addplot[draw=red,smooth,unbounded coords=jump,domain=0:4,restrict y to domain=0:6000]{600*3^(x/2)};
					\end{axis}
				\end{tikzpicture}
			\fi
	
			\solonly{
	
	
				\begin{profonly}
					Src: Swokowski pg 318 ex 23
				\end{profonly}}
	
			\part
			\exonly{Dopo quante ore ci saranno circa $5400$ batteri? }
			\solonly{Dopo circa 4 ore }
	
		\end{parts}
	\end{qblock}

\end{questions}