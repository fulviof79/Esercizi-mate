%%%%%%%%%%%%%%%%%%%%%%%%%%%%%%%%%%%%%%%%%%%%%%%%%%%%%%%%%%%%%%%%%%
\question 
\exonly{
Trois solutions contiennent un certain acide. La première contient 10\% d'acide, la deuxième 30\% et la troisième 50\%. 
Un chimiste aimerait utiliser les trois solutions pour obtenir 50 litres d'un mélange contenant 32\% d'acide. 
S'il veut utiliser deux fois plus de solution à 50\% que de solution à 30\%, combien de litres de chaque solution devrait-il utiliser ?
}

\solonly{
\SI{17}{\litre}  à 10\%, \SI{11}{\litre} à 30\% et \SI{22}{\litre}  à 50\%

}