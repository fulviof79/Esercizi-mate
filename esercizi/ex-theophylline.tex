%%%%%%%%%%%%%%%%%%%%%%%%%%%%%%%%%%%%%%%%%%%%%%%%%%%%%%%%%%%%%%%%%%
\question \exonly{

La théophylline, médicament contre l'asthme, est préparée à partir d'un élixir contenant une concentration de \SI{5}{\milli\gram/\milli\litre} d'un produit et d'un sirop parfumé à la cerise pour masquer le goût du produit. 

Combien de chaque ingrédient doit-on utiliser pour préparer \SI{100}{\milli\litre} de solution avec une concentration de \SI{2}{\milli\gram/\milli\litre} ?

}
\solonly{
\SI{40}{\milli\litre} d'élixir et \SI{60}{\milli\litre} de sirop
}





