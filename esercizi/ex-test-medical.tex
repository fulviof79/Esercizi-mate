%%%%%%%%%%%%%%%%%%%%%%%%%%%%%%%%%%%%%%%%%%%%%%%%%%%%%%%%%%%%%%%%%%
\question
\exonly{
Nell'ambito di un test medico che mira a misurare la tolleranza ai carboidrati si somministrano ad un adulto \SI{7}{\centi\litre}  di una soluzione con una concentrazione del 30\% di glucosio.

Una tale concentrazione non é applicabile nel caso debba essere somministrata ad un bambino. Per un bambino bisognerebbe somministrare \SI{7}{\centi\litre} ad una concentrazione del 20\%.

Quanta soluzione per adulti bisognerà diluire per ottenere tale scopo?



}
\solonly{
$\dfrac{14}{3}\approx \SI{4.67}{\centi\litre}$ de soluzione e $\dfrac{7}{3} \approx \SI{2.33}{\centi\litre}$ d'acqua.
}





