%%%%%%%%%%%%%%%%%%%%%%%%%%%%%%%%%%%%%%%%%%%%%%%%%%%%%%%%%%%%%%%%%%
\question \exonly{
Un corridore comincia  un percorso di allenamento e corre alla velocità costante di \SI{9.7}{\kilo\meter/\hour}.
Cinque minuti più tardi un secondo corridore comincia lo stesso percorso ma corre alla velocità costante di  \SI{12.9}{\kilo\meter/\hour}.

Quanto tempo occorre al secondo corridore per raggiungere il primo?

%Dopo quanto tempo il secondo corridore raggiungerà il primo?
%
%Un coureur part du début d'un parcours d'entraînement et court à la vitesse constante de \SI{9.7}{\kilo\meter/\hour}. 
%Cinq minutes plus tard, un autre coureur part du même point, il court à \SI{12.9}{\kilo\meter/\hour} et suit le même chemin. 
%
%Combien de temps faudra-t-il au second coureur pour rattraper le premier ? 

}
\solonly{
Il seocndo corridore raggiungerà il primo dopo aver corso ca. 15 minuti.
%Le second coureur nécessitera ca. 20 minutes pour rattraper le premier.
}





