
%!TeX spellcheck = it_IT
%!TEX encoding = UTF-8 Unicode
%Author: Fulvio Frapolli
%Last revision: 06.09.2019
\documentclass[
%todo,
%answers,
finale,
%sectnum,
ssectnum,
%twocolumn,
]{DossierExMathIta}


\usepackage{pgf-pie}
\titolo[indice]{Matematica}
%\titolo{PAP1}
\author{CPT}
\date{2020}


%\pgfplotsset{AxisDefaults/.append style={width=\linewidth,}}
\tikzstyle{every node}=[font=\footnotesize] 
%\partlabel{\arabic{partno})} %% Keep consistency in part numbering  with the screenshots

\makeatletter
\def\input@path{{esercizi/}{temi/}}
%or: \def\input@path{{/path/to/folder/}{/path/to/other/folder/}}
\makeatother





\begin{document}
\begin{questions}
\question
	
Determinare il modello $f$ che rappresenti al meglio i dati rappresentati qui sotto.



	\begin{center}
		\begin{tikzpicture}
			\begin{axis}[
					AxisDefaults,
					SmallAxisLabels,
					height=10cm, %height and width of the graph
					width=0.8\linewidth,
					restrict y to domain= 0:11,
					ymin=-1, ymax=10, %set the min and max y tick
					xmin=-3
				]

				\addplot[draw=red,  only marks] coordinates {
					(-3.00,24.00) (-2.00,12.00) (-1.00,6.00) (0.00,3.00) (1.00,1.50) (2.00,0.75) (3.00,0.38) (4.00,0.19) (5.00,0.09) (6.00,0.05) (7.00,0.02) (8.00,0.01) (9.00,0.01) 
					};
				\addplot[mark=none] coordinates {(-1,6)} node[pin=170:{$(-1;6)$}]{} ;
				\addplot[mark=none] coordinates {(0,3)} node[pin=10:{$(0;3)$}]{} ;
			\end{axis}
		\end{tikzpicture}
	\end{center}


	\question
	
Determinare il modello $f$ che rappresenti al meglio i dati rappresentati qui sotto.



	\begin{center}
		\begin{tikzpicture}
			\begin{axis}[
					AxisDefaults,
					SmallAxisLabels,
					height=10cm, %height and width of the graph
					width=0.8\linewidth,
					restrict y to domain= 0:30,
					ymin=-1, ymax=30, %set the min and max y tick
					xmin=-3,
					ytick distance=5,
				]

				\addplot[draw=red,  only marks] coordinates {
					(-3.00,0.05) (-2.00,0.09) (-1.00,0.19) (0.00,0.38) (1.00,0.75) (2.00,1.50) (3.00,3.00) (4.00,6.00) (5.00,12.00) (6.00,24.00) (7.00,48.00) (8.00,96.00) (9.00,192.00)
					};
				\addplot[mark=none] coordinates {(3,3)} node[pin=160:{$(3;3)$}]{} ;
				\addplot[mark=none] coordinates {(5,12)} node[pin=110:{$(5;12)$}]{} ;
			\end{axis}
		\end{tikzpicture}
	\end{center}
	
% \pgfmathsetmacro\radius{4}
% 		\begin{tikzpicture}
% 			\coordinate[label=below right:$O$] (O) at (0,0);
% 			\draw (O) circle (\radius);
% 			\filldraw (O) circle (1pt);
% 			\coordinate (A) at (20:\radius);
% 			\coordinate (B) at (150:\radius);
% 			\coordinate (C) at (250:\radius);

% 			\draw (A) -- (B) -- (C)-- cycle;
% 			\draw (B) -- (O)-- (C);

% 			\draw pic["$\alpha$",draw, angle eccentricity=1.3,angle radius=1cm] {angle=B--A--C};
% 			\draw pic["$\beta$",draw, angle eccentricity=1.3,angle radius=1cm] {angle=A--C--O};
% 			\draw pic["$\gamma$",draw, angle eccentricity=1.3,angle radius=0.6cm] {angle=O--C--B};
% 			\draw pic["$\delta$",draw, angle eccentricity=1.3,angle radius=0.6cm] {angle=C--B--O};
% 			\draw pic["$30^{\circ}$",draw, angle eccentricity=1.4,angle radius=1cm] {angle=O--B--A};
% 			\draw pic["$80^{\circ}$",draw, angle eccentricity=1.4,angle radius=1cm] {angle=B--O--C};
% 		\end{tikzpicture}
\end{questions}
\end{document}